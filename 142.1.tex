\documentclass[10pt,twoside,twocolumn]{article}
\usepackage[latin9]{inputenc}
\usepackage[landscape]{geometry}
\geometry{verbose,tmargin=0.5in,bmargin=0.75in,lmargin=0.5in,rmargin=0.5in}
\setlength{\parskip}{\smallskipamount}
\setlength{\parindent}{0pt}
\usepackage{amsmath}
\usepackage{amssymb}

\makeatletter

\usepackage{amsthm}
\usepackage{mathrsfs}
\usepackage{amsfonts}
\usepackage{dsfont}


\setlength{\columnsep}{0.25in}

\renewcommand*{\thefootnote}{\fnsymbol{footnote}}

\newcommand{\R}[0]{\mathds{R}} % real numbers
\newcommand{\Z}[0]{\mathds{Z}} % integers
\newcommand{\N}[0]{\mathds{N}} % natural numbers
\newcommand{\nat}[0]{\mathds{N}} % natural numbers
\newcommand{\Q}[0]{\mathds{Q}} % rational numbers

\makeatother

\begin{document}

\title{Reference Sheet for CO142.1 Discrete Mathematics I}


\date{Autumn 2016}

\maketitle

\paragraph{Proofs in Discrete Mathematics}
\begin{enumerate}
\item Use Venn diagrams, directed graphs, or another visual representation
to gain an intuition of what needs to be shown.
\item Use definitions to create a logical statement.
\item Use logical arguments to prove the statement.

\begin{enumerate}
\item In general, equivalences from CO140 should be sufficient.
\item If something is false, try to find a simple counterexample.
\item If under a for-all quantifier, consider an arbitrary object.
\item For an if-then statement, assume the antecedent and prove the consequent.
\item For an equality or if-and-only-if, ensure your argument is bidirectional.
\end{enumerate}
\item Use definitions to return to set notation.
\end{enumerate}

\section{Sets}

A \emph{set} is a collection of definite and separate objects.


\paragraph{Russel's Paradox}

The collection $R\triangleq\left\{ X\mbox{ is a set}|X\notin X\right\} $
is not a set. Can be proven by contradiction when considering a set
$R$ (consider the cases $R\in R$ and $R\notin R$).


\paragraph{Comparing Sets}
\begin{enumerate}
\item \emph{Subset}: $A\subseteq B\triangleq\forall x\in A\left(x\in B\right)$.
\item \emph{Equality}: $A=B\triangleq A\subseteq B\land B\subseteq A$.
\end{enumerate}
If $A\subseteq B$ and $B\subseteq C$, then $A\subseteq C$.


\paragraph{Basic Operators}
\begin{enumerate}
\item Union: $A\cup B\triangleq\left\{ x|x\in A\lor x\in B\right\} $.
\item Intersection: $A\cap B\triangleq\left\{ x|x\in A\land x\in B\right\} $.
\item Difference: $A\backslash B\triangleq\left\{ x|x\in A\land x\notin B\right\} $.
\item Symmetric Difference: $A\triangle B\triangleq\left(A\backslash B\right)\cup\left(B\backslash A\right)$.
\end{enumerate}
$A,B\mbox{ are \emph{disjoint}}\triangleq A\cap B=\emptyset$.

To make any union $A\cup B$ disjoint, consider $A\cup\left(B\backslash A\right)$.


\paragraph{Properties of Basic Operators}
\begin{enumerate}
\item \emph{Idempotence}

\begin{enumerate}
\item $A\cup A=A$
\item $A\cap A=A$
\end{enumerate}
\item \emph{Commutativity}

\begin{enumerate}
\item $A\cup B=B\cup A$
\item $A\cap B=B\cap A$
\item $A\triangle B=B\triangle A$
\end{enumerate}
\item \emph{Associativity}

\begin{enumerate}
\item $A\cup\left(B\cup C\right)=\left(A\cup B\right)\cup C$
\item $A\cap\left(B\cap C\right)=\left(A\cap B\right)\cap C$
\end{enumerate}
\item \emph{Empty Set}

\begin{enumerate}
\item $A\cup\emptyset=A$
\item $A\cap\emptyset=\emptyset$
\item $A\triangle A=\emptyset$
\end{enumerate}
\item \emph{Distributivity}

\begin{enumerate}
\item $A\cup\left(B\cap C\right)=\left(A\cup B\right)\cap\left(A\cup C\right)$
\item $A\cap\left(B\cup C\right)=\left(A\cap B\right)\cup\left(A\cap C\right)$
\end{enumerate}
\item \emph{Absorption}

\begin{enumerate}
\item $A\cup\left(A\cap B\right)=A$
\item $A\cap\left(A\cup B\right)=A$
\end{enumerate}
\end{enumerate}

\paragraph{Cardinality}
\begin{enumerate}
\item \emph{Cardinality}: $\left|A\right|$ is defined as the number of
distinct elements contained in $A$.
\item \emph{Principle of Inclusion-Exclusion} (for two sets): $\left|A\cup B\right|=\left|A\right|+\left|B\right|-\left|A\cap B\right|$.
\end{enumerate}

\paragraph{Power Set}
\begin{enumerate}
\item \emph{Power set}: ${\cal P}A\triangleq\left\{ X|X\subseteq A\right\} $.
\item For a finite set $A$ with $\left|A\right|=n$, $\left|{\cal P}A\right|=2^{n}$.
\end{enumerate}

\paragraph{Products}

For arbitrary sets $A$ and $B$:
\begin{enumerate}
\item Ordered pair of elements of $A$ and $B$ is written as $\left\langle a,b\right\rangle $.
\item Cartesian product: $A\times B\triangleq\left\{ \left\langle a,b\right\rangle |a\in A\land b\in B\right\} $.
\item For a finite sets $A$ and $B$, $\left|A\times B\right|=\left|A\right|\times\left|B\right|$.
\item $n$-ary product: $A_{1}\times A_{2}\times\dots\times A_{n}\triangleq\left\{ \left\langle a_{1},a_{2},\dots,a_{n}\right\rangle |\forall1\leq i\leq n\left(a_{i}\in A_{i}\right)\right\} $.
\end{enumerate}

\paragraph{Partitions}

A partition of $S$ is a family $A_{1},A_{2},\dots,A_{n}$ of subsets
$S$ such that:
\begin{enumerate}
\item \noindent $A_{i}$ is not empty: $\forall1\leq i\leq n\left(A_{i}\neq\emptyset\right)$.
\item The $A_{i}$ cover $S$: $S=\cup_{i=1}^{n}A_{i}$.
\item The $A_{i}$ are pairwise disjoint: $\forall1\leq i,j\leq n\left(A_{i}\cap A_{j}=\emptyset\implies i=j\right)$
(or the contrapositive).
\end{enumerate}

\paragraph{Pigeonhole Principle}

If a set of $n$ distinct objects is partitioned into $k$ subsets,
where $0<k<n$, then at least one subset must contain at least two
elements.


\section{Relations}
\begin{enumerate}
\item A \emph{relation} $R$ satisfies $R\subseteq A\times B$. It has \emph{type}
$A\times B$. 
\item A \emph{binary relation} on $A$ has type $A^{2}$. 
\end{enumerate}
Relations can be represented as: 
\begin{enumerate}
\item A subset of a product set. 
\item A diagram with arrows between elements in two sets. 
\item A directed graph, for a binary relation. 
\item A matrix: for $R\subseteq A\times B$, rows are based on $A$ and
columns on $B$. 
\item Special representations, e.g. area on the plane for binary relations
on $\R$. 
\end{enumerate}

\paragraph{Basic Operators}

For $R,S\subseteq A\times B$: 
\begin{enumerate}
\item \emph{Union}: $R\cup S\triangleq\left\{ \left\langle a,b\right\rangle \in A\times B|\left\langle a,b\right\rangle \in R\lor\left\langle a,b\right\rangle \in S\right\} $. 
\item \emph{Intersection}: $R\cap S\triangleq\left\{ \left\langle a,b\right\rangle \in A\times B|\left\langle a,b\right\rangle \in R\land\left\langle a,b\right\rangle \in S\right\} $. 
\item \emph{Complement}: $\overline{R}\triangleq\left\{ \left\langle a,b\right\rangle \in A\times B|\left\langle a,b\right\rangle \not\in R\right\} $. 
\item \emph{Inverse}: $R\cap S\triangleq\left\{ \left\langle b,a\right\rangle \in A\times B|a\:R\:b\right\} $. 
\end{enumerate}

\paragraph{Identity}

$\mbox{id}_{A}=\left\{ \left\langle x,y\right\rangle \in A^{2}|x=y\right\} $.


\paragraph{Composition}

For $R\subseteq A\times B,S\subseteq B\times C$:

$R\circ S\triangleq\left\{ \left\langle a,c\right\rangle \in A\times C|\exists b\in B\left(a\:R\:b\land b\:R\:c\right)\right\} $.


\paragraph{Equivalence Relations}

The binary relation $R$ on $A$ is an \emph{equivalence relation}
when $R$ is reflexive, symmetric, and transitive.
\begin{enumerate}
\item $R\mbox{ is \emph{reflexive}}\triangleq\forall x\in A\left(x\:R\:x\right)$. 
\item $R\mbox{ is \emph{symmetric}}\triangleq\forall x,y\in A\left(x\:R\:y\implies y\:R\:x\right)$. 
\item $R\mbox{ is \emph{transitive}}\triangleq\forall x,z\in A\left(\exists y\in A\left(x\:R\:y\land y\:R\:z\right)\implies x\:R\:z\right)$. 
\end{enumerate}
For a binary relation $R$ on $A$, this is equivalent to: 
\begin{enumerate}
\item $R\mbox{ is \emph{reflexive}}\iff\mbox{id}_{A}\subseteq R$. 
\item $R\mbox{ is \emph{symmetric}}\iff R=R^{-1}$. 
\item $R\mbox{ is \emph{transitive}}\iff R\circ R\subseteq R$.
\end{enumerate}

\paragraph{Equivalence Classes}
\begin{enumerate}
\item For an equivalence relation $R$ on $A$, for any $a\in A$, the \emph{equivalence
class} of $a$ with respect to $R$ is $\left[a\right]_{R}\triangleq\left\{ x\in A|a\sim_{R}x\right\} $.
\item For an equivalence relation $R$ on $A$, the set $\left\{ \left[a\right]{}_{R}|a\in A\right\} $
forms a partition of $A$.
\end{enumerate}

\paragraph{Transitive Closure}

\emph{Transitive closure}: $a\:R^{+}\:b=\exists n\geq1\left(a\:R^{n}\:b\right)$,
i.e. $R^{+}=\cup_{i\geq1}R^{i}$. Contains at `paths' in $A$ through
$R$. This is the smallest transitive relation containing $R$.


\section{Functions}
\begin{enumerate}
\item A \emph{function} $f$ from a set $A$ to a set \textbf{$B$, $f:A\rightarrow B$}
is a relation $f\subseteq A\times B$ such that every element of $A$
is related to one element in $B$.
\item $A$ is the \emph{domain} of $f$.
\item $B$ is the \emph{co-domain} of $f$.
\item Consider $f\left(a\right)=b$: $a$ is the \emph{pre-image} of $b$
under $f$ and $b$ is the \emph{image} of $a$ under $f$. Every
element of the domain has a single image but elements of the co-domain
can have any number of pre-images.
\item An $n$-ary function is written $f\left(a_{1},a_{2},\dots,a_{n}\right)$.
\item $B^{A}$ denotes the set of all functions from $A$ to \textbf{$B$}.
\item If $\left|A\right|=m$ and $\left|B\right|=n$, then $\left|B^{A}\right|=n^{m}$ or $\left(n+1\right)^{m}$ including partial functions.
\end{enumerate}

\paragraph{Formal Definition of a Function $\boldsymbol{\dagger}$}

$f$ is a function if it satisfies:
\begin{enumerate}
\item $f\left(a\right)=b_{1}\land f\left(a\right)=b_{2}\implies b_{1}=b_{2}$.
\item $\forall a\in A\exists b\in B\left(f\left(a\right)=b\right)$.
\end{enumerate}

\paragraph{Equality}

$f=g\triangleq\forall x\in A\left(f\left(x\right)=g\left(x\right)\right)$.


\paragraph{Image Set}
\begin{enumerate}
\item For $X\subseteq A$, $f\left[X\right]\triangleq\left\{ f\left(a\right)\in B|a\in X\right\} $.
\item The \emph{image set} of $f$ is defined as $f\left[A\right]\subseteq B$.
\end{enumerate}

\paragraph{Characteristic Functions}
\begin{enumerate}
\item For sets $A,B\subseteq A$, the \emph{characteristic function} of
$B\subseteq A$ is the function $\chi_{B}:A\rightarrow\{0,1\}$ is
defined as $\chi_{B}\left(a\right)\begin{cases}
1 & \left(a\in B\right)\\
0 & \left(a\in A\backslash B\right)
\end{cases}$.
\item For a relation $R\subseteq A_{1}\times A_{2}\times\dots\times A_{n}$,
the \emph{characteristic function} of $R$ is the function $\chi_{R}:A_{1}\times A_{2}\times\dots\times A_{n}\rightarrow\{0,1\}$
is defined as $\chi_{B}\left(a_{1},a_{2},\dots.,a_{n}\right)\begin{cases}
1 & \left(\left\langle a_{1},a_{2},\dots,a_{n}\right\rangle \in R\right)\\
0 & \left(\left\langle a_{1},a_{2},\dots,a_{n}\right\rangle \notin R\right)
\end{cases}$.
\end{enumerate}

\paragraph{Partial Functions}

A \emph{partial function} need not satisfy clause 2 of $\dagger$
(and so assigns each element in the domain to at most one element
in the range). Functions that satisfy clause 2 are \emph{total functions}.


\paragraph{Properties of Functions}

For a function $f:A\rightarrow B$:
\begin{enumerate}
\item $f\mbox{ is \emph{surjective} (onto)}\triangleq\forall b\in B\exists a\in A\left(f\left(a\right)=b\right)$
(every element of $B$ is in the image of $f$).
\item $f\mbox{ is \emph{injective} (one-to-one)}\triangleq\forall a_{1},a_{2}\in A\left(f\left(a_{1}\right)=f\left(a_{2}\right)\implies a_{1}=a_{2}\right)$
(for each $b\in B$ there exists at most one $a\in A$ with $f\left(a\right)=b$).
\item $f\mbox{ is \emph{bijective}}\triangleq f\mbox{ is both one-to-one and onto}$.
\end{enumerate}
Considering the cardinality of the sets $A$ and $B$:
\begin{enumerate}
\item If $f$ is onto, then $\left|A\right|\geq\left|B\right|$.
\item If $f$ is one-to-one, then $\left|A\right|\leq\left|B\right|$.
\item If $f$ is a bijection, then $\left|A\right|=\left|B\right|$.
\end{enumerate}

\paragraph{The Pigeonhole Principle Applied to Functions}

For $f:A\rightarrow B$ and $X\subseteq A$, $\left|f\left[X\right]\right|\leq\left|X\right|$.


\paragraph{Cantor-Bernstein Theorem $\boldsymbol{\ddagger}$}

If there exists functions $f:A\rightarrow B$ and $g:B\rightarrow A$,
both injective or both surjective, then there exists a bijection $h:A\rightarrow B$.


\paragraph{Operations on Functions}

For functions $f:A\rightarrow B$ and $g:B\rightarrow C$.
\begin{enumerate}
\item \emph{Composition}: $g\circ f\left(a\right)=g\left(f\left(a\right)\right)$,
i.e. $g\circ f\left(a\right)=c\triangleq\exists b\in B\left(f\left(a\right)=b\land g\left(b\right)=c\right)$.
Note that composition is associative. If $f$ and $g$ are bijections,
then so is $g\circ f$.
\item \emph{Identity}: The function $\mbox{id}_{A}:A\rightarrow A$ is defined
as $\mbox{id}_{A}\left(a\right)=a$.
\item \emph{Inverse}: The function $f':B\rightarrow A$ is an inverse of
$f$ whenever: $\forall a\in A\left(f'\circ f\left(a\right)=a\right)$
and $\forall b\in B\left(f\circ f'\left(b\right)=b\right)$, i.e.
$f'\circ f=id_{A}$ and $f\circ f'=id_{B}$. For $f$ to have an inverse,
$f$ must be a bijection, and the inverse is unique.
\end{enumerate}

\paragraph{Cardinality of Sets}
\begin{enumerate}
\item $A\sim B\triangleq\exists f:A\rightarrow B$ ($f$ is a bijection).
The relation $\sim$ is an equivalence relation.
\item Hence if there exist functions $f:A\rightarrow B$ and $g:B\rightarrow A$,
both injective or both surjective, then $A\sim B$ (by $\ddagger$).
\item We say $A$ and $B$ have the same \emph{cardinality}, whenever $A\sim B$.
\end{enumerate}

\paragraph{Cantor's Theorem}

For any set $A$, $A\not\sim{\cal P}A$. To prove, assume a bijection
$f:A\rightarrow{\cal P}A$ exists. Consider $B=\left\{ a\in A|a\notin f\left(a\right)\right\} $.
Since $f$ is a bijection, there exists some $b\in A$ such that $f\left(b\right)=B$.
Then consider individually the cases $b\in B$ and $b\notin B$ to
generate a contradiction.


\section{Infinity}


\paragraph{Countability}

A set $A$ is \emph{countable} if $A$ is finite or $A\sim\N$. This
is equivalent to:
\begin{enumerate}
\item $B$ is countable and $A\subseteq B$.
\item There exists a surjection $f:\N\rightarrow A$.
\end{enumerate}

\paragraph{Uncountability}

Cantor's diagonal argument produces an object that does not exist
in any list. Hence any list is incomplete and so the set is uncountable.


\section{Orderings}

For a binary relation $R$ on $A$:
\begin{enumerate}
\item $R$ is a \emph{pre-order}:$R$ is reflexive and transitive.
\item $R$ is \emph{anti-symmetric}: $\forall a,b\in A\left(x\:R\:y\land y\:R\:x\implies x=y\right)$.
\item $R$ is a \emph{partial order relation}: R is reflexive, transitive
and anti-symmetric. Usually denoted by $\leq_{A}$.
\item $R$ is \emph{irreflexive}: $\forall a\in A\left(\lnot\left(a\:R\:a\right)\right)$.
\item $R$ is a \emph{strict partial order relation}: $R$ is irreflexive
and transitive. Usually denoted by $<_{A}$.
\item R is a \emph{total} (\emph{linear}) \emph{order}: $R$ is a partial
order that also satisfies $\forall a,b\in A\left(a\:R\:b\lor b\:R\:a\right)$.
\end{enumerate}

\paragraph{Ordering of Products}
\begin{enumerate}
\item Product order: $\left\langle a_{1},b_{1}\right\rangle \leq_{P}\left\langle a_{2},b_{2}\right\rangle \triangleq a_{1}\leq_{A}a_{2}\land b_{1}\leq_{B}b_{2}$
\item Lexicographic order: First compare $a_{i}$s, then $b_{i}$s.
\end{enumerate}

\paragraph{Hasse Diagrams}

Definitions:
\begin{enumerate}
\item If $R$ is a partial order on set $A$ and $a\:R\:b$ for $a\neq b$,
$a$ is a \emph{predecessor} of $b$ and $b$ is a \emph{successor}
of $a$.
\item If $a$ is a predecessor of $b$ and there exists no $c\neq a,b$
with $a\:R\:c$ and $c\:R\:b$ then $a$ is the \emph{immediate predecessor}
of $b$.
\end{enumerate}
Hasse diagrams:
\begin{enumerate}
\item Record only immediate predecessors.
\item Direction of lines omitted, lines are directed `up the page'.
\end{enumerate}

\paragraph{Properties of Partial Orders}

For the partial order $\leq_{A}$ and $a\in A$:
\begin{enumerate}
\item $a\mbox{ is minimal}\triangleq\forall b\in A\left(b\leq a\implies b=a\right)$.
\item $a\mbox{ is least}\triangleq\forall b\in A\left(a\leq b\right)$.
\item $a\mbox{ is maximal}\triangleq\forall b\in A\left(a\leq b\implies a=b\right)$.
\item $a\mbox{ is greatest}\triangleq\forall b\in A\left(b\leq a\right)$.
\end{enumerate}
Note that:
\begin{enumerate}
\item Any least / greatest element is a minimal / maximal element respectively.
\item Any least / greatest element is unique.
\item If $A$ is finite and non-empty, then $\leq_{A}$ must have a minimal,
maximal element.
\item If $\leq_{A}$ is a total order, where $A$ is finite and non-empty,
then it has a least, greatest element.
\end{enumerate}

\paragraph{Well-Founded Partial Orders}
\begin{enumerate}
\item A partial order is \emph{well-founded} if it has no infinite decreasing
chain of elements, i.e. for every infinite sequence $a_{1},a_{2},a_{3},\dots$
of elements in $A$ with $a_{1}\geq a_{2}\geq a_{3}\geq\dots$, there
exists $m\in\N$ such that $m\geq1$ and $a_{n}=a_{m}$ for every
$n\geq m$.
\item If two partial orders $\leq_{A}$ and $\leq_{B}$ are well-founded,
then the lexicographical order $\leq_{L}$on $A\times B$ is also
well-founded.\end{enumerate}

\end{document}
