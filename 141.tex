\documentclass[10pt,twoside,twocolumn]{article}
\usepackage[latin9]{inputenc}
\usepackage[landscape]{geometry}
\geometry{verbose,tmargin=0.5in,bmargin=0.75in,lmargin=0.5in,rmargin=0.5in}
\setlength{\parskip}{\smallskipamount}
\setlength{\parindent}{0pt}
\usepackage{color}
\usepackage{amsmath}

\makeatletter
\usepackage{amsmath,amssymb,amsthm,mathrsfs,amsfonts,dsfont} 

\setlength{\columnsep}{0.25in}

\renewcommand*{\thefootnote}{\fnsymbol{footnote}}

\newcommand{\R}[0]{\mathds{R}} % real numbers
\newcommand{\Z}[0]{\mathds{Z}} % integers
\newcommand{\N}[0]{\mathds{N}} % natural numbers
\newcommand{\nat}[0]{\mathds{N}} % natural numbers
\newcommand{\Q}[0]{\mathds{Q}} % rational numbers

\definecolor{grey}{rgb}{0.5,0.5,0.5}
\definecolor{green}{rgb}{0,0.75,0}

\makeatother

\usepackage{listings}
\renewcommand{\lstlistingname}{Listing}

\begin{document}

\title{Reference Sheet for CO141 Reasoning about Programs}


\date{Spring 2017}

\maketitle

\paragraph{Stylised Proofs for Reasoning}
\begin{enumerate}
\item Write out and name each given formula.
\item Write out and name each formula to be shown.
\item Plan out the proof and name intermediate results.
\item Justify each step of the proof.
\end{enumerate}
We use the following methods to plan out a proof for $P$:
\begin{enumerate}
\item \emph{Contradiction} i.e. show $\lnot P\rightarrow\mbox{false}$.
\item If $P=Q\land R$ show both $Q$ and $R$.
\item If $P=Q\lor R$ show either $Q$ or $R$.
\item If $P=Q\rightarrow R$ assume $Q$ and show $R$.
\item If $P=\lnot Q$ show $Q\rightarrow\mbox{false}$.
\item If $P=\forall xQ\left(x\right)$ take arbitrary $c$ and show $Q\left(c\right)$.
\item If $P=\exists xQ\left(x\right)$ find some $c$ and show $Q\left(c\right)$.
\end{enumerate}
We use the following methods to justify our proof:
\begin{enumerate}
\item If false holds then $P$ holds.
\item If $Q\land R$ holds then $Q$ and $R$ both hold.
\item If $Q\lor R$ holds we do case analysis assuming each in turn.
\item If $Q\rightarrow R$ holds and $Q$ holds then $R$ holds.
\item If $\forall xQ\left(x\right)$ holds then $Q\left(c\right)$ holds
for any $c$.
\item If $\exists xQ\left(x\right)$ holds then $Q\left(c\right)$ holds
for some $c$.
\item We can apply any lemma / equivalence given or proven earlier.
\end{enumerate}

\section{Reasoning about Haskell Programs}


\subsection{Mathematical Induction}

\emph{Principle of Mathematical Induction}: For any $P\subseteq\mathbb{N}$:
\[
P\left(0\right)\land\forall k:\mathbb{N}.\left[P\left(k\right)\rightarrow P\left(k+1\right)\right]\rightarrow\forall\mathbb{N}.P\left(n\right)
\]
i.e. to prove by induction, we prove a base case and an inductive
step.

\emph{General Technique}: For any $P\subseteq\mathbb{Z}$ and any
$m:\mathbb{Z}$:
\[
P\left(m\right)\land\forall k\geq m.\left[P\left(k\right)\rightarrow P\left(k+1\right)\right]\rightarrow\forall n\geq m.P\left(n\right)
\]



\subsection{Strong Induction}

\[
P\left(0\right)\land\forall k:\mathbb{N}.\left[\forall j\in\left\{ 0..k\right\} .P\left(j\right)\rightarrow P\left(k+1\right)\right]\rightarrow\forall n:\mathbb{N}.P\left(n\right)
\]
\emph{Note}: for some problems, it may be necessary to split the inductive
step into cases. E.g. $k=0$ or $k\neq0$.

Mathematical induction and strong induction are \emph{equivalent}.


\subsection{Structural Induction over Haskell Data Types}

We generalise the concept of predecessor and successor.

\emph{Example 1}: Structural Induction Principle over Lists:
\[
P\left(\mathtt{[]}\right)\land\forall\mathtt{vs}:\mathtt{[T]}\forall\mathtt{v}:\mathtt{T}.\left[P\left(\mathtt{vs}\right)\rightarrow P\left(\mathtt{v:vs}\right)\right]\rightarrow\forall\mathtt{xs}:\mathtt{[T]}.P\left(\mathtt{xs}\right)
\]


\emph{Example 2}: Structural Induction Principle over \texttt{Data
BExp = Tr \textbar{} Fl \textbar{} BNt BExp \textbar{} BAnd BExp BExp}:
\begin{multline*}
P\left(\mathtt{Tr}\right)\land P\left(\mathtt{Fl}\right)\land\forall\mathtt{b}:\mathtt{BExp}.\left[P\left(\mathtt{b}\right)\rightarrow P\left(\mathtt{BNt}\;\mathtt{b}\right)\right]\land\\
\forall\mathtt{b1},\mathtt{b2}:\mathtt{BExp}.\left[P\left(\mathtt{b1}\right)\land P\left(\mathtt{b2}\right)\rightarrow P\left(\mathtt{BAnd}\;\mathtt{b1}\;\mathtt{b2}\right)\right]\rightarrow\forall\mathtt{b}:\mathtt{BExp}.P\left(\mathtt{b}\right)
\end{multline*}



\paragraph{Proof Methods}
\begin{enumerate}
\item Invent an Auxiliary Lemma.
\item Strengthen the original property. E.g. rewrite $\forall\mathtt{is}:\mathtt{[Int]}.\mathtt{sum}\;\mathtt{is}=\mathtt{sum\_tr\;is\;0}$
as $\forall\mathtt{k}:\mathtt{Int}\forall\mathtt{is}:\mathtt{[Int]}.\mathtt{k}+\mathtt{sum}\;\mathtt{is}=\mathtt{sum\_tr\;is\;k}$.
\end{enumerate}

\subsection{Induction over Recursively Defined Structures}

Sets, relations and functionas can be defined inductively, which leads
to inductive principles.


\paragraph{Sets}

\emph{Example}: Consider the set of ordered lists, $OL\subseteq\mathbb{N}^{*}$:
\begin{align*}
1.\; & []\in OL\\
2.\; & \forall i\in\mathbb{N}.i:[]\in OL\\
3.\; & \forall i,j\in\mathbb{N}\forall js\in\mathbb{N}^{*}.\left[i\leq j\land j:js\in L\rightarrow i:j:js\in OL\right]
\end{align*}
For a property $Q\subseteq\mathbb{N}^{*}$, we get the inductive principle
\begin{multline*}
Q\left([]\right)\land\forall i\in\mathbb{N}.Q\left(i:[]\right)\land\\
\forall i,j\in\mathbb{N}\forall js\in\mathbb{N}^{*}.\left[i\leq j\land j:js\in OL\land Q\left(j:js\right)\rightarrow Q\left(i:j:js\right)\right]\\
\rightarrow\forall ns\in OL.Q\left(ns\right)
\end{multline*}



\paragraph{Relations}

\emph{Example 1}: Consider the strictly less than relation, $SL\subseteq\mathbb{N}\times\mathbb{N}$:
\begin{align*}
1.\; & \forall k\in\mathbb{N}.SL\left(0,k+1\right)\\
2.\; & \forall m,n\in\mathbb{N}.\left[SL\left(m,n\right)\rightarrow SL\left(m+1,n+1\right)\right]
\end{align*}
For a property $Q\subseteq\mathbb{N}\times\mathbb{N}$, we get the
inductive principle
\begin{multline*}
\forall k\in\mathbb{N}.Q\left(0,k+1\right)\land\forall m,n\in\mathbb{N}.,\left[SL\left(m,n\right)\land Q\left(m,n\right)\rightarrow Q\left(m+1,n+1\right)\right]\\
\rightarrow\forall m,n\in\mathbb{N}.\left[SL\left(m,n\right)\rightarrow Q\left(m,n\right)\right]
\end{multline*}


\emph{Example 2}: Consider the set of natural numbers, $S_{\mathbb{N}}$:
\begin{align*}
1.\; & \mbox{Zero}\in S_{\mathbb{N}}\\
2.\; & \forall n.\left[n\in S_{\mathbb{N}}\rightarrow\mbox{Succ }n\in S_{\mathbb{N}}\right]
\end{align*}
and the predicate $Odd\left(S_{\mathbb{N}}\right)$:
\begin{align*}
1.\; & Odd\left(\mbox{Succ Zero}\right)\\
2.\; & \forall n\in S_{\mathbb{N}}.\left[Odd\left(n\right)\rightarrow Odd\left(\mbox{Succ }\left(\mbox{Succ }n\right)\right)\right]
\end{align*}
Here it is much simpler to derive the inductive principle from the
definition of $Odd$, rather than from the definition of $S_{\mathbb{N}}$:
\begin{multline*}
Q\left(\mbox{Succ Zero}\right)\land\forall n\in S_{\mathbb{N}}.\left[Odd\left(n\right)\land Q\left(n\right)\rightarrow Q\left(\mbox{Succ }\left(\mbox{Succ }n\right)\right)\right]\\
\rightarrow\forall n\in S_{\mathbb{N}}.\left[Odd\left(n\right)\rightarrow Q\left(n\right)\right]
\end{multline*}



\paragraph{Functions}

\emph{Example}: Consider the Haskell function:
\begin{lstlisting}[language=Haskell,basicstyle={\small\ttfamily},tabsize=4]
DM (i, j) = DM' (i, j, 0, 0)
DM' (i, j, cnt, acc)
	| acc + j > i = (cnt, i - acc)
	| otherwise = DM' (i, j, cnt +1, acc + j)
\end{lstlisting}
We can define this inductively as follows:
\begin{align*}
1.\; & \forall i,j\in\mathbb{Z}.\left[\mathtt{DM}\left(i,j\right)=\mathtt{DM'}\left(i,j,0,0\right)\right]\\
2.\; & \forall i,j,cnt,acc\in\mathbb{Z}.\left[acc+j>i\rightarrow\mathtt{DM'}\left(i,j,cnt,acc\right)=\left(cnt,i-acc\right)\right]\\
3.\; & \forall i,j,cnt,acc\in\mathbb{Z}.[acc+j\leq i\land\mathtt{DM'}\left(i,j,cnt+1,acc+j\right)=\left(k1,k2\right)\\
 & \rightarrow\mathtt{DM'}\left(i,j,cnt,acc\right)=\left(k1,k2\right)]
\end{align*}
For a predicate $Q\in\mathbb{Z}\times\mathbb{Z}\times\mathbb{Z}\times\mathbb{Z}\times\mathbb{Z}\times\mathbb{Z}$,
we get the following inductive principle for \texttt{DM'}:
\begin{multline*}
\forall i,j,cnt,acc\in\mathbb{Z}.\left[acc+j>i\rightarrow Q\left(i,j,cnt,acc,cnt,i-acc\right)\right]\land\\
\forall i,j,cnt,acc,k1,k2\in\mathbb{Z}.[acc+j\leq i\land\mathtt{DM'}\left(i,j,cnt+1,acc+j\right)=\left(k1,k2\right)\land\\
Q\left(i,j,cnt+1,acc+j,k1,k2\right)\rightarrow Q\left(i,j,cnt,acc,k1,k2\right)]\\
\rightarrow\forall i,j,cnt,acc,k1,k2:\mathbb{Z}.\left[\mathtt{DM'}\left(i,j,cnt,acc\right)=\left(k1,k2\right)\rightarrow Q\left(i,j,cnt,acc,k1,k2\right)\right]
\end{multline*}



\section{Reasoning about Java Programs}


\subsection{Program Specifications}


\paragraph{Pre-Conditions, Mid-Conditions, Post-Conditions}
\begin{enumerate}
\item \emph{Pre-condition}: Must be proven in order to call function, an
assumption that code in method can make.
\item \emph{Mid-condition}: Assumption made at specific point in code, must
be guaranteed by preceeding code and can be assumed by subsequent
code.
\item \emph{Post-condition}: Expected to hold after the code has been executed
(assuming termination and that precondition held).
\end{enumerate}
\emph{Example}: Consider the Java code:
\begin{lstlisting}[language=Java,basicstyle={\small\ttfamily},tabsize=4,mathescape=true,commentstyle={\color{grey}}]
type method(type $x_1$, $\dots$, type $x_n$)
// PRE: $\color{grey}P(x_1, \dots, x_n)$
// POST: $\color{grey}Q(x_1, \dots, x_n)$
{
	code1
	// MID: $\color{grey}R(x_1, \dots, x_n)$
	code2
	// MID: $\color{grey}S(x_1, \dots, x_n)$
	code3
}
\end{lstlisting}
\emph{Note}: if we choose to introduce new (value) variables in our
conditions, there is an implicit universal quantification over the
whole specification.

Here we need to prove:
\begin{enumerate}
\item $P\left(x_{1},\dots,x_{n}\right)\land\mathtt{code1}\rightarrow R\left(x_{1},\dots,x_{n}\right)$
\item $R\left(x_{1},\dots,x_{n}\right)\land\mathtt{code2}\rightarrow S\left(x_{1},\dots,x_{n}\right)$
\item $S\left(x_{1},\dots,x_{n}\right)\land\mathtt{code3}\rightarrow Q\left(x_{1},\dots,x_{n}\right)$
\end{enumerate}

\paragraph{Program Variables}
\begin{enumerate}
\item $\mathtt{x}$ refers to the value of $\mathtt{x}$ before code is
executed.
\item $\mathtt{x'}$ refers to its value after code is executed, \emph{shouldn't
be present in assertions}.
\item $\mathtt{x_{0}}$ refers to its original value, as passed into the
method.
\end{enumerate}
We use $\mathtt{r}$ to refer to the return value of a method.


\paragraph{Arrays}
\begin{enumerate}
\item $\mathtt{a}\sim\mathtt{b}$ means $\mathtt{a}$ is a permutation of
$\mathtt{b}$.
\item $\mathtt{a}\approx\mathtt{b}$ means $\mathtt{a}$ is identical to
$\mathtt{b}$.
\item $\mathtt{a}\left[x..y\right)$ means the elements of $\mathtt{a}$
from index $x$ up to (but not including) $y$.
\item $Sorted\left(\mathtt{a}\right)$ means $\mathtt{a}$ is sorted.
\item $min\left(\mathtt{a}\right)$ is the smallest element in $\mathtt{a}$.
\item $max\left(\mathtt{a}\right)$ is the largest element in $\mathtt{a}$.
\end{enumerate}

\subsection{Conditional Branches}

We can assume the pre-condition and the \texttt{if else} condition.
Have to show post-condition holds on borth branches of the code.

\emph{Example}: Consider the Java code:
\begin{lstlisting}[language=Java,basicstyle={\small\ttfamily},tabsize=4,mathescape=true,commentstyle={\color{grey}}]
// PRE: $\color{grey}true$
if (x >= y) {
	// MID: $\color{grey}\mathtt{x_0} \geq \mathtt{y_0}$
	res = x;
	// MID: $\color{grey}\mathtt{res} = \mathtt{x_0} \land \mathtt{x_0} \geq \mathtt{y_0}$
} else {
	// MID: $\color{grey}\mathtt{y_0} > \mathtt{x_0}$
	res = y;
	// MID: $\color{grey}\mathtt{res} = \mathtt{y_0} \land \mathtt{y_0} > \mathtt{x_0}$
}
// MID: $\color{grey}\mathtt{res} = max\{\mathtt{x_0} , \mathtt{y_0}\}$
\end{lstlisting}



\subsection{Recursion}


\paragraph{Method Calls}

Need to show that the precondition is met before the method call,
then can assume postcondition will hold afterwards. We make necessary
substitutions in order to prove our assertions.

\emph{Example}: Consider the java method:
\begin{lstlisting}[language=Java,numbers=left,numberstyle={\footnotesize},basicstyle={\small\ttfamily},tabsize=4,mathescape=true,xleftmargin={5.0ex},commentstyle={\color{grey}}]
int sumAux(int[] a, int i)
// PRE: $\color{grey}\mathtt{a} \neq \mathtt{null} \land 0 \leq \mathtt{i} \leq \mathtt{a.length}$
// POST: $\color{grey}\mathtt{a} \approx \mathtt{a_0} \land \mathtt{r} = \sum \mathtt{a}[\mathtt{i}..\mathtt{a.length})$
{
	if (i == a.length) {
		// MID: $\color{grey}\mathtt{a} \approx \mathtt{a_0} \land \mathtt{i} = \mathtt{a.length}$
		return 0;
	} else {
		// MID: $\color{grey}\mathtt{a} \approx \mathtt{a_0} \land \mathtt{a} \neq \mathtt{null} \land 0 \leq \mathtt{i} < \mathtt{a.length}$
		int val = a[i] + sumAux(a, i+1);
		// MID: $\color{grey}\mathtt{a} \approx \mathtt{a_0} \land \mathtt{val} = \mathtt{a}[\mathtt{i}..\mathtt{a.length})$
		return val;
	}
}
\end{lstlisting}
We need to prove:
\begin{enumerate}
\item \emph{Line 6}: Show mid-condition holds: ${\color{blue}\mathtt{a_{0}}\neq\mathtt{null}\land0\leq\mathtt{i}\leq\mathtt{a_{0}.length}}\land{\color{red}\mathtt{i=a_{0}.length}}\land{\color{green}\mathtt{a'}\approx\mathtt{a_{0}}}\rightarrow\mathtt{a'}\approx\mathtt{a_{0}}\land\mathtt{i}=\mathtt{a'.length}$.
\item \emph{Line 7}: Show post-condition holds: ${\color{blue}\mathtt{a}\approx\mathtt{a_{0}}\land\mathtt{i}=\mathtt{a.length}}\land{\color{red}\mathtt{r}=0}\rightarrow\mathtt{a}\approx\mathtt{a_{0}}\land\mathtt{r}=\sum\mathtt{a}\left[\mathtt{i}..\mathtt{a.length}\right)$.
\item \emph{Line 9}: Show mid-condition holds:${\color{blue}\mathtt{a_{0}}\neq\mathtt{null}\land0\leq\mathtt{i}\leq\mathtt{a_{0}.length}}\land{\color{red}\mathtt{i\neq a'.length}}\land{\color{green}\mathtt{a'}\approx\mathtt{a_{0}}}\rightarrow\mathtt{a'}\approx\mathtt{a_{0}}\land\mathtt{a'}\neq\mathtt{null}\land0\leq\mathtt{i}<\mathtt{a'.length}$.
\item \emph{Line 10}: Show pre-condition for called method holds: ${\color{blue}\mathtt{a}\approx\mathtt{a_{0}}\land\mathtt{a}\neq\mathtt{null}\land0\leq\mathtt{i}<\mathtt{a.length}}\rightarrow\mathtt{a}\neq\mathtt{null}\land0\leq\mathtt{i+1}\leq\mathtt{a.length}$.
\item \emph{Line 11}: Show mid-condition holds: ${\color{blue}\mathtt{a}\approx\mathtt{a_{0}}\land\mathtt{a}\neq\mathtt{null}\land0\leq\mathtt{i}<\mathtt{a.length}}\land{\color{red}{\color{magenta}\mathtt{a'}\approx\mathtt{a}\land\mathtt{r}=\sum\mathtt{a'}\left[\mathtt{i+1}..\mathtt{a'.length}\right)}}\land{\color{red}\mathtt{val'}=\mathtt{a[i]+r}}\rightarrow\mathtt{a'}\approx\mathtt{a_{0}}\land\mathtt{val'}=\sum\mathtt{a'}\left[\mathtt{i}..\mathtt{a'.length}\right)$.
\item \emph{Line 12}: Show post-condition holds: ${\color{blue}\mathtt{a}\approx\mathtt{a_{0}}\land\mathtt{val}=\sum\mathtt{a}\left[\mathtt{i}..\mathtt{a.length}\right)}\land{\color{red}\mathtt{r}=\mathtt{val}}\rightarrow\mathtt{a}\approx\mathtt{a_{0}}\land\mathtt{r}=\sum\mathtt{a}\left[\mathtt{i}..\mathtt{a.length}\right)$.
\end{enumerate}
Blue statements come from the \textcolor{blue}{pre-condition or previous
mid-condition}, green statements \textcolor{green}{implicitly from
code}, red statements \textcolor{red}{explicitly from code} and purple
statements from the \textcolor{magenta}{post-condition of a called
method}.


\subsection{Iteration}


\paragraph{Invariant}

To prove a property holds throughout the loop, we need to prove that
the \emph{invariant} holds before entering the loop, and is preserved
by the loop body (including at termination). The invariant and $\lnot cond$
can be used to prove the following mid-condition.


\paragraph{Variant}

To prove a loop will terminate, we find an integer expression which
is bounded below, and decreases in \emph{every} loop iteration.

\emph{Example}: Consider the java method:
\begin{lstlisting}[language=Java,numbers=left,numberstyle={\footnotesize},basicstyle={\small\ttfamily},tabsize=4,mathescape=true,xleftmargin={5.0ex},commentstyle={\color{grey}}]
int culSum(int[] a)
// PRE $\color{grey}P$: $\color{grey}\mathtt{a}\neq\mathtt{null}$
// POST $\color{grey}Q$: $\color{grey}\mathtt{a.length}=\mathtt{a_{0}.length}\land\mathtt{r}=\sum\mathtt{a_{0}}\left[0..\mathtt{a.length}\right)\land$
	       $\color{grey}\forall k\in\left[0..\mathtt{a.length}\right).\left[\mathtt{a[}k\mathtt{]}=\sum\mathtt{a}_{\mathtt{0}}\left[0..k+1\right)\right]$
{
	int res = 0;
	int i = 0;
	// INV $\color{grey}I$: $\color{grey}\mathtt{a}\neq\mathtt{null}\land\mathtt{a.length}=\mathtt{a_{0}.length}\land0\leq\mathtt{i}\leq\mathtt{a.length}\land$
			  $\color{grey}\mathtt{res}=\sum\mathtt{a_{0}}\left[0..\mathtt{i}\right)\land\forall k\in\left[0..\mathtt{i}\right).\left[\mathtt{a[}k\mathtt{]}=\sum\mathtt{a_{0}}\left[0..k+1\right)\right]\land$
			  $\color{grey}\forall k\in\left[\mathtt{i}..\mathtt{a.length}\right).\left[\mathtt{a[}k\mathtt{]}=\mathtt{a_{0}[}k\mathtt{]}\right]$
	// VAR $\color{grey}V$: $\color{grey}\mathtt{a.length}-\mathtt{i}$
	while (i < a.length) {
		res = res + a[i];
		a[i] = res; i++;
	}
	// MID $\color{grey}M$: $\color{grey}\mathtt{a.length}=\mathtt{a_{0}.length}\land\mathtt{res}=\sum\mathtt{a_{0}\left[0..\mathtt{a.length}\right)}\land$
		  	   $\color{grey}\forall k\in\left[0..\mathtt{a.length}\right).\left[\mathtt{a[}k\mathtt{]}=\sum\mathtt{a}_{\mathtt{0}}\left[0..k+1\right)\right]$
	return res;
}
\end{lstlisting}
We need to prove:

1. Invariant holds before loop is entered. 
\begin{multline*}
{\color{blue}P\left[\mathtt{a}\mapsto\mathtt{a}_{\mathtt{0}}\right]}\land{\color{red}\mathtt{res}=0\land\mathtt{i}=0}\land{\color{green}\mathtt{a}\approx\mathtt{a_{0}}}\\
\rightarrow I
\end{multline*}


2. Loop body re-establishes invariant. 
\begin{multline*}
{\color{blue}I}\land{\color{red}\mathtt{i}<\mathtt{a.length}\land\mathtt{res'}=\mathtt{res}+\mathtt{a[i]}\land\mathtt{a'[i]}=\mathtt{res'}\land\mathtt{i'}=\mathtt{i}+1}\land\\
{\color{green}\forall k\in\left[0..\mathtt{a.length}\right)\backslash\left\{ i\right\} .\left[\mathtt{a'}\mathtt{[}k\mathtt{]}=\mathtt{a[}k\mathtt{]}\right]}\\
\rightarrow I\left[\mathtt{a\mapsto a',i\mapsto i',res\mapsto res'}\right]
\end{multline*}


3. Mid-condition holds straight after loop. 
\begin{multline*}
{\color{blue}I}\land{\color{red}\mathtt{i}\geq\mathtt{a.length}}\\
\rightarrow M
\end{multline*}


4. Loop terminates. 
\begin{multline*}
{\color{blue}I}\land{\color{red}\mathtt{i}<\mathtt{a.length}\land\mathtt{res'}=\mathtt{res}+\mathtt{\mathtt{a[}i\mathtt{]}}\land\mathtt{\mathtt{a'[}i]}=\mathtt{res'}\land\mathtt{i'}=\mathtt{i}+1}\land\\
{\color{green}\forall k\in\left[0..\mathtt{a.length}\right)\backslash\left\{ \mathtt{i}\right\} .\left[\mathtt{\mathtt{a'}[}k\mathtt{]}=\mathtt{a[}k\mathtt{]}\right]}\\
\rightarrow V\geq0\land V\left[\mathtt{a\mapsto a',i\mapsto i',res\mapsto res'}\right]<V
\end{multline*}


5. Post-condition established. 
\begin{multline*}
{\color{blue}M}\land{\color{red}\mathtt{r=res}}\\
\rightarrow Q
\end{multline*}


6. Array accesses are legal. 
\begin{multline*}
{\color{blue}I}\land{\color{red}\mathtt{i<a.length}}\\
\rightarrow0\leq\mathtt{i<a.length}
\end{multline*}

\end{document}
