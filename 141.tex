\documentclass[10pt,twoside,twocolumn]{article}
\usepackage[latin9]{inputenc}
\usepackage[landscape]{geometry}
\geometry{verbose,tmargin=0.5in,bmargin=0.75in,lmargin=0.5in,rmargin=0.5in}
\setlength{\parskip}{\smallskipamount}
\setlength{\parindent}{0pt}

\makeatletter

\usepackage{amsmath,amssymb,amsthm,mathrsfs,amsfonts,dsfont} 

\setlength{\columnsep}{0.25in}

\renewcommand*{\thefootnote}{\fnsymbol{footnote}}

\newcommand{\R}[0]{\mathds{R}} % real numbers
\newcommand{\Z}[0]{\mathds{Z}} % integers
\newcommand{\N}[0]{\mathds{N}} % natural numbers
\newcommand{\nat}[0]{\mathds{N}} % natural numbers
\newcommand{\Q}[0]{\mathds{Q}} % rational numbers

\makeatother

\begin{document}

\title{Reference Sheet for CO141 Reasoning about Programs}


\date{Spring 2017}

\maketitle

\paragraph{Stylised Proofs for Reasoning}
\begin{enumerate}
\item Write out and name each given formula.
\item Write out and name each formula to be shown.
\item Plan out the proof and name intermediate results.
\item Justify each step of the proof.
\end{enumerate}
We use the following methods to plan out a proof for $P$:
\begin{enumerate}
\item \emph{Contradiction} i.e. show $\lnot P\rightarrow\mbox{false}$.
\item If $P=Q\land R$ show both $Q$ and $R$.
\item If $P=Q\lor R$ show either $Q$ or $R$.
\item If $P=Q\rightarrow R$ assume $Q$ and show $R$.
\item If $P=\lnot Q$ show $Q\rightarrow\mbox{false}$.
\item If $P=\forall xQ\left(x\right)$ take arbitrary $c$ and show $Q\left(c\right)$.
\item If $P=\exists xQ\left(x\right)$ find some $c$ and show $Q\left(c\right)$.
\end{enumerate}
We use the following methods to justify our proof:
\begin{enumerate}
\item If false holds then $P$ holds.
\item If $P\land Q$ holds then $Q$ and $R$ both hold.
\item If $P\lor Q$ holds we do case analysis assuming each in turn.
\item If $P\rightarrow Q$ holds and $P$ holds then $Q$ holds.
\item If $P=\forall xQ\left(x\right)$ holds then $P\left(c\right)$ holds
for any $c$.
\item If $P=\exists xQ\left(x\right)$ holds then $P\left(c\right)$ holds
for some $c$.
\item We can apply any lemma / equivalence given or proven earlier.
\end{enumerate}

\section{Reasoning about Haskell Programs}


\subsection{Mathematical Induction}


\paragraph{Principle of Mathematical Induction}

For any $P\subseteq\mathbb{N}$:

\[
P\left(0\right)\land\forall k:\mathbb{N}.\left[P\left(k\right)\rightarrow P\left(k+1\right)\right]
\]



\subsection{Strong Induction}


\subsection{Structural Induction over Haskell Data Types}


\subsection{Induction over Recursively Defined Structures}


\section{Reasoning about Java Programs}


\subsection{Program Specifications}


\subsection{Conditional Branches}


\subsection{Recursion}


\subsection{Iteration}
\end{document}
