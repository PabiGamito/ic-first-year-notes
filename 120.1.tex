\documentclass[twocolumn,english]{article}
\usepackage[T1]{fontenc}
\usepackage[latin9]{inputenc}
\usepackage[landscape]{geometry}
\geometry{verbose,tmargin=0.5in,bmargin=0.75in,lmargin=0.5in,rmargin=0.5in}

\makeatletter

\setlength{\columnsep}{0.25in}
\usepackage{xcolor}
\usepackage{textcomp}
\usepackage{listings}
\lstset{
  language=haskell,
  tabsize=2,
  basicstyle=\small\ttfamily,
}

\makeatother

\usepackage{babel}
\usepackage{listings}
\renewcommand{\lstlistingname}{Listing}

\begin{document}

\title{Reference Sheet for CO120.1 Programming I}


\date{Autumn 2016}

\maketitle
This reference sheet does not cover many interesting types and classes,
such as \texttt{Either}, \texttt{IO}, \texttt{Complex}, and \texttt{Monad}.
These are not included in the 120.1 syllabus, but even a limited understanding
of them may be helpful in examinations.


\section{Bools}

\begin{lstlisting}[language=Haskell,frame=single]
(&&) :: Bool -> Bool -> Bool
(||) :: Bool -> Bool -> Bool
not  :: Bool -> Bool
\end{lstlisting}



\section{Maybes}

\begin{lstlisting}[language=Haskell,frame=single]
maybe :: b -> (a -> b) -> Maybe a -> b
\end{lstlisting}


Given default value, a function, and \texttt{Maybe} value: Returns
default value if \texttt{Maybe} value is \texttt{Nothing}; Otherwise
applies function to value inside \texttt{Just}.

\begin{lstlisting}[language=Haskell,frame=single]
isJust    :: Maybe a -> Bool
isNothing :: Maybe a -> Bool
\end{lstlisting}


\emph{Requires }\texttt{\emph{Data.Maybe}}\emph{.}

\begin{lstlisting}[language=Haskell,frame=single]
fromJust :: Maybe a -> a
fromMaybe :: a-> Maybe a -> a
\end{lstlisting}


\emph{Requires }\texttt{\emph{Data.Maybe}}\emph{. }

\texttt{fromJust}: Given a \texttt{Maybe} value: Returns value inside
\texttt{Just} or error if \texttt{Nothing}.

\texttt{fromMaybe}: Given default value and \texttt{Maybe} value:
Returns value inside \texttt{Just} or default value if \texttt{Nothing}.
\begin{lstlisting}[language=Haskell,frame=single]
catMaybes :: [Maybe a] -> [a]
mapMaybe  :: (a -> Maybe b) -> [a] -> [b]
\end{lstlisting}


\emph{Requires }\texttt{\emph{Data.Maybe}}\emph{. }

\texttt{catMaybes}: Given list of \texttt{Maybe} values: Returns a
list of all \texttt{Just} values.

\texttt{mapMaybe}: Given function from value to \texttt{Maybe} value
and list of values: Returns a list of all \texttt{Just} values from
mapping function to list of values.


\section{Tuples}

\begin{lstlisting}[language=Haskell,frame=single]
fst :: (a, b) -> a
snd :: (a, b) -> b
\end{lstlisting}


It is usually better to use pattern matching unless used with higher
order functions.

\begin{lstlisting}[language=Haskell,frame=single]
curry   :: ((a, b) -> c) -> a -> b -> c
uncurry :: a -> b -> c -> (a, b) -> c
\end{lstlisting}


\texttt{curry}: Given uncurried function $f(x,y)$: Returns curried
function $f\:x\:y$.

\texttt{uncurry}: Given curried function $f\:x\:y$: Returns uncurried
function $f(x,y)$.
\begin{lstlisting}[language=Haskell,frame=single]
swap :: (a,b) -> (b,a)
\end{lstlisting}


\emph{Requires }\texttt{\emph{Data.Tuple}}\emph{.}


\section{Enums}

\begin{lstlisting}[language=Haskell,frame=single]
succ :: a -> a
pred :: a -> a
\end{lstlisting}


\texttt{succ}: Given a value: Returns its successor.

\texttt{pred}: Given a value: Returns its predecessor.

\begin{lstlisting}[language=Haskell,frame=single]
[n..]
[n,n'..]
[n..m]
[n,n'..m]
\end{lstlisting}


Returns enum from $n$.

Returns enum from $n$ then $n'$.

Returns enum from $n$ to $m$.

Returns enum from $n$ then $n'$ to $m$.


\section{Numbers}


\subsection{Common Operators}

\begin{lstlisting}[language=Haskell,frame=single]
(+), (-), (*) :: Num a => a -> a -> a
(/)           :: Fractional a => a -> a -> a
(^), (^^)     :: (Fractional a, Integral b) => a -> b -> a
\end{lstlisting}


\begin{lstlisting}[language=Haskell,frame=single]
negate :: Num a => a -> a
recip  :: Fractional a => a -> a
\end{lstlisting}


\begin{lstlisting}[language=Haskell,frame=single]
abs    :: Num a => a -> a
signum :: Num a => a -> a
\end{lstlisting}


\texttt{abs}: Given a value, returns its absolute value.

\texttt{signum}: Given a value, returns its sign.


\subsection{Mathematical Constants and Functions}

\begin{lstlisting}[language=Haskell,frame=single]
pi :: Floating a => a
\end{lstlisting}


\begin{lstlisting}[language=Haskell,frame=single]
exp, log, sqrt :: Floating a => a -> a
(**), logBase  :: Floating a => a -> a -> a
\end{lstlisting}


\texttt{log}: Natural log.

\begin{lstlisting}[language=Haskell,frame=single]
sin, cos, tan, asin, acos, atan :: Floating a => a -> a
\end{lstlisting}


In Radians.


\subsection{Number Theoretical Functions}

\begin{lstlisting}[language=Haskell,frame=single]
even :: Integral a => a -> Bool
odd  :: Integral a => a -> Bool
\end{lstlisting}


\begin{lstlisting}[language=Haskell,frame=single]
div    :: Num a => a -> a -> a
mod    :: Num a => a -> a -> a
divMod :: Num a => a -> a -> (a, a)
\end{lstlisting}


\texttt{divMod}: Simultaneous div and mod.
\begin{lstlisting}[language=Haskell,frame=single]
gcd :: Integral a => a -> a -> a
lcm :: Integral a => a -> a -> a
\end{lstlisting}



\subsection{Conversion between Numerical Types}

\begin{lstlisting}[language=Haskell,frame=single]
round, ceiling, floor :: (Floating a, Num b) => a -> b
\end{lstlisting}


\texttt{round}: Rounds to the closest integer, or even integer if
closest integers are equidistant.

\begin{lstlisting}[language=Haskell,frame=single]
fromIntegral :: (Integral a, Num b) a -> b
\end{lstlisting}



\section{Characters}

\emph{Requires }\texttt{\emph{Data.Char}}\emph{.}


\subsection{Classification}

\begin{lstlisting}[language=Haskell,frame=single]
isSpace       :: Char -> Bool
isLower       :: Char -> Bool
isUpper       :: Char -> Bool
isAlpha       :: Char -> Bool
isAlphaNum    :: Char -> Bool
isDigit       :: Char -> Bool
isPunctuation :: Char -> Bool
isSeparator   :: Char -> Bool
\end{lstlisting}



\subsection{Case Conversion}

\begin{lstlisting}[language=Haskell,frame=single]
toUpper :: Char -> Char
toLower :: Char -> Char
\end{lstlisting}



\subsection{Numeric Conversion}

\begin{lstlisting}[language=Haskell,frame=single]
ord :: Char -> Int
chr :: Int -> Char
\end{lstlisting}



\section{Lists}


\subsection{Working with Lists}


\subsubsection{List Operations and Transformations}

\begin{lstlisting}[language=Haskell,frame=single]
(:)  :: a -> [a] -> [a]
(++) :: [a] -> [a] -> [a]
\end{lstlisting}


\texttt{(:)}: Cons: Adds single element to beginning of a list.

\texttt{(++)}: Appends two lists.

\begin{lstlisting}[language=Haskell,frame=single]
map    :: (a -> b) -> [a] -> [b]
filter :: (a -> Bool) -> [a] -> [a]
\end{lstlisting}


\texttt{map}: Maps a function over a list. The more general function
\texttt{fmap} (infix \texttt{<\$>} works for all functor types).

\texttt{filter}: Returns a list of elements for which a predicate
holds.

\begin{lstlisting}[language=Haskell,frame=single]
head :: [a] -> a
last :: [a] -> a
tail :: [a] -> [a]
init :: [a] -> [a]
\end{lstlisting}

\texttt{head, last}: Returns the first, last element respectively.

\texttt{tail, init}: Returns all elements but the first, last respectively.

\begin{lstlisting}[language=Haskell,frame=single]
null :: [a] -> Bool
\end{lstlisting}


Returns \texttt{True} if and only if the given list is empty.

\begin{lstlisting}[language=Haskell,frame=single]
reverse :: [a] -> [a]
\end{lstlisting}


\begin{lstlisting}[language=Haskell,frame=single]
intersperse :: a -> [a] -> [a]
intercalate :: [a] -> [[a]] -> [a]
\end{lstlisting}


\emph{Requires }\texttt{\emph{Data.List}}\emph{.}

\texttt{intersperse}: Given an element and a list: Returns a list
with element interspersed between each element of the list.

\texttt{intercalate}: Intersperses elements between each list in a
set of lists.

\begin{lstlisting}[language=Haskell,frame=single]
transpose :: [[a]] -> [[a]]
\end{lstlisting}


\emph{Requires }\texttt{\emph{Data.List}}\emph{.} E.g. \texttt{transpose}
$[[1,2,3,4],[11,12],[1,2,13],[5]]=[[1,11,1,5],[2,12,2],[3,13],[4]]$

\begin{lstlisting}[language=Haskell,frame=single]
subsequences :: [a] -> [[a]]
permutations :: [a] -> [[a]]
\end{lstlisting}


\texttt{subsequences}: Returns a list of all subsequences.

\texttt{permutations}: Returns a list of all permutations.


\subsubsection{Sublists}

\begin{lstlisting}[language=Haskell,frame=single]
take    :: Int -> [a] -> [a]
drop    :: Int -> [a] -> [a]
splitAt :: Int -> [a] -> ([a], [a])
\end{lstlisting}


\texttt{splitAt n xs}: \texttt{take n xs} and \texttt{drop n xs}.

\begin{lstlisting}[language=Haskell,frame=single]
takeWhile :: (a -> Bool) -> [a] -> [a]
dropWhile :: (a -> Bool) -> [a] -> [a]
span      :: (a -> Bool) -> [a] -> ([a], [a])
break     :: (a -> Bool) -> [a] -> ([a], [a])
\end{lstlisting}


\textbf{Don't forget to use }\texttt{\textbf{takeWhile}}\textbf{ and
}\texttt{\textbf{dropWhile}}\textbf{!}

\texttt{span}: Given a predicate and a list: Returns a tuple where
the first element is the longest prefix of the list that satisfy the
predicate and the second element is the remainder.

\texttt{break}: As for \texttt{span}, but for the longest prefix that
does not satisfy the predicate.

\begin{lstlisting}[language=Haskell,frame=single]
dropWhileEnd :: (a -> Bool) -> [a] -> [a]
\end{lstlisting}


\emph{Requires }\texttt{\emph{Data.List}}\emph{. }Drops the largest
suffix of the list for which the predicate holds.

\begin{lstlisting}[language=Haskell,frame=single]
stripPrefix :: [a] -> [a] -> Maybe [a]
\end{lstlisting}


\emph{Requires }\texttt{\emph{Data.List}}\emph{. }Drops the given
prefix from a list, or returns \texttt{Nothing} if the list did not
start with the given prefix.

\begin{lstlisting}[language=Haskell,frame=single]
group :: [a] -> [[a]]
\end{lstlisting}


\emph{Requires }\texttt{\emph{Data.List}}\emph{. }Given a list; Returns
a list of lists, that when concatenated reform the original list,
and such that each sublist only contains equal elements.

\begin{lstlisting}[language=Haskell,frame=single]
inits :: [a] -> [[a]]
tails :: [a] -> [[a]]
\end{lstlisting}


\emph{Requires }\texttt{\emph{Data.List}}\emph{.}

\texttt{inits}: Returns all initial segments, shortest first.

\texttt{tails}: Returns all final segments, shortest first.


\subsubsection{Predicates}

\begin{lstlisting}[language=Haskell,frame=single]
isPrefixOf      :: [a] -> [a] -> Bool
isSuffixOf      :: [a] -> [a] -> Bool
isInfixOf       :: [a] -> [a] -> Bool
isSubsequenceOf :: [a] -> [a] -> Bool
\end{lstlisting}


\emph{Requires }\texttt{\emph{Data.List}}\emph{.}

\texttt{isPrefixOf}: Returns \texttt{True} if the first list is a
prefix of the second.

\texttt{isSuffixOf}: Returns \texttt{True} if the first list is a
suffix of the second.

\texttt{isInfixOf}: Returns \texttt{True} if the first list is contained,
wholly and intact, in the second.

\texttt{isSubsequenceOf}: Returns \texttt{True} if the first list
is contained, in order, in the second (elements not necessarily consecutive).


\subsubsection{Indexing Lists}

\begin{lstlisting}[language=Haskell,frame=single]
(!!) :: [a] -> Int -> a
\end{lstlisting}


Returns element at given index.

\begin{lstlisting}[language=Haskell,frame=single]
elemIndex   :: a -> [a] -> Maybe Int
elemIndices :: a -> [a] -> [Int]
findIndex   :: (a -> Bool) -> [a] -> Maybe Int
findIndices :: (a -> Bool) -> [a] -> [Int]
\end{lstlisting}


\emph{Requires }\texttt{\emph{Data.List}}\emph{.}

\texttt{elemIndex, findIndex}: Returns the index of the first element
in the given list that is equal to the given value, satisfies the
given predicate respectively; Or returns \texttt{Nothing} if none
exists.

\texttt{elemIndices, findIndices},: Returns the indices of all elements
in the given list that are equal to the given value, satisfy the given
predicate respectively.


\subsubsection{Searching Lists}

\begin{lstlisting}[language=Haskell,frame=single]
elem    :: a -> [a] -> Bool
notElem :: a -> [a] -> Bool
\end{lstlisting}


\texttt{elem, notElem}: Returns \texttt{True} if the given element
is, is not an element of the given list respectively.

\begin{lstlisting}[language=Haskell,frame=single]
lookup :: a -> [(a, b)] -> Maybe b
\end{lstlisting}


Looks up a key in a dictionary; Returns \texttt{Nothing} if key not
found.

\begin{lstlisting}[language=Haskell,frame=single]
find :: (a -> Bool) -> [a] -> Maybe a
\end{lstlisting}


\emph{Requires }\texttt{\emph{Data.List}}\emph{.} Returns the first
element that satisfies the predicate, or \texttt{Nothing} if none
exists.

\begin{lstlisting}[language=Haskell,frame=single]
partition :: (a -> Bool) -> [a] -> ([a], [a])
\end{lstlisting}


\emph{Requires }\texttt{\emph{Data.List}}\emph{.} Returns the pair
of lists of elements which do and do not satisfy the predicate respectively.


\subsubsection{Zipping and Unzipping Lists}

\begin{lstlisting}[language=Haskell,frame=single]
zip  :: [a] -> [b] -> [(a, b)]
zip3 :: [a] -> [b] -> [c] -> [(a, b, c)]
\end{lstlisting}


\begin{lstlisting}[language=Haskell,frame=single]
zipWith  :: (a -> b -> c) -> [a] -> [b] -> [c]
zipWith3 :: (a -> b -> c -> d) -> [a] -> [b] -> [c] -> [d]
\end{lstlisting}


\begin{lstlisting}[language=Haskell,frame=single]
unzip  :: [(a, b)] -> ([a], [b])
unzip3 :: [(a, b, c)] -> ([a], [b], [c])
\end{lstlisting}



\subsection{Building Lists}


\subsubsection{Building Lists}

\begin{lstlisting}[language=Haskell,frame=single]
scanr          :: (a -> b -> b) -> b -> [a] -> [b]
scanl          :: (b -> a -> b) -> b -> [a] -> [b]
scanr1, scanl1 :: (a -> a -> a) -> [a] -> [a]
\end{lstlisting}


Returns a list of successive values from respective fold function.

\begin{lstlisting}[language=Haskell,frame=single]
unfoldr :: (b -> Maybe (a, b)) -> b -> [a]
\end{lstlisting}


Given a function that produces a \texttt{Maybe} pair and a starting
value; Applies the function to the starting value; If the function
returns a \texttt{Just} pair, the first element is added to the resulting
list and the function is applied again to the second element; If the
function returns \texttt{Nothing}, the resulting list is returned.

\begin{lstlisting}[language=Haskell,frame=single]
mapAccumR (or L) :: (a -> b -> (a, c)) -> a -> [b] -> (a, [c])
\end{lstlisting}


Given a function to a pair of an accumulator and a result, and a starting
value for the accumulator; Returns a pair of the final accumulator
and the list of results.


\subsubsection{Infinite Lists}

\begin{lstlisting}[language=Haskell,frame=single]
iterate   :: (a -> a) -> a -> [a]
repeat    :: a -> [a]
replicate :: Int -> a -> [a]
cycle     :: [a] -> [a]
\end{lstlisting}


\texttt{iterate}: Returns an infinite list of applications of a function
to a starting value.

\texttt{repeat}: Returns an infinite list with each element of the
given value.

\texttt{replicate}: Returns a list of given length of a repeated element.

\texttt{cycle}: Returns an infinite list produced by cycling the given
list.


\subsection{Reducing Lists}


\subsubsection{Folds}

\begin{lstlisting}[language=Haskell,frame=single]
foldr  :: (a -> b -> b) -> b -> [a] -> b
foldl  :: (b -> a -> b) -> b -> [a] -> b
foldr1 :: (a -> a -> a) -> [a] -> a
foldl1 :: (a -> a -> a) -> [a] -> a
\end{lstlisting}


\textbf{Don't forget to use folds!}

Given a binary operator, starting value and list: Reduces list using
the binary operator in the given direction, starting from the given
starting value.

e.g. \texttt{foldr} $f$ $x_{0}$ $[x_{1},x_{2},x_{3},\dots,x_{n}]$
$=$ $f$ $x_{1}(f$ $x_{2}(f$ $x_{3}($ $\dots$ $(f$ $x_{n}$
$x_{0}))))$,

and \texttt{foldl} $f$ $x_{0}$ $[x_{1},x_{2},x_{3},\dots,x_{n}]$
$=$ $(f$ $(f$ $(f$ $(f$ $x_{0}$ $x_{1})$ $x_{2})$ $x_{3})$$\dots$
$x_{n})$.

\texttt{fold1}: Has no starting value.


\subsubsection{Special Folds}

\begin{lstlisting}[language=Haskell,frame=single]
length  :: [a] -> Int
sum     :: Num a => [a] -> a
product :: Num a => [a] -> a
\end{lstlisting}


\begin{lstlisting}[language=Haskell,frame=single]
maximum :: [a] -> a
minimum :: [a] -> a
\end{lstlisting}


\begin{lstlisting}[language=Haskell,frame=single]
or  :: [Bool] -> Bool
and :: [Bool] -> Bool
any :: (a -> Bool) -> [a] -> Bool
all :: (a -> Bool) -> [a] -> Bool
\end{lstlisting}


\texttt{or, any}: Returns \texttt{True} if \texttt{True} for any element.

\texttt{and, all}: Returns \texttt{True} if \texttt{True} for all
elements.

\begin{lstlisting}[language=Haskell,frame=single]
concat    :: [[a]] -> [a]
concatMap :: (a -> [b]) -> [a] -> b
\end{lstlisting}


\texttt{concat}: Concatenates a list of lists into a list.

\texttt{concatMap}: Maps a function, then concatenates result.


\subsection{Special Lists}


\subsubsection{Set Functions}

\begin{lstlisting}[language=Haskell,frame=single]
nub :: [a] -> [a]
\end{lstlisting}


\emph{Requires }\texttt{\emph{Data.List}}\emph{.} Removes duplicates.

\begin{lstlisting}[language=Haskell,frame=single]
delete :: a -> [a] -> [a]
(\\)   :: [a] -> [a] -> [a]
\end{lstlisting}


\emph{Requires }\texttt{\emph{Data.List}}\emph{.} Deletes first occurrence
of given element, each element of given list.

\begin{lstlisting}[language=Haskell,frame=single]
union     :: [a] -> [a] -> [a]
intersect :: [a] -> [a] -> [a]
\end{lstlisting}


\emph{Requires }\texttt{\emph{Data.List}}\emph{.} Returns union, intersection
of two lists.


\subsubsection{Ordered Lists}

\begin{lstlisting}[language=Haskell,frame=single]
sort :: [a] -> [a]
\end{lstlisting}


\emph{Requires }\texttt{\emph{Data.List}}\emph{.} Sorts list.

\begin{lstlisting}[language=Haskell,frame=single]
sortOn :: (a -> b) -> [a] -> [a]
\end{lstlisting}


\emph{Requires }\texttt{\emph{Data.List}}\emph{.} Sorts list by comparing
values generated by given function.

\begin{lstlisting}[language=Haskell,frame=single]
insert :: a -> [a] -> [a]
\end{lstlisting}


\emph{Requires }\texttt{\emph{Data.List}}\emph{.} Inserts element
so that a sorted list remains sorted.


\subsubsection{Functions on Strings}

\begin{lstlisting}[language=Haskell,frame=single]
lines   :: String -> [String]
words   :: String -> [String]
unlines :: [String] -> String
unwords :: [String] -> String
\end{lstlisting}


\texttt{words}: Breaks up into a list of words, delimited by white
space.

\texttt{lines}: Breaks up into a list of strings separated by newlines.


\section{Functions}

\begin{lstlisting}[language=Haskell,frame=single]
id       :: a -> a
const    :: a -> b -> a
asTypeOf :: a -> a -> a
\end{lstlisting}


\texttt{id}: Identity function.

\texttt{const}: Evaluates to constant value for all inputs.

\texttt{asTypeOf}: Forces first argument to have the same type as
the second.

\begin{lstlisting}[language=Haskell,frame=single]
(.) :: (b -> c) -> (a -> b) -> a -> c
($) :: (a -> b) -> a -> b
\end{lstlisting}


\texttt{(.)}: Function composition: $(f.g)\:x=f(g\:x)$. Note the
requirements for the type and number of arguments. You can use this
to combine two higher order functions into one.

\texttt{(\$)}: Function application: $f\:\$\:g\:\$\:h\:x=f(g(h\:x))$.
Often used to omit parentheses. To be avoided in general!

\begin{lstlisting}[language=Haskell,frame=single]
flip :: (a -> b -> c) -> b -> a -> c
\end{lstlisting}


Use to invert arguments. Alternative to back-quotes or lambdas.

\begin{lstlisting}[language=Haskell,frame=single]
until :: (a -> Bool) -> (a -> a) -> a -> a
\end{lstlisting}


Given a predicate and function; Applies function until predicate holds.

\begin{lstlisting}[language=Haskell,frame=single]
error "Error string"
\end{lstlisting}


Stops execution and displays error message.


\section{Types and Common Type Classes}

Use \texttt{data} to define a new data-type (you \emph{must} define
type constructors), \texttt{type} to define a type synonym, \texttt{class}
to define a new type-class, and \texttt{instance} to make a type an
instance of a type-class.

Don't forget you can use \texttt{deriving} in most cases. Example:

\begin{lstlisting}[language=Haskell,frame=single]
data Tree a = Empty | Leaf a | Node a (Tree a) (Tree a) 
            deriving (Show) 
\end{lstlisting}
Remember you can use \texttt{:i} in ghci for information about types
and type-classes.


\section{Syntactic Features and Good Practice}


\paragraph{Spacing}

Use proper spacing and alignment. Keep lines short.


\paragraph{Function Names}

Give functions meaningful names. Don't ever give two different things
the same name!


\paragraph{Comments}

Use to keep code ordered and to explain complicated functions. Begin
with \texttt{-{}-}. Multli-line comments use \texttt{\{-} and \texttt{-\}}.


\paragraph{Multi-Line Strings}

End and start each line with \texttt{\textbackslash{}}.


\paragraph{Where clauses}

Use instead of \texttt{let} statements to avoid computing something
multiple times or to clean up code.


\paragraph{Helper Functions}

Use helper functions to provide complicated functionality (e.g. creating
a list \emph{and} checking for convergence as it is created). Accumulating
parameters are also particularly useful:

\begin{lstlisting}[language=Haskell,frame=single]
function input = function' input startingAcc
  where
    function' baseCase  acc = acc
    function' otherCase acc = function nextCase updatedAcc
\end{lstlisting}



\paragraph{Lambdas}

Use as an alternative to higher-order functions for very complicated
functions.


\paragraph*{List comprehensions}

Often functions involving lists can be written using comprehensions,
recursion, or higher-order functions. Choose carefully.

\begin{lstlisting}[language=Haskell,frame=single]
[f x | x <- xs, p] = map f $ filter p xs
\end{lstlisting}



\paragraph{Pattern matching}

Use often. Especially helpful for breaking up lists using \texttt{:}
and for working with tuples. Don't forget to use \texttt{\_} and \texttt{@}
where appropriate.


\paragraph{Guards}

Use instead of \texttt{if} ... \texttt{then} ... \texttt{else} statements.


\paragraph*{Tips for Lexis Tests}
\begin{itemize}
\item Read instructions carefully. Don't rewrite a given function!
\item Be prepared to write very little for the first questions and much
more (helper functions, etc.) towards the end. You should check your
answers to the first parts before beginning the last part. You should
spend time planning your answer (on paper) for the last part.
\item Aim firstly to get a working implementation, then consider optimisations.
\item Don't forget to use functions defined earlier in the program.
\item Test often (using \texttt{undefined} where necessary to allow compilation)
and read any error messages carefully.
\item Make use of any provided test cases but do not rely upon them.
\item Use an editor with syntax highlighting and \texttt{ghci} in terminal.
\item Use \texttt{:set -W} to turn on warnings and \texttt{:set -w} to turn
off warnings in \texttt{ghci}. Use \texttt{:browse} to see a list of all functions included in a module.
\end{itemize}
\end{document}
