\documentclass[10pt,twoside,twocolumn]{article}
\usepackage[latin9]{inputenc}
\usepackage[landscape]{geometry}
\geometry{verbose,tmargin=0.5in,bmargin=0.75in,lmargin=0.5in,rmargin=0.5in}
\setlength{\parskip}{\smallskipamount}
\setlength{\parindent}{0pt}
\usepackage{calc}
\usepackage{amsmath}
\usepackage{amssymb}
\usepackage{esint}

\makeatletter




\usepackage{calc}




\usepackage{calc}


\usepackage{amsthm}
\usepackage{mathrsfs}
\usepackage{amsfonts}
\usepackage{dsfont}


\setlength{\columnsep}{0.25in}

\renewcommand*{\thefootnote}{\fnsymbol{footnote}}

\newcommand{\R}[0]{\mathds{R}} % real numbers
\newcommand{\Z}[0]{\mathds{Z}} % integers
\newcommand{\N}[0]{\mathds{N}} % natural numbers
\newcommand{\nat}[0]{\mathds{N}} % natural numbers
\newcommand{\Q}[0]{\mathds{Q}} % rational numbers

\makeatother

\begin{document}

\title{Reference Sheet for CO145 Mathematical Methods}


\date{Autumn 2016}

\maketitle

\part{Analysis}


\section{Sets}

For a set $S$ of real numbers: 
\begin{enumerate}
\item $u$ is an upper bound if $u\geq s\:\forall\:s\in S$. If such a $u$
exists, $S$ is bounded above. 
\item $l$ is an lower bound if $l\leq s\:\forall\:s\in S$. If such a $l$
exists, $S$ is bounded below. 
\end{enumerate}
Every set, $S$ has a supremum and infimum: 
\begin{enumerate}
\item $\sup(S)$ is the least upper bound of $S$. 
\item $\inf(S)$ is the greatest lower bound of $S$. 
\end{enumerate}

\paragraph{Proving the Convergence of a Bounded Sequence.}

Every \emph{increasing} sequence of real numbers that is \emph{bounded
above} must converge (this is especially useful when combined with
proof by induction).


\section{Sequences}


\subsection{Given Results}


\subsubsection{Absolute Values}
\begin{enumerate}
\item $\left|xy\right|=\left|x\right|\times\left|y\right|$ 
\item $\left|\frac{x}{y}\right|=\frac{\left|x\right|}{\left|y\right|}$ 
\end{enumerate}

\subsubsection{Converging Sequences}
\begin{enumerate}
\item $\lim_{n\rightarrow\infty}\frac{1}{n^{c}}=0$ for all $c>0$ 
\item $\lim_{n\rightarrow\infty}\frac{1}{c^{n}}=0$ for all $\left|c\right|>0$ 
\item $\lim_{n\rightarrow\infty}\frac{1}{n!}=0$ 
\item $\lim_{n\rightarrow\infty}\frac{1}{\log n}=0$ for $n>1$ 
\end{enumerate}

\subsubsection{Combinations of Sequences}
\begin{enumerate}
\item $\lim_{n\rightarrow\infty}\left(\lambda a_{n}\right)=\lambda\lim_{n\rightarrow\infty}a_{n}$ 
\item $\lim_{n\rightarrow\infty}\left(a_{n}+b_{n}\right)=\lim_{n\rightarrow\infty}a_{n}+\lim_{n\rightarrow\infty}b_{n}$ 
\item $\lim_{n\rightarrow\infty}\left(a_{n}-b_{n}\right)=\lim_{n\rightarrow\infty}a_{n}-\lim_{n\rightarrow\infty}b_{n}$ 
\item $\lim_{n\rightarrow\infty}\left(a_{n}\times b_{n}\right)=\lim_{n\rightarrow\infty}a_{n}\times\lim_{n\rightarrow\infty}b_{n}$ 
\item $\lim_{n\rightarrow\infty}\frac{a_{n}}{b_{n}}=\frac{\lim_{n\rightarrow\infty}a_{n}}{\lim_{n\rightarrow\infty}b_{n}}$ 
\end{enumerate}

\subsection{Proving Convergence}


\subsubsection{Direct Proof}
\begin{enumerate}
\item Apply the definition of convergence:\\
 %
\fbox{\begin{minipage}[t]{1\linewidth - 2\fboxsep - 2\fboxrule}Suppose
$a_{n}$ converges to $L$. Then there must exist an $N\left(\epsilon\right)\in\R$
such that for all $n\geq N\left(\epsilon\right)$ and for all $\epsilon>0$,
$*$ holds: 
\begin{align*}
\left|a_{n}-L\right|<\epsilon & *
\end{align*}
\end{minipage}} 
\item Rearrange $*$ such that $n$ is the subject, ensuring that you show
implications in the correct direction. 
\item Propose an $N(\epsilon)$ such that $*$ holds (the ceiling function
applied to the previous result is sufficient). 
\item Check that this result is sensible. 
\end{enumerate}

\subsubsection{Sandwich Theorem}
\begin{enumerate}
\item Show that $\lim_{n\rightarrow\infty}l_{n}=L$. 
\item Show that $\lim_{n\rightarrow\infty}u_{n}=L$. 
\item Show that $l_{n}\leq a_{n}$ and $u_{n}\geq a_{n}$ for all $n\geq N$
for some $N\in\R$. 
\item Apply the sandwich theorem to show that $\lim_{n\rightarrow\infty}a_{n}=L$. 
\end{enumerate}

\subsubsection{Ratio Test}
\begin{enumerate}
\item Determine the value of $\lim_{n\rightarrow\infty}\left|\frac{a_{n+1}}{a_{n}}\right|$. 
\item Conclude appropriately.

\begin{enumerate}
\item If $r<1$ then $a_{n}$ converges to 0. 
\item If $r>1$ then $a_{n}$ diverges. 
\end{enumerate}
\end{enumerate}

\paragraph{Proving the convergence of a sequence $b_{n}$ to $L\protect\neq0$}

Consider the ratio of the modified sequence $a_{n}=b_{n}-L$.


\section{Series}


\subsection{Given Results}


\subsubsection{Converging Series}
\begin{enumerate}
\item $\sum_{n=1}^{\infty}x^{n}$ converges to $\frac{x}{1-x}$ for $\left|x\right|<1$.
{[}Geometric Series{]} 
\item $\sum_{n=1}^{\infty}\frac{1}{n^{2}}$ converges to $\frac{\pi}{6}$.
{[}Inverse Square Series{]} 
\item $\sum_{n=1}^{\infty}\frac{1}{n^{c}}$ converges for $c>1$. 
\end{enumerate}

\subsubsection{Diverging Series}
\begin{enumerate}
\item $\sum_{n=1}^{\infty}x^{n}$ diverges for $\left|x\right|>1$. {[}Geometric
Series{]} 
\item $\sum_{n=1}^{\infty}\frac{1}{n}$ diverges. {[}Harmonic Series{]} 
\item $\sum_{p:\text{prime}}\frac{1}{p}$ diverges. {[}Harmonic Primes{]} 
\end{enumerate}

\subsection{Proving Convergence}


\subsubsection{Using Partial Sums}
\begin{enumerate}
\item Construct the partial sum, $S_{n}=\sum_{i=1}^{n}a_{i}$. 
\item Prove the convergence or divergence of the partial sum. 
\end{enumerate}

\subsubsection{Comparison Test}


\paragraph{Standard Comparison Test}
\begin{enumerate}
\item Show that $\sum_{i=1}^{\infty}c_{i}$ converges or that $\sum_{i=1}^{\infty}d_{i}$
diverges. 
\item Make a comparison and conclude appropriately.

\begin{enumerate}
\item Show that $a_{i}\leq\lambda c_{i}$ for some $\lambda>0$ and for
all $i>N\in\R$ and conclude $\sum_{n=1}^{\infty}a_{i}$ converges. 
\item Show that $a_{i}\geq\lambda d_{i}$ for some $\lambda>0$ and for
all $i>N\in\R$ and conclude $\sum_{n=1}^{\infty}a_{i}$ diverges. 
\end{enumerate}
\end{enumerate}

\paragraph{Limit Comparison Test}
\begin{enumerate}
\item Show that $\sum_{i=1}^{\infty}c_{i}$ converges or that $\sum_{i=1}^{\infty}d_{i}$
diverges. 
\item State the appropriate limit and conclusion.

\begin{enumerate}
\item Show that $\lim_{i\rightarrow\infty}\frac{a_{i}}{c_{i}}$ exists and
conclude $\sum_{n=1}^{\infty}a_{i}$ converges. 
\item Show that $\lim_{i\rightarrow\infty}\frac{d_{i}}{a_{i}}$ exists and
conclude $\sum_{n=1}^{\infty}a_{i}$ diverges. 
\end{enumerate}
\end{enumerate}

\subsubsection{Ratio Test}
\begin{enumerate}
\item Determine $\lim_{n\rightarrow\infty}\frac{a_{i+1}}{a_{i}}$. 
\item Conclude appropriately.

\begin{enumerate}
\item If $\lim_{n\rightarrow\infty}\frac{a_{i+1}}{a_{i}}<1$ then $\sum_{i=1}^{\infty}a_{i}$
converges. 
\item If $\lim_{n\rightarrow\infty}\frac{a_{i+1}}{a_{i}}>1$ then $\sum_{i=1}^{\infty}a_{i}$
diverges. 
\end{enumerate}
\end{enumerate}

\subsubsection{Integral Test}

Where $f\left(n\right)=a_{n}$ is a continuous, positive, and decreasing
function: 
\begin{enumerate}
\item Determine $\int_{N}^{\infty}f\left(x\right)\text{d}x$. 
\item Conclude appropriately.

\begin{enumerate}
\item If $\int_{N}^{\infty}f\left(x\right)\text{d}x$ converges, so does
$\sum_{n=N}^{\infty}a_{n}$. 
\item If $\int_{N}^{\infty}f\left(x\right)\text{d}x$ diverges, so does
$\sum_{n=N}^{\infty}a_{n}$. 
\end{enumerate}
\end{enumerate}

\section{Power Series}


\subsection{Given Results}
\begin{enumerate}
\item $e^{x}=1+x+\frac{x^{2}}{2!}+\frac{x^{3}}{3!}+\dots+\frac{x^{r}}{r!}+\dots$
for all $x$ 
\item $\ln\left(1+x\right)=x-\frac{x^{2}}{2}+\frac{x^{3}}{3}-\dots+\left(-1\right)^{r+1}\frac{x^{r}}{r}+\dots$
for $-1<x\leq1$ 
\item $\sin x=x-\frac{x^{3}}{3!}+\frac{x^{5}}{5!}-\dots+\left(-1\right)^{r}\frac{x^{2r+1}}{\left(2r+1\right)!}+\dots$
for all $x$ 
\item $\cos x=1-\frac{x^{2}}{2!}+\frac{x^{4}}{4!}-\dots+\left(-1\right)^{r}\frac{x^{2r}}{\left(2r\right)!}+\dots$
for all $x$ 
\end{enumerate}

\paragraph{Combinations of Functions}

Note that you can add, subtract, multiply, divide, differentiate and
integrate power series term-wise.


\subsection{Determining a Maclaurin Series}
\begin{enumerate}
\item Repeatedly differentiate $f\left(x\right)$ and set $x$ to 0. Use
these results to propose a value for $f^{\left(n\right)}0$. 
\item The Maclaurin series expansion for $f\left(x\right)$ is then given
by $\sum_{n=0}^{\infty}f^{\left(n\right)}\left(0\right)\frac{x^{n}}{n!}$. 
\end{enumerate}

\subsection{Determining a Taylor Series}
\begin{enumerate}
\item Repeatedly differentiate $f\left(x\right)$ and set $x$ to $a$.
Use these results to propose a value for $f^{\left(n\right)}a$. 
\item The Taylor series expansion at $a$ for $f\left(x\right)$ is then
given by $\sum_{n=0}^{\infty}\frac{f^{\left(n\right)}\left(a\right)}{n!}\left(x-a\right)^{n}$. 
\end{enumerate}

\subsection{Finding the Radius of Convergence}

\emph{Radius of Convergence}: Size of $x$ set for which a power series
converges.
\begin{enumerate}
\item Apply a ratio test using absolute values. 
\item Set $\left|r\right|<1$ to show the region of convergence. 
\item If the series converges for $\left|x-a\right|<R$, the series has
a radius of convergence of $R$ (about $a$). 
\end{enumerate}

\subsection{Error Terms}

Where $c$ is a constant that lies between $x$ and $a$:


\subsubsection{Lagrange Error Term}

\begin{align*}
f\left(x\right) & =\sum_{n=0}^{\infty}\frac{f^{\left(n\right)}\left(a\right)}{n!}\left(x-a\right)^{n}\\
 & =\sum_{n=0}^{k}\frac{f^{\left(n\right)}\left(a\right)}{n!}\left(x-a\right)^{n}+\underbrace{\frac{f^{\left(k+1\right)}\left(c\right)}{\left(k+1\right)!}\left(x-a\right)^{k+1}}_{\text{Lagrange Error Term}}
\end{align*}


This result follows from the mean value theorem.


\subsubsection{Cauchy Error Term}

\begin{align*}
f\left(x\right) & =\sum_{n=0}^{k}\frac{f^{\left(n\right)}\left(a\right)}{n!}\left(x-a\right)^{n}+\underbrace{\frac{f^{\left(k+1\right)}\left(c\right)}{k!}\left(x-c\right)^{k}\left(x-a\right)}_{\text{Cauchy Error Term}}
\end{align*}



\subsection{Using Power Series to Solve ODEs}

\textbf{Example:} Given the differential equation, $\frac{\text{d}y}{\text{d}x}=ky$: 
\begin{enumerate}
\item Express $y$ as the power series, $\sum_{i=0}^{\infty}a_{i}x^{i}$. 
\item Differentiate $y$ to give $\frac{\text{d}y}{\text{d}x}=\sum_{i=0}^{\infty}ia_{i}x^{i-1}$. 
\item Compare coefficients to give $a_{i}=\frac{k}{i}a_{i-1}$. 
\item Deduce that $a_{i}=\frac{k^{i}}{i!}a_{0}$. Hence $y=a_{0}e^{kx}$
(where $a_{0}$ is the value of $x$ at $y=0$). 
\end{enumerate}
\pagebreak{}


\part{Linear Algebra}


\section{Introduction}
\begin{enumerate}
\item \emph{Vectors} can be added together or multiplied by scalars to produce
another vector. 
\item \emph{Examples of vector objects}: Geometric vectors, polynomials,
$\R^{n}$, audio signals.
\end{enumerate}

\section{Groups}

For $\left(G,\cdot\right)$ to be called a group, it must have the
properties: 
\begin{enumerate}
\item \textbf{Closure:} $\forall\:x,y\in G$, $x\cdot y\in G$ 
\item \textbf{Associativity:} $\forall\:x,y,z\in G$, $\left(x\cdot y\right)\cdot z=x\cdot\left(y\cdot z\right)$ 
\item \textbf{Identity:} $\exists\;e\in G$ s.t. $\forall\:x\in G$, $x\cdot e=e\cdot x=x$ 
\item \textbf{Inverse:} $\forall\:x\in G\:\exists\:x^{-1}\in G$ s.t. $x\cdot x^{-1}=x^{x-1}\cdot x=e$ 
\end{enumerate}
For the group to be \emph{abelian} it must also have the additional
property: 
\begin{enumerate}
\item \textbf{Commutativity:} $\forall\:x,y\in G$, $x\cdot y=y\cdot x$ 
\end{enumerate}

\section{Methods on Matrices}


\subsection{Multiplication by a Scalar}

Has the properties: 
\begin{enumerate}
\item Associativity 
\item Distributivity 
\item $\left(\lambda\mathbf{C}\right)^{\top}=\lambda\mathbf{C}^{\top}$ 
\end{enumerate}

\subsection{Matrix Multiplication}

Matrix multiplication is defined for $\mathbf{A}\in\R^{m\times n},\mathbf{B}\in\R^{n\times p}$,
giving the product, $\mathbf{C}=\mathbf{AB}\in\R^{m\times p}$. It
has the properties: 
\begin{enumerate}
\item Associativity 
\item Distributivity 
\item $\forall\:\mathbf{A}\in\R^{m\times n}$, $\mathbf{I}_{m}\mathbf{A}=\mathbf{A}\mathbf{I}_{n}=\mathbf{A}$. 
\item Note that $\mathbf{AB}\neq\mathbf{BA}$. 
\end{enumerate}

\subsection{Inverse and Transpose}


\subsubsection{Further Properties of Matrix Arithmetic}
\begin{enumerate}
\item $\mathbf{A}\mathbf{A}^{-1}=\mathbf{A}^{-1}\mathbf{A}=\mathbf{I}$ 
\item Note $\left(\mathbf{A}+\mathbf{B}\right)^{-1}\neq\mathbf{A}^{-1}+\mathbf{B}^{-1}$ 
\item $\left(\mathbf{A}\mathbf{B}\right)^{-1}=\mathbf{B}^{-1}\mathbf{A}^{-1}$\\

\item $\left(\mathbf{A}^{\top}\right)^{\top}=\mathbf{A}$ 
\item $\left(\mathbf{A}+\mathbf{B}\right)^{\top}=\mathbf{A}^{\top}+\mathbf{B}^{\top}$ 
\item $\left(\mathbf{A}\mathbf{B}\right)^{\top}=\mathbf{B}^{\top}\mathbf{A}^{\top}$\\

\item $\left(\mathbf{A}^{-1}\right)^{\top}=\left(\mathbf{A}^{\top}\right)^{-1}$ 
\end{enumerate}

\subsubsection{Determining the Inverse of a Matrix}
\begin{enumerate}
\item Recall that $\mathbf{A}\mathbf{A}^{-1}=\mathbf{I}$. We therefore
wish to solve $\mathbf{A}\mathbf{X}=\mathbf{I}$. 
\item Starting with $\left[\mathbf{A}|\mathbf{I}_{n}\right]$, use Gaussian
elimination to reach reduced row echelon form. 
\item $\mathbf{A}^{-1}$ can be read off from the result $\left[\mathbf{I}_{n}|\mathbf{A}^{-1}\right]$. 
\end{enumerate}

\subsection{Gaussian Elimination}


\subsubsection{Elementary Transformations}

We are allowed to: 
\begin{enumerate}
\item Swap two rows. 
\item Multiply a row by a constant $\lambda\neq0$. 
\item Add a row to another row. 
\end{enumerate}
We use these transformations to reach (reduced) row echelon form.


\subsubsection{Row Echelon Form}
\begin{enumerate}
\item The pivot of a non-zero row is strictly to the right of a pivot of
the row above it. 
\item Any rows containing only zeros are at the bottom of the matrix. 
\end{enumerate}
Columns with pivots define \emph{basic variables}, other columns give
\emph{free variables}.


\subsubsection{Reduced Row Echelon Form}
\begin{enumerate}
\item The system is in row echelon form. 
\item Every pivot is 1. 
\item The pivot is the only non-zero entry in its column. 
\end{enumerate}

\subsection{Rank and Determinant}


\subsubsection{Finding the Rank of a Matrix}

\emph{Rank}: number of linearly independant columns of a matrix.
\begin{enumerate}
\item Apply Gaussian elimination to reach RREF. 
\item The rank is given by the number of pivots (linearly independent cols
/ rows). 
\end{enumerate}

\paragraph{Points to Note}
\begin{enumerate}
\item Column rank is equal to row rank. 
\item $\mathbf{A}$ is regular (invertible) $\iff$ $\mbox{rk}\left(\mathbf{A}\right)=n$. 
\item A matrix has full rank if its rank is equal to the lesser of the number
of rows and columns, or is rank-deficient otherwise. 
\end{enumerate}

\subsubsection{Finding the Determinant of a Matrix}
\begin{enumerate}
\item By the given transformations, bring the matrix into triangular form
(all 0 above or below the leading diagonal) or into a $2\times2$
form. 
\item The determinant is given the product of the leading diagonal of a
triangular matrix, or the product of the leading diagonal minus the
product of the other elements in a $2\times2$ matrix. 
\end{enumerate}

\paragraph{Transformations}
\begin{enumerate}
\item Adding and subtracting multiples of cols or rows does not change the
determinant. 
\item Multiplying a col or row by by a constant scales the determinant by
that constant. 
\item Swapping rows or cols changes the sign. 
\item \textbf{Laplace expansion} can be used to reduce no. of rows and cols
by one. Best used when there is a row or col with only one non-zero
entry. 
\end{enumerate}

\paragraph{Points to Note}
\begin{enumerate}
\item $\det\left(\mathbf{A}\mathbf{B}\right)=\det\left(\mathbf{A}\right)\times\det\left(\mathbf{B}\right)$ 
\item $\det\left(\mathbf{A}\right)=0\iff\mathbf{A}\mbox{ is singular}$ 
\item $\det\left(\mathbf{A}\right)=\det\left(\mathbf{A}^{\top}\right)$ 
\item $\det\left(\mathbf{A}^{-1}\right)=1/\det\left(\mathbf{A}\right)$ 
\item Similar matrices have the same determinant. 
\end{enumerate}

\section{Linear Equation Systems}


\subsection{Solving Inhomogeneous Linear Equation Systems}

For $\mathbf{A}\mathbf{x}=\mathbf{b}$: 
\begin{enumerate}
\item Build an augmented matrix from the system of equations. 
\item Use elementary transformations to reach row echelon form (you \emph{must}
justify every step). 
\item By reading from the resulting rows and setting free variables = 0,
find a particular solution. 
\item Solve the homogeneous linear equation system $\mathbf{A}\mathbf{x}=\mathbf{0}$,
using the row echelon form derived in 2. 
\item Combine the solutions from 3 and 4 to form the general solution. 
\end{enumerate}

\subsection{Solving Homogeneous Linear Equation Systems}

For $\mathbf{A}\mathbf{x}=\mathbf{0}$: 
\begin{enumerate}
\item Use elementary transformations to reach row echelon form as before. 
\item For each free variable (non-pivot col), equate it to a sum of basic
variables (pivot cols). 
\item Rearrange the equations formed in 1 so that they are of the form something
= 0. 
\item The solutions are given by the column vector of the coefficients of
the variables, multiplied by any real scalar value. 
\end{enumerate}

\subsubsection{The Minus-1 Trick}
\begin{enumerate}
\item Use Guassian elimination to reach reduced row echelon form. 
\item Extend the matrix from 1 by adding rows of the form $\left[0\:\dots\:0\;-1\;0\;\dots\;0\right]$
such that the leading diagonal is made up entirely of 1 (pivots) or
-1 (from the introduced rows). 
\item The columns containing -1 in the diagonal form the solutions. 
\end{enumerate}

\section{Vector Spaces}


\subsection{Defining Vector Spaces}


\subsubsection{Vector Spaces}

A vector space is a set $V$ with two operations: 
\begin{enumerate}
\item $V+V\rightarrow V$ (inner operation) 
\item $\R\cdot V\rightarrow V$ (outer operation) 
\end{enumerate}
where: 
\begin{enumerate}
\item $\left(V,+\right)$ is an Abelian group and has the distributivity
property. 
\item The outer operation has distributivity and associativity properties
and has a neutral element of 1. 
\end{enumerate}

\subsubsection{Vector Subspaces and Generating Sets}

For a vector space, $V$: 
\begin{enumerate}
\item If every vector in $V$ can be expressed as a linear combination of
$A\left\{ \mathbf{x}_{1},\mathbf{x}_{2},\dots,\mathbf{x}_{n}\right\} $,
then $A$ is a \emph{generating set} for $V$. 
\item For $U\subset V$ and $U\neq\emptyset$, $U$ is a \emph{vector subspace}
of $V$ if $U$ is a vector space. 
\end{enumerate}

\paragraph{Proving a Set is a Vector Subspace}

For a vector subspace $U$, we need to show: 
\begin{enumerate}
\item $U\neq\emptyset$. Equivalently, $0\in U$. 
\item Closure of $U$ with respect to the inner operation: $\forall\mathbf{x},\mathbf{y}\in U\left(\mathbf{x}+\mathbf{y}\in U\right)$. 
\item Closure of $U$ with respect to the outer operation: $\forall\lambda\in\R\forall\mathbf{x}\in U\left(\lambda\mathbf{x}\in U\right)$. 
\end{enumerate}

\subsection{Determining Linear Dependence or Independence}

\emph{Linear Dependence}: For a vector space $\mathbf{x}_{1},\dots,\dot{\mathbf{x}_{k}}$
here is a non-trivial linear combination such that $\mathbf{0}=\sum_{i=1}^{k}\lambda_{i}\mathbf{x}_{i}$.

To prove $\mathbf{x}_{1},\dots,\mathbf{x}_{k}$ are linearly independent: 
\begin{enumerate}
\item Write the vectors as columns of a matrix. 
\item Apply Gaussian elimination to reach RREF. 
\item Conclude appropriately:

\begin{enumerate}
\item Pivot columns are linearly independent of the previous vectors. 
\item Non-pivot columns can be expressed as linear combinations of previous
pivot columns. 
\end{enumerate}
\end{enumerate}

\paragraph{Points to Note}
\begin{enumerate}
\item If at least one of the vectors is $\mathbf{0}$ or at least two of
the vectors are identical then they are linearly dependent. 
\item The set of vectors, $\mathbf{x}_{1}=\sum_{i=1}^{k}\lambda_{i1}\mathbf{b}_{i},\dots,\mathbf{x}_{m}=\sum_{i=1}^{k}\lambda_{im}\mathbf{b}_{i}$
(where $\mathbf{b}_{1},\dots,\mathbf{b}_{k}$ are linearly independent)
are linearly independent if and only if $\lambda_{1},\dots,\lambda_{m}$
are linearly independent. 
\end{enumerate}

\subsection{Determining Bases and Dimensions}


\subsubsection{Determining a Basis and Dimension}

\emph{Basis}: Minimal (linearly independent) generating set for $V$.
It can be determined as follows: 
\begin{enumerate}
\item Write the spanning vectors as columns of a matrix. 
\item Apply Gaussian elimination to reach RREF. 
\item The original values of the pivot columns form a basis. 
\item The \emph{dimension} is the number of basis vectors. 
\end{enumerate}

\subsubsection{Determining a Simple Basis}
\begin{enumerate}
\item Write the spanning vectors as rows of a matrix. 
\item Apply Gaussian elimination to reach RREF. 
\item The rows with leading ones form a simple basis. 
\end{enumerate}

\subsubsection{Determining a Basis of the Intersection of Subspaces}

For $U_{1}=\left[\mathbf{b}_{1},\dots,\mathbf{b}_{k}\right]$ and
$U_{2}=\left[\mathbf{c}_{1},\dots,\mathbf{c}_{l}\right]$: 
\begin{enumerate}
\item Find the respective bases of $U_{1}$ and $U_{2}$. 
\item We want to solve $\sum_{i=1}^{k}\lambda_{i}\mathbf{b}_{i}=\sum_{j=1}^{l}\mu_{j}\mathbf{c}_{j}$,
i.e. $\sum_{i=1}^{k}\lambda_{i}\mathbf{b}_{i}-\sum_{j=1}^{l}\mu_{j}\mathbf{c}_{j}=\mathbf{0}$. 
\item This can be solved by the method for homogeneous linear equations,
where $\mathbf{b}_{1},\dots,\mathbf{b}_{k},-\mathbf{c}_{1},\dots,-\mathbf{c}_{l}$
form the columns of the augmented matrix. 
\item Solve for either $\lambda_{1},\dots,\lambda_{k}$ or $\mu_{1},\dots,\mu_{l}$
and determine the basis accordingly. 
\end{enumerate}
The \emph{linear hull} is the intersection of a set of subspaces.


\subsection{Affine Spaces}


\subsubsection{Defining Affine Spaces}

Affine spaces can be defined as: 
\begin{enumerate}
\item $L=\mathbf{x}_{0}+U$ 
\item Parametric Equation: $\exists\lambda_{1}\dots\lambda_{k}\forall\mathbf{x}\in L$
such that $\mathbf{x}=\mathbf{x}_{0}+\lambda_{1}\mathbf{b}_{1}+\dots+\lambda_{k}\mathbf{b}_{k}$ 
\end{enumerate}

\subsubsection{Finding the Intersection of Affine Spaces}
\begin{enumerate}
\item Recall that for $\mathbf{x}\in L_{1}$ and $\mathbf{x}\in L_{2}$,
$\mathbf{x}_{1}+\sum_{i=1}^{k}\lambda_{i}\mathbf{b}_{i}=\mathbf{x}=\mathbf{x}_{2}+\sum_{j=1}^{l}\mu_{j}\mathbf{c}_{j}$.
Hence $\sum_{i=1}^{k}\lambda_{i}\mathbf{b}_{i}-\sum_{j=1}^{l}\mu_{j}\mathbf{c}_{j}=\mathbf{x}_{2}-\mathbf{x}_{1}$.
This can be solved by the method for inhomogeneous linear equations,
where the basis vectors $\mathbf{b}_{1},\dots,\mathbf{b}_{k},-\mathbf{c}_{1},\dots,-\mathbf{c}_{l}$
form the columns of the augmented matrix. 
\item Determine the basis vectors. 
\item Solve the resulting inhomogeneous LEQS. 
\item Use the solution to determine a value for $\mathbf{x}$ using one
of the original equations. 
\item You can check your answer with the other equation. 
\end{enumerate}

\subsubsection{Determining Parallelism}

For $L_{1}=\mathbf{x}_{1}+U_{1}$ and $L_{2}=\mathbf{x}_{2}+U_{2}$,
$L_{1}||L_{2}$ if $U_{1}\subseteq U_{2}$ or $U_{2}\subseteq U_{1}$.


\section{Linear Mappings}


\subsection{Defining Linear Mappings}

To prove a mapping $\Phi$ is linear (a homomorphism), we must show
that: 
\begin{enumerate}
\item $\Phi\left(\mathbf{x}+\mathbf{y}\right)=\Phi\left(\mathbf{x}\right)+\Phi\left(\mathbf{y}\right)$ 
\item $\Phi\left(\lambda\mathbf{y}\right)=\lambda\Phi\left(\mathbf{x}\right)$ 
\end{enumerate}

\paragraph{Special Cases}
\begin{enumerate}
\item \textbf{Isomorphism:} also bijective.
\item \textbf{Endomorphism:} also maps from $V$ to $V$.
\item \textbf{Automorphism:} also maps from $V$ to $V$ and bijective.
\end{enumerate}

\paragraph{Points to Note}
\begin{enumerate}
\item For linear mappings $\Phi:V\rightarrow W$ and $\Psi:W\rightarrow X$,
the mapping $\Psi\circ\Phi:V\rightarrow X$ is also linear. 
\item If $\Phi$ is an isomorphism, then so is $\Phi^{-1}$. 
\item If $\Phi$ and $\Psi$ are linear, then so are $\Phi+\Psi$ and $\lambda\Phi$. 
\end{enumerate}

\subsection{Image and Kernel (Null Space)}

For a mapping $\Phi:\mathbf{x}\in V\rightarrow\mathbf{A}\mathbf{x}\in W$:


\paragraph{Determining the Image}

\emph{Image}: $\left\{ \mathbf{w}\in W|\exists\mathbf{v}\in V:\Phi\left(\mathbf{v}\right)=\mathbf{w}\right\} $:
Set of vectors in $W$ that can be reached by $\Phi$ from any vector
in $V$.

Return the column space of $\mathbf{A}$ (find its basis).


\paragraph{Determining the Kernel}

\emph{Kernel}: $\left\{ \mathbf{v}\in V|\Phi\left(\mathbf{v}\right)=\mathbf{0}_{W}\right\} $:
Set of vectors in $V$ that $\Phi$ maps onto the neutral element
in $W$. \emph{Note}: If kernel is $\left\{ \mathbf{0}\right\} $,
$\Phi$ is injective.

Return the solution to the LEQS $\mathbf{A}\mathbf{x}=\mathbf{0}$.


\paragraph{Rank-Nullity Theorem}

For $\Phi:V\rightarrow W$, $\dim\left(\mbox{Im}\left(\Phi\right)\right)=\dim\left(V\right)-\dim\left(\ker\left(\Phi\right)\right)$.


\subsection{Matrix Representation}

For the mapping $\Phi:V\rightarrow W$, and ordered bases $B\subseteq V$
and $C\subseteq W$, the \emph{transformation matrix} $\mathbf{A}_{\Phi}$
is defined such that for the coordinates $\mathbf{\hat{x}}$ of $\mathbf{x}\in V$
with respect to $B$ and $\mathbf{\hat{y}}$ of $\Phi\left(\mathbf{x}\right)\in W$
with respect to $C$: $\mathbf{\hat{y}}=\mathbf{A}_{\Phi}\mathbf{\hat{x}}$.


\subsection{Basis Change}

Given an $\mathbf{A}_{\Phi}$ with respect to bases $B$ and $C$,
we want an $\mathbf{\tilde{A}_{\Phi}}$ w.r.t. $\tilde{B}$ and $\tilde{C}$: 
\begin{enumerate}
\item Write the vectors of $\tilde{B}$ as a linear combination of the vectors
of $B$. These form the columns of the matrix $\mathbf{S}$. 
\item Write the vectors of $\tilde{C}$ as a linear combination of the vectors
of $C$. These form the columns of the matrix $\mathbf{T}$. 
\item $\mathbf{\tilde{A}_{\Phi}}$ can be calculated by $\mathbf{\tilde{A}_{\Phi}}=\mathbf{T}^{-1}\mathbf{A}_{\Phi}\mathbf{S}$. 
\end{enumerate}
This can be derived by considering the composition of the required
linear mappings.


\paragraph{Points to Note}
\begin{enumerate}
\item $\mathbf{A}$ and $\mathbf{\tilde{A}}$ are \emph{equivalent} if $\mathbf{\tilde{A}}$
can be expressed as $\mathbf{\tilde{A}}=\mathbf{T}^{-1}\mathbf{A}\mathbf{S}$. 
\item $\mathbf{A}$ and $\mathbf{\tilde{A}}$ are \emph{similar} if $\mathbf{\tilde{A}}$
can be expressed as $\mathbf{\tilde{A}}=\mathbf{S}^{-1}\mathbf{A}\mathbf{S}$. 
\end{enumerate}

\subsection{Eigenvalues}

For an endomorphism $\Phi:V\rightarrow V$, $\lambda$ is an \emph{eigenvalue}
if there exists an $\mathbf{x}\in V\backslash\left\{ \mathbf{0}\right\} $
s.t. $\Phi\left(\mathbf{x}\right)=\lambda\mathbf{x}$. $\mathbf{x}$
is the corresponding eigenvector.

For a transformation matrix \textbf{$\mathbf{A}$:}


\subsubsection{Determining the Spectrum (Eigenvalues)}
\begin{enumerate}
\item Calculate the determinant, $\left|\mathbf{A}-\lambda\mathbf{I}\right|$. 
\item Solve (equal to 0) the result (the \emph{characteristic polynomial})
for $\lambda$. 
\item The eigenvalues of $\mathbf{A}$ are given by the solutions. 
\end{enumerate}

\subsubsection{Determining the Corresponding Eigenspaces}
\begin{enumerate}
\item For each eigenvalue $\lambda$, find the solutions to the LEQS $\left(\mathbf{A}-\lambda\mathbf{I}\right)\mathbf{x}=\mathbf{0}$. 
\end{enumerate}

\subsubsection{Applications}
\begin{enumerate}
\item Used in principal component analysis (principle components have largest
eigenvalues) for dimensionality reduction in machine learning applications.
\item Used to determine the theoretical limit to how much information can
be transferred through a communication medium by calculating eigenvalues
and eigenvectors of communication channel.
\item Used in the PageRank algorithm to determine the rank of a page for
a search (based on maximal eigenvalue).
\item Determine numerical stability, e.g. when inverting matrices, by looking
at condition numbers (ratio of biggest to smallest eigenvalue).
\end{enumerate}

\subsection{Diagonalization}


\subsubsection{Determining Diagonalizability}
\begin{enumerate}
\item The characteristic polynomial must decompose into linear factors (and
the sign must be correct). 
\item The dimension of each eigenspace must be equal to the power (algebraic
multiplicity) of its respective factor in the characteristic polynomial. 
\end{enumerate}

\subsubsection{Diagonalization}
\begin{enumerate}
\item Determine the eigenspaces of the given matrix $\mathbf{A}$. 
\item Collect the eigenvectors in a single matrix $\mathbf{S}$. 
\item The diagonalization of $\mathbf{A}$ is given by $\mathbf{D}=\mathbf{S}^{-1}\mathbf{A}\mathbf{S}$.
This corresponds to a matrix with the eigenvalues of $\mathbf{A}$
along the diagonal.
\end{enumerate}

\subsubsection{Applications}
\begin{enumerate}
\item Diagonal matrices are easily raised to a power.
\item Decouple variables - uses in probability theory to interpret random
variables.
\item Easier to analyse properties of differential equations.
\end{enumerate}

\subsection{Projections}

A linear mapping $\pi$ is a projection if $\pi^{2}=\pi\circ\pi=\pi$.


\subsubsection{Projection onto a Line}

An arbitrary point $\mathbf{x}$ onto a line with basis vector $\mathbf{b}$
is projected onto the point $\mathbf{p}$: 
\begin{enumerate}
\item Find a $\lambda$ such that $\mathbf{p}=\lambda\mathbf{b}$ and $\mathbf{x}-\mathbf{p}\bot\mathbf{b}$.
Hence $\left(\mathbf{x}-\lambda\mathbf{b}\right)\cdot\mathbf{b}=0\iff\mathbf{x}\cdot\mathbf{b}-\lambda\mathbf{b}\cdot\mathbf{b}=0\iff\boxed{\lambda=\frac{\mathbf{x}\cdot\mathbf{b}}{\mathbf{b}\cdot\mathbf{b}}}$. 
\item Find the projection point, $\boxed{\mathbf{p}=\lambda\mathbf{b}}=\mathbf{b}\frac{\mathbf{x}\cdot\mathbf{b}}{\mathbf{b}\cdot\mathbf{b}}=\mathbf{b}\frac{\mathbf{b}^{\top}\mathbf{x}}{\mathbf{b}^{\top}\mathbf{b}}$. 
\item Conclude that the projection matrix $\boxed{\mathbf{P}_{\pi}=\frac{\mathbf{b}\mathbf{b}^{\top}}{\mathbf{b}^{\top}\mathbf{b}}}$. 
\end{enumerate}

\subsubsection{Projection onto a Subspace}

An arbitrary point $\mathbf{x}$ onto a subspace with ordered basis
$\mathbf{B}=\left(\mathbf{b}_{1}|\dots|\mathbf{b}_{n}\right)$ is
projected onto the point $\mathbf{p}$: 
\begin{enumerate}
\item Find a $\lambda$ such that $\mathbf{p}=\sum_{i=1}^{n}\lambda_{i}\mathbf{b}_{i}=\mathbf{B}\mathbf{\lambda}$
and $\mathbf{x}-\mathbf{p}\bot\mathbf{b}_{i}$. Hence $\left(\mathbf{x}-\mathbf{B}\mathbf{\lambda}\right)\cdot\mathbf{b}_{i}=0\iff\mathbf{B}^{\top}\left(\mathbf{x}-\mathbf{B}\mathbf{\mathbf{\lambda}}\right)=\boldsymbol{0}\iff\mathbf{B}^{\top}\mathbf{B}\mathbf{\mathbf{\lambda}}=\mathbf{B}^{\top}\mathbf{x}$.
Hence $\mathbf{\lambda}$ can be found by $\boxed{\mathbf{\mathbf{\lambda}}=\left(\mathbf{B}^{\top}\mathbf{B}\right)^{-1}\mathbf{B}^{\top}\mathbf{x}}$. 
\item Find the projection point, $\boxed{\mathbf{p}=\mathbf{B}\mathbf{\lambda}}=\mathbf{B}\left(\mathbf{B}^{\top}\mathbf{B}\right)^{-1}\mathbf{B}^{\top}\mathbf{x}$. 
\item Conclude that the projection matrix $\boxed{\mathbf{P}_{\pi}=\mathbf{B}\left(\mathbf{B}^{\top}\mathbf{B}\right)^{-1}\mathbf{B}^{\top}}$. 
\item To check your answer, ensure that:

\begin{enumerate}
\item \textbf{$\mathbf{p}-\mathbf{x}\bot\mathbf{b}$} for each $\mathbf{b}$
in $\mathbf{B}$. 
\item $\mathbf{P}_{\pi}=\mathbf{P}_{\pi}^{2}$. 
\end{enumerate}
\end{enumerate}

\subsubsection{Projection Error}

$d\left(\mathbf{x},U\right)$ for a point $\mathbf{x}$ projected
onto the basis $U$. 
\begin{enumerate}
\item Calculate the value $\left\Vert \mathbf{x}-\mathbf{p}\right\Vert $. 
\end{enumerate}

\subsubsection{Applications}
\begin{enumerate}
\item In graphics, e.g. to generate shadows.
\item Optimisation: orthogonal projections used to iteratively minimise
residual errors.
\item Project high dimensional data into a lower dimensional feature space,
e.g. for ML.
\end{enumerate}

\subsection{Rotations}


\subsubsection{In Two Dimensions}

\[
\mathbf{R}\left(\theta\right)=\left[\begin{array}{cc}
\cos\theta & -\sin\theta\\
\sin\theta & \cos\theta
\end{array}\right]
\]
(for a counter-clockwise rotation by $\theta$)


\subsubsection{In Three Dimensions}

We keep one axis constant and view this axis ``from the end towards
the origin''. A counter-clockwise rotation is then given by:

\[
\mathbf{R}_{1}\left(\theta\right)=\left[\begin{array}{ccc}
1 & 0 & 0\\
0 & \cos\theta & -\sin\theta\\
0 & \sin\theta & \cos\theta
\end{array}\right]
\]


\[
\mathbf{R}_{2}\left(\theta\right)=\left[\begin{array}{ccc}
\cos\theta & 0 & \sin\theta\\
1 & 1 & 0\\
-\sin\theta & 0 & \cos\theta
\end{array}\right]
\]
\[
\mathbf{R}_{3}\left(\theta\right)=\left[\begin{array}{ccc}
\cos\theta & -\sin\theta & 0\\
\sin\theta & \cos\theta & 0\\
0 & 0 & 1
\end{array}\right]
\]



\subsubsection{In $n$ Dimensions}

We keep all but two axes the same:

\[
\mathbf{R}_{ij}\left(\theta\right)=\left[\begin{array}{ccccc}
\mathbf{I}_{i-1} & \mathbf{0} & \mathbf{0} & \mathbf{0} & \mathbf{0}\\
\mathbf{0} & \cos\theta & \mathbf{0} & -\sin\theta & \mathbf{0}\\
\mathbf{0} & \mathbf{0} & \mathbf{I}_{j-i} & \mathbf{0} & \mathbf{0}\\
\mathbf{0} & \sin\theta & \mathbf{0} & \cos\theta & \mathbf{0}\\
\mathbf{0} & \mathbf{0} & \mathbf{0} & \mathbf{0} & \mathbf{I}_{n-j}
\end{array}\right]
\]



\subsubsection{Properties}
\begin{enumerate}
\item Composition of rotations is $\mathbf{R}\left(\phi\right)\mathbf{R}\left(\theta\right)=\mathbf{R}\left(\phi+\theta\right)$ 
\item Preserves lengths and distances, i.e. $\left\Vert \mathbf{x}\right\Vert =\left\Vert \mathbf{R}\left(\theta\right)\mathbf{x}\right\Vert $
and $\left\Vert \mathbf{x}-\mathbf{y}\right\Vert =\left\Vert \mathbf{R}\left(\theta\right)\mathbf{x}-\mathbf{R}\left(\theta\right)\mathbf{y}\right\Vert $ 
\item Not commutative (except in two dimensions) 
\end{enumerate}

\subsection{Cayley-Hamilton Theorem}

Useful for ``Find an expression for $\mathbf{A}^{m}$ in terms of
$\mathbf{I},\mathbf{A},\mathbf{A}^{2},\dots$'':

For an endomorphism with transformation matrix $\mathbf{A}_{\Phi}$
with characteristic polynomial $p$: 
\[
p\left(\mathbf{A}_{\Phi}\right)=\mathbf{0}
\]



\subsection{Affine Mappings}


\subsubsection{Definition}

An affine mapping is defined as $x\rightarrow a+\Phi\left(x\right)$
where $\Phi$ is a linear mapping.


\paragraph{Points to Note}
\begin{enumerate}
\item The composition of affine mappings is an affine mapping (same as for
linear mappings). 
\item Affine mappings preserve distances and parallelism. 
\end{enumerate}

\section{Scalar Products}


\subsection{Proving a Mapping is a Scalar Product}

For a mapping $\left\langle \mathbf{x},\mathbf{y}\right\rangle :V\times V\rightarrow\R$: 
\begin{enumerate}
\item Prove the mapping is linear in both arguments. 
\item Prove the mapping is symmetric: $\left\langle \mathbf{x},\mathbf{y}\right\rangle =\left\langle \mathbf{y},\mathbf{x}\right\rangle $
for all $\mathbf{x},\mathbf{y}\in V$. 
\item Prove the mapping is positive definite: $\left\langle \mathbf{x},\mathbf{x}\right\rangle >0$
except for $\left\langle \mathbf{0},\mathbf{0}\right\rangle =0$. 
\end{enumerate}
The standard scalar product is $\left\langle \mathbf{x},\mathbf{y}\right\rangle =\mathbf{x}^{\top}\mathbf{y}$.


\subsubsection{Applications}
\begin{enumerate}
\item Compute angles between vectors or distances, determine whether orthogonal.
\item Allows us to determine specific bases where each vector is orthogonal
to others - important for optimistations of numerical algortihms for
solving LEQs.
\item Kernel methods in machine learning. Allows for non-linearisation of
many ML algorithms.
\end{enumerate}

\subsection{Useful Geometric Properties}


\subsubsection{Finding the Length of a Vector}

\[
\left\Vert \mathbf{x}\right\Vert =\sqrt{\left\langle \mathbf{x},\mathbf{x}\right\rangle }
\]


A \emph{norm} has the properties: 
\begin{enumerate}
\item $\left\Vert \mathbf{x}\right\Vert >0$ except for $\left\Vert \mathbf{0}\right\Vert =0$ 
\item $\left\Vert \lambda\mathbf{x}\right\Vert =\left|\lambda\right|\cdot\left\Vert \mathbf{x}\right\Vert $ 
\item $\left\Vert \mathbf{x}+\mathbf{y}\right\Vert \leq\left\Vert \mathbf{x}\right\Vert +\left\Vert \mathbf{y}\right\Vert $ 
\end{enumerate}

\subsubsection{Finding the Distance between two Vectors}

\[
d\left(\mathbf{x},\mathbf{y}\right)=\left\Vert \mathbf{x}-\mathbf{y}\right\Vert 
\]


A \emph{metric} has the properties: 
\begin{enumerate}
\item Symmetric 
\item Positive Definite 
\item Obeys triangle inequality: $d\left(\mathbf{x},\mathbf{z}\right)\leq d\left(\mathbf{x},\mathbf{y}\right)+d\left(\mathbf{y},\mathbf{z}\right)$ 
\end{enumerate}

\subsubsection{Showing Orthogonality}

\[
\mathbf{x}\bot\mathbf{y}\iff\left\langle \mathbf{x},\mathbf{y}\right\rangle =0
\]



\subsubsection{Finding the Angle between Vectors}

\[
\cos\omega=\frac{\left\langle \mathbf{x},\mathbf{y}\right\rangle }{\left\Vert \mathbf{x}\right\Vert \left\Vert \mathbf{y}\right\Vert }
\]



\subsubsection{Properties of Euclidean Vector Spaces}
\begin{enumerate}
\item Cauchy-Schwarz Inequality: $\left\langle \mathbf{x},\mathbf{y}\right\rangle \leq\left\Vert \mathbf{x}\right\Vert \left\Vert \mathbf{y}\right\Vert $ 
\item Minkowski Inequality: $\left\Vert \mathbf{x}+\mathbf{y}\right\Vert \leq\left\Vert \mathbf{x}\right\Vert +\left\Vert \mathbf{y}\right\Vert $ 
\item Triangle Inequality: $d\left(\mathbf{x},\mathbf{z}\right)\leq d\left(\mathbf{x},\mathbf{y}\right)+d\left(\mathbf{y},\mathbf{z}\right)$ 
\item Parallelogram Law: $\left\Vert \mathbf{x}+\mathbf{y}\right\Vert +\left\Vert \mathbf{x}-\mathbf{y}\right\Vert =2\left\Vert \mathbf{x}\right\Vert ^{2}+2\left\Vert \mathbf{y}\right\Vert ^{2}$ 
\item $4\left\langle \mathbf{x},\mathbf{y}\right\rangle =\left\Vert \mathbf{x}+\mathbf{y}\right\Vert ^{2}-\left\Vert \mathbf{x}-\mathbf{y}\right\Vert ^{2}$ 
\item $\mathbf{x}\bot\mathbf{y}\iff\left\Vert \mathbf{x}+\mathbf{y}\right\Vert ^{2}=\left\Vert \mathbf{x}\right\Vert ^{2}+\left\Vert \mathbf{y}\right\Vert ^{2}$ \end{enumerate}

\end{document}
