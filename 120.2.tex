\documentclass[twocolumn,english]{article}
\usepackage[latin9]{inputenc}
\usepackage[landscape]{geometry}
\geometry{verbose,tmargin=0.5in,bmargin=0.75in,lmargin=0.5in,rmargin=0.5in}
\usepackage{float}
\usepackage{booktabs}

\makeatletter

\providecommand{\tabularnewline}{\\}

\setlength{\columnsep}{0.25in}
\usepackage{xcolor}
\usepackage{textcomp}
\usepackage{listings}
\lstset{
  language=java,
  tabsize=2,
  basicstyle=\small\ttfamily,
}
\let\oldenumerate=\enumerate
\def\enumerate{
\oldenumerate
\setlength{\itemsep}{3.5pt}
}
\let\olditemize=\itemize
\def\itemize{
\olditemize
\setlength{\itemsep}{0pt}
}

\makeatother

\usepackage{babel}
\usepackage{listings}
\renewcommand{\lstlistingname}{Listing}

\begin{document}

\title{Reference Sheet for CO120.2 Programming II}


\date{Spring 2017}

\maketitle

\section{Java Language Features}


\subsection{Control Flow}
\begin{itemize}
\item \texttt{if} \texttt{else}, \texttt{switch} statements. Don't forget
to \texttt{break} in a \texttt{switch} statement.
\item \texttt{while}, \texttt{do} \texttt{while}, \texttt{for}, \texttt{for}
\texttt{each} loops.
\item You can label loops, and use \texttt{break} and \texttt{continue}.
\end{itemize}

\subsection{Input and Ouput}


\paragraph{Processing of Strings}
\begin{itemize}
\item \texttt{Integer.parseInt(String s)} converts \texttt{s} to an \texttt{int}.
\item \texttt{.toString()} converts object to \texttt{String}.
\item \texttt{.charAt(int i)} returns character at index \texttt{i}.
\item \texttt{.substring(int i, int j)} returns substring between indices
\texttt{i} and \texttt{j}.
\item \texttt{.split(String s)} returns array which splits string around
matches of \texttt{s}.
\end{itemize}

\paragraph{Reading from Input}
\begin{itemize}
\item \texttt{BufferedReader br = new BufferedReader(new InputStreamReader(System.in))}
creates buffered reader.
\item \texttt{String line = br.readLine()} reads line of input.
\end{itemize}

\paragraph{Writing to Output}
\begin{itemize}
\item \texttt{PrintStream ps = new PrintStream(System.out)} creates print
stream.
\item \texttt{.println(String s)} prints new line of output.
\end{itemize}

\subsection{Enums}

E.g. \texttt{public enum Day \{MON, TUE, WED, THU, FRI, SAT, SUN;\}}.
\begin{itemize}
\item Can add fields and constructor to give properties.
\item Can add methods within \texttt{enum}.
\end{itemize}

\subsection{Arrays}

E.g. \texttt{char{[}{]} abc = \{'a', 'b', 'c'\}}. Cannot be extended
once defined. Can define values at given indices. Useful functionality
in \texttt{util.Arrays}:
\begin{itemize}
\item \texttt{copyOfRange(T{[}{]} a, int i, int j)} copies array between
indices \texttt{i} and \texttt{j}.
\item \texttt{sort(T{[}{]} a, Comparator\textless{}T\textgreater{} c)} sorts
array \texttt{a}.
\item \texttt{asList(T{[}{]} a)} converts \texttt{a} into list.
\item \texttt{toString(Object{[}{]} a)} converts array \texttt{a} to \texttt{String}.
\end{itemize}

\subsection{Collections}


\paragraph{Lists}
\begin{itemize}
\item \texttt{List\textless{}E\textgreater{} list = new ArrayList\textless{}\textgreater{}()}
or \texttt{new LinkedList\textless{}\textgreater{}()} creates new
list.
\item \texttt{.add(E e)} adds given \texttt{e} to list.
\item \texttt{.contains(E e)} returns whether \texttt{e} is in list
\item \texttt{.get(int i)} returns element at index \texttt{i}.
\item \texttt{.isEmpty()} returns whether list is empty.
\item \texttt{.remove(int i)} removes element at index \texttt{i}.
\item \texttt{.set(int i, E e)} sets element at index \texttt{i} to element
\texttt{e}.
\item \texttt{.size()} returns number of items in list.
\item \texttt{.stream()} creates an ordered stream from list elements.
\end{itemize}

\paragraph{Sets}
\begin{itemize}
\item \texttt{Set\textless{}E\textgreater{} set = new HashSet\textless{}\textgreater{}()}
creates new set.
\item \texttt{.add(E e)} adds given \texttt{e} to set.
\item \texttt{.contains(E e)} returns whether \texttt{e} is in set
\item \texttt{.isEmpty()} returns whether set is empty.
\item \texttt{.size()} returns number of items in set.
\item \texttt{.stream()} creates an unordered stream from set elements.
\end{itemize}

\paragraph{Maps}
\begin{itemize}
\item \texttt{Map\textless{}K,V\textgreater{} set = new HashMap\textless{}\textgreater{}()}
creates new map.
\item \texttt{.put(K key, V val)} maps \texttt{key} to \texttt{value}.
\item \texttt{.get(Object key)} returns value associated with \texttt{key},
or null if \texttt{key} isn't present.
\item \texttt{.containsKey(Object key)} returns whether \texttt{key} is
present.
\item \texttt{.keySet()} returns set of all keys in map.
\item \texttt{.values()} returns all vals in map.
\item \texttt{.isEmpty()} returns whether map is empty.
\item \texttt{.size()} returns number of items in map.
\end{itemize}

\paragraph{Queues}
\begin{itemize}
\item \texttt{Queue\textless{}E\textgreater{} queue = new PriorityQueue\textless{}\textgreater{}()}
creates new priority queue.
\item \texttt{.add(E e)} adds e to front of queue.
\item \texttt{.remove()} removes last element from queue.
\item \texttt{.poll()} retrieves \emph{and removes} head of queue, returns
\texttt{null} if empty.
\item \texttt{.peek()} retrieves head of queue, returns \texttt{null} if
empty.
\end{itemize}

\paragraph{Deques}
\begin{itemize}
\item \texttt{Deque\textless{}E\textgreater{} deque = new ArrayDeque\textless{}\textgreater{}()}
creates new double-ended queue.
\item Can use methods above at first or last element. E.g. \texttt{pollFirst()}
and \texttt{pollLast()}.
\item Can use stack methods \texttt{push(E e)}, \texttt{pop()} and \texttt{peek()}.
\end{itemize}

\subsection{Iterators}

Only use if a \texttt{for} loop is not straightforward.
\begin{itemize}
\item \texttt{.iterator()} converts collection to Iterator.
\item \texttt{.hasNext()} returns if there is another element.
\item \texttt{.next()} returns the next element.
\item \texttt{.remove()} removes the next element.
\end{itemize}

\subsection{Streams}
\begin{itemize}
\item \texttt{.stream()} converts \texttt{List} or \texttt{Set} into \texttt{Stream}.
\item \texttt{.collect(Collectors.toList())} and \texttt{.collect(Collectors.toSet())}
convert \texttt{Stream} into \texttt{List} and \texttt{Set} respectively.
\item \texttt{.map(Class::f)} maps \texttt{f} (a method or constructor)
to a stream.
\item \texttt{.filter(Class::f)} filters out elements for which \texttt{f}
applied to the element returns \texttt{false}.
\item \texttt{.reduce(id, Class::f)} reduces the stream starting from \texttt{id},
using the function \texttt{f}. Note that \texttt{f} has the type \texttt{T
f(T first, T second)}.
\item \texttt{.toMap(f, g)} creates map where keys and vals comes from \texttt{f}
and \texttt{g} applied to each stream element respectively.
\item You can perform a \emph{pipeline} of operations on a stream.
\end{itemize}

\paragraph{Lambda}

We can pass anonymous functions of the form \texttt{x -\textgreater{}
x {*} 2} or \texttt{(x, y) -\textgreater{} x {*} y}.


\paragraph{Optionals}

We can also \texttt{reduce} without an identity element, returning
an \texttt{Optional\textless{}T\textgreater{}}:
\begin{itemize}
\item \texttt{.isPresent()} returns whether object is present.
\item \texttt{.get()} returns value if present, else throws exception.
\item \texttt{.orElse(T alternative)} returns value if present, else \texttt{alternative}.
\item \texttt{.reduce(f)}. Now if steam is empty, an empty \texttt{Optional\textless{}T\textgreater{}}
is returned.
\end{itemize}

\subsection{Random}
\begin{itemize}
\item \texttt{Random generator = new Random()} creates new \texttt{Random}
object.
\item \texttt{.nextInt(n)} returns random \texttt{int} in range $[0..\mathtt{n})$.
\end{itemize}

\section{Object-Oriented Programming in Java}


\subsection{Types}

\begin{table}[H]
\centering{}%
\begin{tabular}{lcc}
\toprule 
 & \textbf{Primitive Types} & \textbf{Reference Types}\tabularnewline
\midrule 
\textbf{Definition} & built-in & class definition\tabularnewline
\textbf{Creation} & literals & \texttt{new}\tabularnewline
\textbf{Initialisation} & default (e.g. \texttt{0}) & \texttt{null} or constructor\tabularnewline
\textbf{Usage} & operators (e.g. \texttt{+}, \texttt{{*}}) & methods\tabularnewline
\textbf{Content} & value & pointer to an object\tabularnewline
\bottomrule
\end{tabular}
\end{table}


Should try to make objects immutable as much as possible - using the
keyword \texttt{final}.


\subsection{Classes}

Contain fields, constructor, methods.


\paragraph{Fields}
\begin{enumerate}
\item \emph{Constants}: should be \texttt{final} (cannot change) and \texttt{static}
(one per class, not one per object). Typically \texttt{public} (accessible
from anywhere).
\item \emph{Non-constants}: should be \texttt{private} (accessible only
from within class).
\end{enumerate}

\paragraph{Methods}

\texttt{public} methods should provide service to class users. \texttt{private}
methods support methods in the class.


\subsection{Interfaces}
\begin{enumerate}
\item Can define method \emph{signatures} (implicity public): describe required
capabilities of a class that \texttt{implements}\emph{ }the interface.
\item Can define \texttt{default} methods in interface (can be overriden
in implementing classes).
\item Can define \emph{constant} fields.
\end{enumerate}

\paragraph{Implementing Interfaces}

Use notation: \texttt{@Override} when a class method implements an
interface method.


\paragraph{Apparent and Actual Types}

Consider \texttt{Shape circle = new Circle()}. Has apparent type \texttt{Shape}
but actual type \texttt{Circle}. Only methods and fields from apparent
type (\texttt{Shape}) are available, but methods are implimented by
actual type (\texttt{Circle}).


\subsection{Inheritance}
\begin{enumerate}
\item A subclass (\texttt{extends} a superclass) inherits all fields and
methods from it's superclass, can also override / add additional functionality.
\item \texttt{protected} visibility in superclass allows access to its fields
/ methods from a subclass (also accessible anywhere within package).
\end{enumerate}

\paragraph{\texttt{super}}
\begin{enumerate}
\item Can call superclass constructur using \texttt{super()} (called implicitly
if no other constructor provided).
\item Can call superclass method \texttt{method} from subclass using \texttt{super.method()}.
\end{enumerate}

\paragraph{\texttt{abstract}}
\begin{enumerate}
\item Used to define class which cannot be instantiated.
\item \texttt{abstract} methods have no body, needs to be overriden by subclass.
\item \emph{Style}: usually a class \texttt{extends} an abstract class (with
constructor and fields) which \texttt{implements} an interface (caters
for case where abstract class is too specific).
\end{enumerate}

\paragraph{\texttt{final}}
\begin{enumerate}
\item Methods which are \texttt{final} cannot be overriden.
\item Classes which are \texttt{final} cannot be extended.
\end{enumerate}

\subsection{Casting}
\begin{enumerate}
\item \emph{Upcasting}: cast from subclass to superclass. Done automatically,
cannot fail.
\item \emph{Downcasting}: cast from superclass to subclass. E.g. for \texttt{Shape
shape}, define \texttt{Circle circle = (Circle)shape}. Narrow apparent
class so you can call certain methods / access certain fields.
\item Downcasting can lead to a \texttt{ClassCastException}. We can avoid
this by using the \texttt{instanceOf} keyword to determine the actual
type.
\item \emph{Style}: \texttt{instanceOf} can indicate poor design. Often
better to use subclass methods.
\end{enumerate}

\subsection{Object Equality}

\texttt{Object} implements \texttt{equals} based on identity. Often
we want it to compare field contents:
\begin{enumerate}
\item We \texttt{@Override} the method \texttt{public boolean equals(Object
other)}.
\item Start by handling standard object equality cases (\texttt{==} and
\texttt{null}).
\item Check incoming object appropriate type, then downcast.
\item Compare fields.
\end{enumerate}
We \emph{must} also override \texttt{public int hashCode()}.
\begin{enumerate}
\item Must return same value when \texttt{equals} returns true.
\item Should tend to return different values for objects which are not equal.
\end{enumerate}

\subsection{Generics}
\begin{enumerate}
\item \emph{Classes}: we can define a class \texttt{ClassName\textless{}A,
B\textgreater{}} where \texttt{A} and \texttt{B} are \emph{type parameters}.
We can then use \texttt{A} and \texttt{B} as normal type names within
that class.
\item \emph{Methods}: example: \texttt{public static Pair\textless{}S, T\textgreater{} makePair(s First, T second)}. Then
use \texttt{S} and \texttt{T} freely inside method.
\end{enumerate}

\paragraph{Wildcards}
\begin{enumerate}
\item \texttt{Set\textless{}Shape\textgreater{}} refers to a set of shapes.
We can add any shape to this set, and can retrieve shapes. \texttt{Set\textless{}Circle\textgreater{}}
however is \emph{not} a subtype of \texttt{Set\textless{}Shape\textgreater{}}.
\item \texttt{Set\textless{}? extends Shape\textgreater{}} refers to any
set whose elements are a subclass of \texttt{Shape}. We cannot add
to this set, but we can retrieve shapes. A \texttt{Set\textless{}Circle\textgreater{}}
\emph{is} a subtype of \texttt{Set\textless{}? extends Shape\textgreater{}}.
\item \emph{Note}: We could instead use \texttt{Set\textless{}T extends
Shape\textgreater{}} if we care about the type of \texttt{?}.
\end{enumerate}

\paragraph{Inheritance}
\begin{enumerate}
\item \emph{Extending a generic class / Implementing a generic interface}:
E.g. \texttt{public abstract class AbstractSet\textless{}E\textgreater{}
implements Set\textless{}E\textgreater{}} and \texttt{public class
HashSet\textless{}E\textgreater{} extends AbstractSet\textless{}E\textgreater{}}.
\item \emph{Extending / Implementing a class to be specific}: E.g. \texttt{public
interface Comparable\textless{}T\textgreater{}} and \texttt{public
class ClassName implements Comparable\textless{}ClassName\textgreater{}}.
\end{enumerate}

\subsection{Functional Interfaces}
\begin{enumerate}
\item Annotated by \texttt{@FunctionalInterface}.
\item Declares exactly one (abstract) method. E.g. \texttt{public interface
Comparator\textless{}T\textgreater{} \{int compare(T o1, T o2);\}.}
\item Sort list using void \texttt{.sort(Comparator\textless{}E\textgreater{}
c)}.
\item Means we can write \texttt{strings.sort((a, b) -\textgreater{} a.compareTo(b))}
since this provides the single method required by \texttt{Comparator\textless{}String\textgreater{}}.
\end{enumerate}

\subsection{Singleton Pattern}

Only allows one instance to be created:

\begin{lstlisting}[language=Java,basicstyle={\footnotesize\ttfamily},tabsize=2,frame=single]
public class OnlyOne {
	private static OnlyOne instance;
	private onlyOne () {}
	public static OnlyOne getInstance () {
		if (instance == null) {
			instance = new OnlyOne(); 
		}
		return instance;
	}
}
\end{lstlisting}



\subsection{Cloning}

If you really think it's necessary:
\begin{enumerate}
\item Implement \texttt{Clonable}.
\item Override \texttt{clone}.
\item Increase visibility to \texttt{public}.
\item Restrict return type. Call \texttt{Object}'s clone to create bitwise
copy, using \texttt{(myClass)super.clone()}.
\item Deep-clone fields if appropriate.
\item Return the clone.
\end{enumerate}

\subsection{Exceptions}

Exceptions can be thrown by called methods. Can either be caught in
\texttt{try}, \texttt{catch}, \texttt{finally} block or propagated.


\section{Abstract Data Types in Java}


\subsection{Linear Data Structures}


\paragraph{Lists}
\begin{enumerate}
\item Arbitrary number of elements ordered by position.
\item Need to be able to create, check if empty, obtain size, get, add and
remove.
\item \emph{Array-based}: use dynamic expansion (use copy of old array).
\item \emph{Linked}: use nodes that store current elem and reference to
next node, keep a reference to the head node. Note that you need seperate
cases for dealing with the first element!
\item \emph{Ordered}: Use \texttt{compareTo} from \texttt{Comparable} interface.
Add using recursion.
\item \emph{Iterators}: Usually implemented as inner classes. Need to be
careful of concurrent modifications.
\end{enumerate}

\paragraph{Stacks}
\begin{enumerate}
\item Linear sequence of items with insertions and deletions only allowed
at top. Implements LIFO access.
\item Example applications: frame stack for recursive functions, reverting
text, validating parentheses.
\item Need to be able to create, check if empty, push, pop and peek.
\item \emph{Array-based}: Use array of specific length, keep the index of
top element.
\item \emph{List Based}: Use a linked list. Head of list is the top.
\end{enumerate}

\paragraph{Queues}
\begin{enumerate}
\item Allows insertion only at back, deletion only at front. Implements
FIFO access. 
\item Example applications: scheduling, demerging.
\item Need to be able to create, check if empty, enqueue and dequeue.
\item \emph{Array-based}: Use circular array (using \texttt{\%}) of specific
max length. Keep indexes for first and last elements.
\item \emph{List-based}: Use linked list. Keep references to first and last
elements.
\item \emph{Priority}: Extend node class to include priority. Keep an ordered
linked list whose elements are orderd according to priority value.
Or use a min-heap.
\end{enumerate}

\paragraph{Maps}
\begin{enumerate}
\item Collection of items described by (key, value) pairs. No duplicate
keys.
\item Example applications: caches, finding duplicates, random access to
large data sets.
\item Need to be able to create, check if empty, obtain size, check if contains
element, put, get and remove elements.
\item \emph{Array-based}: use an array to store values, index computed by
hash function applied to key.
\item Hash function should: minimise collisions, be fast to compute, distriibute
elements uniformly through the array, be deterministic.
\item Example hash functions: selecting digits, adding digits together,
modulo arithmetic.
\end{enumerate}

\paragraph{Sets}
\begin{enumerate}
\item Models mathematical set.
\item Need to be able to create, check if empty, obtain size, add and remove.
\item Can use hash map to implement.
\end{enumerate}

\subsection{Tree-Based Data Structures}


\paragraph{Binary Trees}
\begin{enumerate}
\item Consist of data element (root) and two disjoint binary trees.
\item Need to be able to create, check if empty, obtain size, set and get
left and right subtrees.
\item Example applications: syntax trees, Huffman coding trees.
\item \emph{Height}: the number of \emph{levels} in a tree\emph{.}
\item \emph{Perfectly balanced}: height equal to length of shortest path.
A perfectly balanced tree has $2^{h}-1$ nodes.
\item \emph{Complete}: full down to height $h-1$ and level $h$ filled
from left to right.
\item \emph{Linked}: contain current element and references to left and
right subtrees.
\item \emph{Array-based}: root at index 0, children at $2i+1$ and $2i+2$,
parent at $(i-1)/2$.
\end{enumerate}

\paragraph{General Trees}

Nodes can have any number of children. Linked implementation possible
with lists.


\paragraph{Traversal}
\begin{enumerate}
\item \emph{Depth First}: Recursive (or using stack).\emph{ Pre-order}:
root, left, right. \emph{In-order}: left, root, right. \emph{Post-order}:
left, right, root.
\item \emph{Breadth First}: Level by level. Use a queue.
\end{enumerate}

\paragraph{Binary Search Trees}
\begin{enumerate}
\item All values on left subtree smaller than in root, on right are larger.
\item Provide a \texttt{Comparator} or implement \texttt{Comparable} to
provide search key.
\item Search, insert and size all defined recursively.
\item Remove has several cases:

\begin{enumerate}
\item If leaf, just delete
\item If has one child (L or R), use that child.
\item If has two children, create replacement node with value of one, make
the other the correct child of replacement node.
\end{enumerate}
\end{enumerate}

\paragraph{Self-Balancing Trees}
\begin{enumerate}
\item \emph{Balanced}: Nodes at level $\leq h-2$ have 2 children.
\item \emph{Rotations}:

\begin{enumerate}
\item Reparent child on opposite side.
\item Set child on rotation side to the old parent.
\end{enumerate}
\item \emph{Double Rotations }(e.g. Left-Right rotation):

\begin{enumerate}
\item Left side (left rotation).
\item Right rotation.
\end{enumerate}
\item \emph{AVL Tree}: Store height values. After each insertion, check
node and rebalance if necessary:

\begin{enumerate}
\item Left bigger than Right and Left side (left bigger than right): left-right
rotation.
\item Left bigger than Right: right rotation.
\item Right bigger than Left and Right side (right bigger than left): right-left
rotation.
\item Right bigger than Left: left rotation.
\end{enumerate}
\item \emph{Red-Black Tree}: Store a color value (R or B). Root is black.
Number of black nodes on every path from the root is the same. No
two consecutive nodes are red. Start by setting the node red. Then
five cases to handle:

\begin{enumerate}
\item \emph{Root node}: If no parent, colour current black, finish.
\item \emph{Black parent}: If parent is black, finish.
\item \emph{Red uncle}: If uncle exists and is red, colour parent and uncle
black and grandparent red. Begin case (a) on grandparent.
\item \emph{Fix zig-zags}: 

\begin{enumerate}
\item If parent left of grandparent and current right of parent: rotate
parent left. Begin case (e) on parent. 
\item If parent right of grandparent and current left of parent: rotate
parent right. Begin case (e) on parent.
\end{enumerate}
\item \emph{Fix colouring and rotate}:

\begin{enumerate}
\item If is left child, colour parent black and grandparent red. Rotate
grandparent right, finish.
\item If is right child, colour parent black and grandparent red. Rotate
grandparent left, finish.
\end{enumerate}
\end{enumerate}
\end{enumerate}

\paragraph{Heaps}
\begin{enumerate}
\item Complete binary tree with weak ordering.
\item Need to be able to create, check if empty, obtain size, add elements,
peek and poll max.
\item \emph{Max Heap}: root contains largest element. Each subtree is a
max heap.
\item \emph{Array-based} is efficient implementation (see bin trees).
\item \emph{Retrieving max element}: leaves two subheaps to be merged:

\begin{enumerate}
\item Replace root with last element in array to create semi-heap.
\item \emph{Sift} by swapping root with largest child recursively.
\end{enumerate}
\item \emph{Adding element}: add element to end of array and sift upwards.
\item \emph{Heap sort}: convert array to heap first, then sort array heap.
\end{enumerate}

\section{Concurrency in Java}


\subsection{Threads}
\begin{enumerate}
\item Specify a thread.

\begin{enumerate}
\item Extend \texttt{Thread} and override \texttt{run()}.
\item Implement \texttt{Runnable} and override \texttt{run()}.
\item \texttt{Thread t = new Thread(() -\textgreater{} \{code goes here\})}.
\item \texttt{Runnable r = () -\textgreater{} \{code goes here\}; Thread
t = new Thread(r)}.
\end{enumerate}
\item Start the thread using \texttt{t.start()}.
\item Wait for termination using \texttt{t.join()}.
\item Sleep \texttt{x} seconds using \texttt{TimeUnit.SECONDS.sleep(x)}
(goes inside \texttt{Thread}).
\end{enumerate}

\subsection{Avoiding Race Conditions}
\begin{enumerate}
\item Use the \texttt{synchronized} keyword (do this for all methods that
can read or write shared fields - note \texttt{final} fields do not
need it!).
\item Use locks:

\begin{enumerate}
\item Always use a \texttt{try-finally} block to ensure lock is released!
\item \texttt{ReentrantLock}

\begin{enumerate}
\item Use \texttt{Lock lock = new ReentrantLock(true)} to set a fair lock
\item \texttt{lock.lock()} and \texttt{lock.unlock()} to lock and unlock.
\end{enumerate}
\item \texttt{ReadWriteLock}

\begin{enumerate}
\item \texttt{ReadWriteLock lock = new ReentrantReadWriteLock(true)} for
new lock.
\item \texttt{lock.readLock().lock()} and \texttt{lock.readLock().unlock()}
for read lock and unlock.
\item \texttt{write.readLock().lock()} and \texttt{write.readLock().unlock()}
for write lock and unlock.
\end{enumerate}
\item \texttt{StampedLock}

\begin{enumerate}
\item \texttt{StampedLock sl = new StampedLock()}.
\item \texttt{long stamp = sl.readLock()} and \texttt{sl.unlockRead(stamp)}.
\item \texttt{long stamp = sl.tryOptimisticRead()} followed by if \texttt{(!sl.validate(stamp))\{stamp
= sl.readLock()\}}.
\end{enumerate}
\end{enumerate}
\item Use semaphores:

\begin{enumerate}
\item Create new using \texttt{Sephamore(int permits, bool fair)}.
\item \texttt{aquire()} to take a permit, waiting if necessary.
\end{enumerate}
\item Use atomic classes. E.g. \texttt{new AtomicBoolean(x)}, \texttt{new
AtomicInteger(x)} and new \texttt{AtomicReference(x)}, use \texttt{.get()}
and \texttt{.set(x)}.
\item Use adders and accumulators. E.g. \texttt{new DoubleAccumulator((x,
y) -\textgreater{} x {*} y, 1.00)} followed by \texttt{.accumulate(x)}
and \texttt{.get()}.
\item Use the \texttt{volatile} keyword. Guarantees visibility of changes
(ignores cache). Can avoid locks and synchronisation if operations
are atomic.
\item Use the \texttt{final} keyword. Enforces visibility (as above).
\item Immutable objects are always thread safe. Objects are immutable if
all fields are final and immutable (if a field is not immutable, return
copies of it using \texttt{clone()}).
\item Use a \texttt{ConcurrentHashMap}. Also includes functionality such
as:

\begin{enumerate}
\item \texttt{.forEach((k, v) -\textgreater{} ...)}
\item \texttt{.putIfAbsent(k, v)}
\item \texttt{.getOrDefault(k, defaultV)}
\item \texttt{.replaceAll ((k, v) -\textgreater{} ...)}
\item \texttt{.search((k, v) -\textgreater{} \{...\})}
\item \texttt{.reduce((k, v) -\textgreater{} \{...\}, (p1, p2) -\textgreater{}
\{...\})}
\end{enumerate}
\item Use a \texttt{parallelStream}.
\item \emph{Data parallelism} approaches (e.g. for linear algebra and image
processing - can avoid locks entirely).
\end{enumerate}

\section{Tips for Lexis Tests}
\begin{itemize}
\item Read the questions and guidance very carefully.
\item Think carefully anywhere you have to change mutable objects. Don't
forget to store the old information in a temp variable if you need
it later.
\item Avoid repetition of code. Move it to a private helper method or abstract
class if you can.
\item Handle unexpected cases. Make use of the exceptions provided.
\item Be careful. Provided ADTs are often 1-indexed.\end{itemize}

\end{document}
