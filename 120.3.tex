\documentclass[twocolumn,english]{article}
\usepackage[T1]{fontenc}
\usepackage[latin9]{inputenc}
\usepackage[landscape]{geometry}
\geometry{verbose,tmargin=0.5in,bmargin=0.75in,lmargin=0.5in,rmargin=0.5in}

\makeatletter

\setlength{\columnsep}{0.25in}
\usepackage{xcolor}
\usepackage{textcomp}
\usepackage{listings}
\lstset{
  language=haskell,
  tabsize=2,
  basicstyle=\small\ttfamily,
}

\makeatother

\usepackage{babel}
\usepackage{listings}
\renewcommand{\lstlistingname}{Listing}

\begin{document}

\title{Reference Sheet for CO120.3 Programming III}


\date{Summer 2017}

\maketitle

\paragraph{Basic Types}\label{basic-types}

\begin{enumerate}
\def\labelenumi{\arabic{enumi}.}

\item
  Don't use the \texttt{char}, \texttt{short}, \texttt{int} and
  \texttt{long} types.
\end{enumerate}

In order to increase portability and readability, use \texttt{stdint.h}.

Included types are: \texttt{uint8\_t}, \texttt{uint16\_t},
\texttt{uint32\_t}, \texttt{uint64\_t}, \texttt{int8\_t},
\texttt{int16\_t}, \texttt{int32\_t}, \texttt{int64\_t}.

\paragraph{Enumerated Types}\label{enumerated-types}

Example:

\begin{verbatim}
/**
 * @brief An enum type that represents flags for rendering.
 *
 * Each bit represents a different flag. Use bitwise and 
 * to check if a flag is set.
 */
enum render_flag {
  /** The ambient flag (bit 0). */
  AMBIENT = 1,
  /** The diffuse flag (bit 1). */
  DIFFUSE = 2,
  /** The specular flag (bit 2). */
  SPECULAR = 4,
};
\end{verbatim}

\paragraph{Functions}\label{functions}

\begin{enumerate}
\def\labelenumi{\arabic{enumi}.}

\item
  Use underscore\_separation for function names.
\item
  Don't put a space between the function name and parameter list.
\item
  Never leave the parameter list for a function empty.
\item
  Leave a single space between the closing parenthesis of the parameter
  list and the opening brace of the function body.
\end{enumerate}

Note: Try to keep functions under 40 lines long.

Remember to include \texttt{void} in your parameter list for functions
without parameters.

\paragraph{Variables}\label{variables}

Note: In general, you should initialise variables at point of
definition.

Example:
\texttt{double\ numbers{[}2{]}{[}3{]}\ =\ \{\{1.0,\ 2.0,\ 3.0\},\ \{2.0,\ 3.0,\ 4.0\}\}}.

A string can be represented as an array of type \texttt{char}, ending
with \texttt{`/0'}.

Example: \texttt{uint8\_t\ hello{[}{]}\ =\ "Hello!"}.

\paragraph{Flow of Control}\label{flow-of-control}

\begin{enumerate}
\def\labelenumi{\arabic{enumi}.}

\item
  Always use braces where it is possible to do so.
\item
  Leave a space and after the opening and closing parentheses for the
  condition, respectively.
\end{enumerate}

Note: You should usually break at the end of each case in a switch
statement.

Examples:

\begin{verbatim}
if (condition) {
  statement;
  ...
} else if (condition) {
  ...
} else {
  ...
}
\end{verbatim}

\begin{verbatim}
while (condition) {
  ...
}
\end{verbatim}

\begin{verbatim}
for (initialiser; condition; increment / decrement) {
  ...
}
\end{verbatim}

\begin{verbatim}
switch (expression) {
  case constant1:
    ...
    break;
  case constant2:
    ...
    break;
  default:
    assert(false);
}
\end{verbatim}

\paragraph{Logical Expressions}\label{logical-expressions}

\begin{enumerate}
\def\labelenumi{\arabic{enumi}.}

\item
  Place constants before variables when using the equality operator.
\end{enumerate}

Note: Use parentheses to clarify precedence in complex expressions.

Example: \texttt{3\ ==\ x}, NOT \texttt{x\ ==\ 3}.

\paragraph{Input Output}\label{input-output}

Use \texttt{printf} to print to standard output. Sanitise using: -
\texttt{\%c} for character. - \texttt{\%d} for signed integer. -
\texttt{\%u} for unsigned integer. - \texttt{\%o} for octal. -
\texttt{\%x} for hexadecimal. - \texttt{\%f} for floating point value. -
\texttt{\%e} for floating point value using scientific notation. -
\texttt{\%s} for string. - \texttt{\%p} for pointer value.

Use \texttt{scanf} to read from standard input. Beware of secuirity
risks when inputting, say, a string. You can use \texttt{fgets} instead.

Examples:

\begin{verbatim}
int i;
int ret = scanf("%i", &i);
assert(1 == ret);
\end{verbatim}

\begin{verbatim}
char buffer[100];
int size = sizeof(buffer);
fgets(buffer, size, stdin);
\end{verbatim}

\paragraph{Bitwise Operations}\label{bitwise-operations}

\begin{itemize}

\item
  \texttt{\&} for bitwise AND.
\item
  \texttt{\textbar{}} for bitwise OR.
\item
  \texttt{\textgreater{}\textgreater{}} for RIGHT SHIFT.
\item
  \texttt{\textless{}\textless{}} for LEFT SHIFT.
\item
  \texttt{\textasciitilde{}} for bitwise NOT.
\item
  \texttt{\^{}} for bitwise XOR.
\end{itemize}

\paragraph{Pointers}\label{pointers}

\begin{enumerate}
\def\labelenumi{\arabic{enumi}.}

\item
  When declaring a pointer, place the \texttt{*} adjacent to the
  variable name, not the type.
\item
  When passing by reference, pointers must be declared as
  \texttt{const}.
\end{enumerate}

Example: \texttt{int\ *pointer}, NOT \texttt{int*\ pointer} or
\texttt{int\ *\ pointer}.

Note: Use the \texttt{const} keyword wherever possible.

Examples:

\begin{verbatim}
int val = 5;
// Value cannot be modified
const int *ptr1 = &val;
// Pointer cannot be modified
int const *ptr2 = &val;
// Neither value nor pointer can be modified
const int const *ptr3 = &val;
\end{verbatim}

Think of \texttt{*} as an operator that takes an address and returns the
value which it points to. \texttt{\&} is an operator that takes a value
and returns its address.

We can also use function pointers to pass functions be reference.

Example:

\begin{verbatim}
/**
 * @brief A function which sums its two inputs.
 */
int sum(int a, int b) {
  return a + b;
}

/**
 * @brief A main function which does nothing.
 */
int main(void) {
  int (*sum_pointer)(int, int) = &sum;

  return EXIT_SUCCESS;
}
\end{verbatim}

\paragraph{Command Line Arguments}\label{command-line-arguments}

\texttt{main} can have a type signature where it receives arguments from
the command line:

\begin{itemize}

\item
  \texttt{argc}, the number of passed parameters.
\item
  \texttt{argv}, an array of strings.
\end{itemize}

The first argument is the name of the file!

Example:

\begin{verbatim}
/**
 * @brief A main function which prints its command line arguments.
 */
int main(int argc, char **argv) {
  for (int i = 0; i < argc; i++) {
    printf("argv[%i] = %s\n", i, argv[i]);
  }
  return EXIT_SUCCESS;
}
\end{verbatim}

\paragraph{Memory Management}\label{memory-management}

\begin{enumerate}
\def\labelenumi{\arabic{enumi}.}

\item
  \texttt{main} should only return \texttt{EXIT\_SUCCESS} or
  \texttt{EXIT\_FAILURE}.
\end{enumerate}

\begin{itemize}

\item
  \texttt{stlib.h}'s \texttt{void\ *malloc(size\_t\ size)} allocates a
  region of memory of \texttt{size} bytes and returns a pointer to the
  allocated memory.
\item
  \texttt{void\ free(void\ *ptr)} frees a previously allocated memory
  region.
\item
  You need to check that these don't fail (not NULL).
\item
  \texttt{string.h}'s
  \texttt{void\ *memset\ (void\ *s,\ int\ c,\ size\_t\ n)} sets teh
  \texttt{n}-byte region starting at \texttt{s} to \texttt{c}.
\item
  \texttt{void\ *memcpy(void\ *dest,\ const\ void\ *src,\ size\_t\ n)}
  copies \texttt{n} bytes from \texttt{src} to \texttt{dst}, returning
  \texttt{dst}.
\item
  You can use the \texttt{memcheck} tool provided by \texttt{valgrind}
  to check for memory leaks.
\end{itemize}

\paragraph{Assertions}\label{assertions}

\texttt{assert.h} provides \texttt{assert(logical\_expression)}.

\paragraph{Headers}\label{headers}

\begin{enumerate}
\def\labelenumi{\arabic{enumi}.}

\item
  All headers should be surrounded by include guards,
  \texttt{\#ifndef\ THIS\_H}, \texttt{\#define\ THIS\_H}.
\end{enumerate}

Note: In general, every \texttt{.c} file should have an associated
\texttt{.h} file.

\paragraph{Makefile}\label{makefile}

Example:

\begin{verbatim}
# Set the flags for the C compiler
CFLAGS = -Wall -pedantic -std=c99

# Build rule for final executable
sum: add.o

# Build rules for the .o files
sum.o: add.h
add.o: add.h

# Rule to clean generated files
clean: rm -f sum sum.o add.o

# Tell make that 'clean' is not a real file
.PHONY: clean
\end{verbatim}

\paragraph{More Variables}\label{more-variables}

\begin{itemize}

\item
  \texttt{static} (local): value is retained.
\item
  \texttt{static} (top level): only seen within this file.
\item
  \texttt{extern} variables: defined in header, can be used in multiple
  files.
\item
  Global variables: try to avoid them.
\end{itemize}

\paragraph{File Input}\label{file-input}

\begin{itemize}

\item
  Use \texttt{FILE\ fopen(const\ char\ *path,\ const\ char\ *mode)}.
\item
  Modes are \texttt{r} to read from start, \texttt{w} to write from
  start, \texttt{a} to append.
\item
  Need to check file could be opened (not NULL).
\item
  Use \texttt{int\ feof(FILE\ *stream)} to check if end of file has been
  reached (NULL).
\item
  Use \texttt{int\ fclose(FILE\ *fp)} to close file.
\item
  Need to check it doesn't fail.
\end{itemize}

Reading a file:

\begin{itemize}

\item
  \texttt{int\ fscanf(FILE\ *stream,\ const\ char\ *format,\ ...)}.
\item
  \texttt{char\ *fgets(char\ *s,\ int\ size,\ FILE\ *stream)}.
\end{itemize}

Writing to a file:

\begin{itemize}

\item
  \texttt{int\ fprintf(FILE\ *stream,\ const\ char\ *format,\ ...)}.
\item
  \texttt{int\ fputs(const\ char\ *s,\ FILE\ *stream)}.
\end{itemize}

Binary data:

\begin{itemize}

\item
  Use the mode \texttt{b}.
\item
  \texttt{size\_t\ fread(void\ *ptr,\ size\_t\ size,\ size\_t\ nmemb,\ FILE\ *stream)}
\item
  \texttt{size\_t\ fwrite(const\ void\ *ptr,\ size\_t\ size,\ size\_t\ nmemb,\ FILE\ *stream)}
\item
  \texttt{ptr} is the address to read or write, \texttt{nmemb} is the
  number of elements, \texttt{size} is the number of bytes for each
  element.
\item
  Need to check returned value is the number of elements.
\end{itemize}

Error checking:

You can also use \texttt{int\ ferror(FILE\ *stream)}.

Printing errors:

\begin{itemize}

\item
  Use \texttt{perror(char\ *msg)} when printing non user defined errors.
\item
  Otherwise, use \texttt{fprintf(stderr,\ char\ *msg)}.
\end{itemize}

\paragraph{Structs}\label{structs}

Example:

\begin{verbatim}
/**
 * @brief A struct that hold the co-ordinates of a 2D vertex.
 */
typedef struct {
  /** The x co-ordinate */
  float x;
  /** The y co-ordinate */
  float y;
} vertex;
\end{verbatim}

\begin{itemize}

\item
  Use \texttt{.} to refer to a struct component.
\item
  Use \texttt{-\textgreater{}} to refer to a struct component from a
  pointer to a struct.
\end{itemize}

Use \texttt{typedef} to declare an alias for an existing type.

Use a \texttt{union} where requirements to store components are mutually
exclusive.

\paragraph{Constants}\label{constants}

Three approaches:

\begin{enumerate}
\def\labelenumi{\arabic{enumi}.}

\item
  Use a marco using \texttt{\#define}.
\item
  Define a global \texttt{static\ const} variable (not a true constant).
\item
  Use an unnamed \texttt{enum}.
\end{enumerate}

\paragraph{Strings}\label{strings}

Use \texttt{string.h} to get:

\begin{itemize}

\item
  \texttt{size\_t\ strlen(const\ char\ *s)}.
\item
  \texttt{char\ *strncat(char\ *dest,\ const\ char\ *src,\ size\_t\ n)}.
\item
  \texttt{int\ strncmp(const\ char\ *s1,\ const\ char\ *s2,\ size\_t\ n)}.
\item
  \texttt{char\ *strncpy(char\ *dest,\ const\ char\ *src,\ size\_t\ n)}.
\end{itemize}

\paragraph{Tips for Lexis Tests}\label{tips-for-lexis-tests}

Make sure you check for all errors and initialise variables immediately. In particular:
\begin{itemize}
\item All functions with a pointer as an argument should check that the given pointer is not NULL.
\item Any memory that you malloc must be checked not NULL and should be initialised immediately to 0.
\end{itemize}

\end{document}
