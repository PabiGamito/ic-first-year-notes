\documentclass[twocolumn,english]{article}
\usepackage[latin9]{inputenc}
\usepackage[landscape]{geometry}
\geometry{verbose,tmargin=0.5in,bmargin=0.75in,lmargin=0.5in,rmargin=0.5in}
\setlength{\parskip}{\medskipamount}
\setlength{\parindent}{0pt}
\usepackage{amsbsy}
\usepackage{amssymb}
\usepackage{esint}

\makeatletter
\setlength{\columnsep}{0.25in}
\usepackage{xcolor}
\usepackage{textcomp}
\usepackage{listings}
\lstset{
  language=python,
  tabsize=2,
  frame=single,
  basicstyle=\small\ttfamily,
}

\makeatother

\usepackage{babel}
\usepackage{listings}
\renewcommand{\lstlistingname}{Listing}

\begin{document}

\title{Reference Sheet for CO142.2 Discrete Mathematics II}


\date{Spring 2017}

\maketitle

\section{Graphs}


\paragraph{Defintions}
\begin{enumerate}
\item \emph{Graph}: set of $N$ nodes and $A$ arcs such that each $a\in A$
is associated with an unordered pair of nodes.
\item \emph{Simple graph}: no parallel arcs and no loops.
\item \emph{Degree}: number of arcs incident on given node.

\begin{enumerate}
\item Sum of degrees of all nodes in a graph is even.
\item Number of nodes with odd degree is even.
\end{enumerate}
\item \emph{Subgraph}: $G_{S}$ is a subgraph of $G$ iff $\mbox{nodes}\left(G_{S}\right)\subseteq\mbox{nodes}\left(G\right)$
and $\mbox{arcs}\left(G_{S}\right)\subseteq\mbox{arcs}\left(G\right)$.
\item \emph{Full (induced) subgraph}: contains all arcs of $G$ which join
nodes present in $G_{S}$.
\item \emph{Spanning subgraph}: $\mbox{nodes}\left(G_{S}\right)=\mbox{nodes}\left(G\right)$.
\end{enumerate}

\paragraph{Representation}
\begin{enumerate}
\item \emph{Adjacency matrix}: symmetric matrix with number of arcs between
nodes, count each loop twice.
\item \emph{Adjacency list}: array of linked lists, if multiple arcs then
multiple entries, count loops once.
\item An adjacency matrix has size $n^{2}$, an adjacency list has size
$\leq n+2a$ (better for sparse graphs).
\end{enumerate}

\paragraph{Isomorphisms}
\begin{enumerate}
\item \emph{Isomorphism}: bijection $f$ on nodes together with bijection
$g$ on arcs such that each arc $a$ between nodes $n_{1}$ and $n_{2}$
maps to the node $g\left(a\right)$ between nodes $f\left(n_{1}\right)$
and $f\left(n_{2}\right)$. Same adjacency matrix (possibly reordered).
\item \emph{Automorphism}: isomorphism from a graph to itself.
\end{enumerate}
\emph{Testing for isomorphisms}:
\begin{enumerate}
\item Check number of nodes, arcs and loops. Check degrees of nodes.
\item Attempt to find a bijection on nodes. Check using adjacency matrices.
\end{enumerate}
\emph{Finding the number of automorphisms}: find number of possibilities
for each node in turn. Then multiply together.


\paragraph{Planar Graphs}
\begin{enumerate}
\item \emph{Planar}: can be drawn so that no arcs cross.
\item \emph{Kuratowski's thm}: a graph is planar iff it does not contain
a subgraph homeomorphic to $K_{5}$ or $K_{3,3}$.\emph{ Homeomorphic}:
can be obtained from the graph by replacing the arc $x-y$ by $x-z-y$
where required.
\item For a planar graph, $F=A-N+2$.
\end{enumerate}

\paragraph{Graph Colouring}
\begin{enumerate}
\item \emph{$k$-colourable}: nodes can be coloured using no more than $k$
colours.
\item \emph{Four colour thm}: every simple planar graph can be coloured
with at most four colours.
\item \emph{Bipartite}: nodes can be partitioned into two sets such that
no two nodes from the same set are joined. Bipartite is equivalent
to 2-colourable.
\end{enumerate}

\paragraph{Paths and Connectedness}
\begin{enumerate}
\item \emph{Path}: sequence of adjacent arcs.
\item \emph{Connected}: there is a path joining any two nodes.
\item $x\sim y$ iff there is a path from $x$ to $y$. The equivalence
classes of $\sim$ are the \emph{connected components}.
\item \emph{Cycle}: path which finishes where it starts, has at least one
arc, does not use the same arc twice.
\item \emph{Acyclic}: graph with no cycles.
\item \emph{Euler path:} path which uses each arc exactly once. Exists iff
number of odd nodes is 0 or 2.
\item \emph{Euler circuit:} cycle which uses each arc exactly once. Exists
iff every node is even.
\item \emph{Hamiltonian path}: path which visits every node exactly once.
\item \emph{Hamiltonian circuit}: HP which returns to start node.
\end{enumerate}

\paragraph{Trees}
\begin{enumerate}
\item \emph{Tree}: acyclic, connected, rooted (has distinguished / root
node) graph.

\begin{enumerate}
\item Exists a unique non-repeating path between any two nodes.
\item A tree with $n$ nodes has $n-1$ arcs.
\end{enumerate}
\item \emph{Depth}: distance of node (along unique path) from root.
\item \emph{Spanning tree}: spanning subgraph which is a non-rooted tree
(lowest-cost network which still connects the nodes). All connected
graphs have spanning trees, not necessarily unique.
\item \emph{Minimum spanning tree}: spanning tree, no other spanning tree
has smaller weight. \emph{Weight} of spanning tree is sum of weights
of arcs.
\end{enumerate}

\paragraph{Directed Graphs}
\begin{enumerate}
\item \emph{Graph}: set of $N$ nodes and $A$ arcs such that each $a\in A$
is associated with an \emph{ordered} pair of nodes.
\item \emph{Indegree}: number of arcs entering given node.
\item \emph{Outdegree}: number of arcs leaving given node.
\item Sum of indegrees = sum of outdegrees = number of arcs.
\end{enumerate}

\paragraph{Weighted Graphs}

\emph{Weighted graph}: simple graph $G$ together with a weight function
$W:\mbox{arcs}\left(G\right)\rightarrow\mathbb{R}^{+}$.


\section{Graph Algorithms}


\subsection{Traversing a Graph}


\subsubsection{Depth First Search}

\begin{lstlisting}[basicstyle={\footnotesize\ttfamily}]
#### Depth First Search ####

visited[x] = true
print x
for y in adj[x]:        # Follow first available arc
  if not visited[y]:
    parent[y] = x
    dfs(y)              # Repeat from the discovered node
                        # Once all arcs followed, backtrack
\end{lstlisting}



\subsubsection{Breadth First Search}

\begin{lstlisting}[basicstyle={\footnotesize\ttfamily}]
#### Breadth First Search ####

visited[x] = true
print x
enqueue(x,Q)
while not isEmpty(Q):
  y = front(Q)
  for z in adj[y]:         # Follow all available arcs
    if not visited[z]:
      visited[z] = true
      print z
      parent[z] = y
         enqueue (z,Q)     # Repeat for each discovered node
  dequeue(Q)
\end{lstlisting}


For both methods, each node is processed once, each adjacency is list
processed once, giving running time of $O\left(n+a\right)$.


\paragraph{Applications}
\begin{enumerate}
\item Find if a graph is connected.
\item Find if a graph is cyclic: use DFS, if we find a node that we have
already visited (except by backtracking) there is a cycle.
\item Find distance from start node: BFS with a counter.
\end{enumerate}

\subsection{Finding Minimum Spanning Trees}


\subsubsection{Prim's Algorithm}

\begin{lstlisting}[basicstyle={\footnotesize\ttfamily}]
#### Prim's Algorithm ####

                                  # Choose any node as start
tree[start] = true                # Initialise tree as start
for x in adj[start]:              # and fringe as adjacent nodes
  fringe[x] = true
  parent[x] = start
  weight[x] = W[start,x]
while fringe nonempty:
  f = argmin(weight)              # Select node f from fringe 
                                  # s.t. weight is minimum
  tree[f] = true                  # Add f to tree
  fringe[f] = false               # and remove from fringe
  for y in adj[f]:
    if not tree[y]:
      if fringe[y]:               # Update min weights
                                  # for nodes already in fringe
        if W[f,y] < weight[y]:
          weight[y] = W[f,y]
          parent[y] = f
      else:                       # and add any unseen nodes 
                                  # adjacent to f to fringe
        fringe[y] = true
        weight[y] = W[f,y]
        parent[y] = f
\end{lstlisting}


Method takes $O\left(n^{2}\right)$ time. For each node, we check
at most $n$ nodes to find $f$.


\paragraph{Correctness of Prim's}

\emph{Inductive step}: Assume that $T_{k}\subseteq T'$ where $T'$
is some MST of a graph $G$. Suppose we choose $a_{k+1}\notin\mbox{arcs}\left(T'\right)$,
between nodes $x$ and $y$. There must be another path from $x$
to $y$. So this path uses another arc $a$ that crosses the fringe
- but $W\left(a_{k+1}\right)\leq W\left(a\right)$ so $T_{k+1}\subseteq T'$.


\paragraph{Implementation of Prim's with Priority Queues}

Using binary heap implementation, operations are $O\left(\log n\right)$
except isEmpty and getMin which are $O\left(1\right)$. This gives
$O\left(m\log n\right)$ overall - better for sparse graphs.


\subsubsection{Kruskal's Algorithm}

\begin{lstlisting}[basicstyle={\footnotesize\ttfamily},showstringspaces=false]
#### Kruskal's Algorithm ####

                         # Assign nodes of G numbers 1..n
Q = ...                  # Build a PQ of arcs with weights as keys
sets = UFcreate(n)       # Initialise Union-Find with singletons
F = {}                   # Initialise empty forest
while not isEmpty(Q):
  (x,y) = getMin(Q)      # Choose the arc of least weight
  deleteMin(Q)
  x' = find(sets,x)      # Find which components it connects
  y' = find(sets,y)
  if x' != y':           # If components are different/no cycle made
    add (x,y) to F       # add arc to forest
    union(sets,x',y')    # and merge two components
\end{lstlisting}



\paragraph{Implementation of Kruskal's with Non-binary Trees}
\begin{enumerate}
\item Each set stored as non-binary tree, where root node represents leader
of set.
\item Merge sets by appending one tree to another. We want to limit depth,
so always append tree of lower size to one of greater size.
\item Depth of a tree of size $k$ is then $\leq\left\lfloor \log k\right\rfloor $.
\end{enumerate}
Gives $O\left(m\log m\right)$ which, since $m$ is bounded by $n^{2}$,
is equivalent to $O\left(m\log n\right)$. Better for sparse graphs.


\paragraph{Improvements for Kruskal's with Path Compression}
\begin{enumerate}
\item When finding root/leader of component containing the node $x$, if
it is not $\mbox{parent}\left[x\right]$, then we make $\mbox{parent}\left[y\right]=\mbox{root}$
for all $y$ on the path from $x$ to the root.
\item This gives $O\left(\left(n+m\right)\log^{*}n\right)$ where $\log^{*}n$
is a very slowly growing function.
\end{enumerate}

\subsection{Shortest Path Problem}


\subsubsection{Dijkstra's Algorithm}

\begin{lstlisting}[basicstyle={\footnotesize\ttfamily},showstringspaces=false]
#### Dijkstra's Algorithm ####

tree[start] = true                    # Init tree as start node
for x in adj[start]:                  # Init fringe as adj nodes
  fringe[x] = true
  parent[x] = start
  distance[x] = W[start,x]
while not tree[finish] and fringe nonempty:
  f = argmin(distance)                # Choose node with min distance
  fringe[f] = false                   # remove from fringe
  tree[f] = true                      # and add to tree
  for y in adj[f]:                    # For adj nodes
    if not tree[y]:                   # if in fringe, update distance
      if fringe[y]:
        if distance[f] + W[f,y] < distance[y]:
          distance[y] = distance[f] + W[f,y]
          parent[y] = f
      else:                           # otherwise, add to fringe
        fringe[y] = true
        distance[y] = distance[f] + W[f,y]
        parent[y] = f
return distance[finish]
\end{lstlisting}


Requires $O\left(n^{2}\right)$ steps (or $O\left(m\log n\right)$
with PQ).


\paragraph{Correctness}

Prove by the following invariant:
\begin{enumerate}
\item If $x$ is a tree or fringe node, then $\mbox{parent}\left[x\right]$
is a tree node.
\item If $x$ is a tree node, then $\mbox{distance}\left[x\right]$ is the
length of the shortest path and $\mbox{parent}\left[x\right]$ is
its predecessor along that path.
\item If $f$ is a fringe node then $\mbox{distance}\left[f\right]$ is
the length of the shortest path where all nodes except $f$ are tree
nodes.
\end{enumerate}

\subsubsection{Floyd's Algorithm}

\begin{lstlisting}[basicstyle={\footnotesize\ttfamily},showstringspaces=false]
#### Floyd's Algorithm ####

A                                  # Input adj matrix
B[i,j]
  if i = j     : 0
  if A[i,j] > 0: A[i,j]
  otherwise    : INFINITY          # Init shortest paths via no nodes
for k = 1 to n:
  for i = 1 to n:
    for j = 1 to n:
      B[i,j] = min(B[i,j],         # New shortest path is direct
               B[i,k] + B[k,j])    # or concat shortest paths via k
return B
\end{lstlisting}


\emph{Warshall's algorithm} is the same but uses Boolean values (determines
whether a path exists). Both have complexity $O\left(n^{3}\right)$.


\paragraph{Correctness}

We discuss the \emph{inductive step}: Suppose there is a shortest
path $p$ from $i$ to $j$ using nodes with identifiers $<k$, of
length $d$. Either:
\begin{enumerate}
\item $k$ is not an intermediate node of $p$. Then $B_{k-1}\left[i,j\right]=d$
already.
\item $k$ is an intermediate node of $p$. Then $B_{k-1}\left[i,j\right]=B_{k-1}\left[i,k\right]+B_{k-1}\left[k,j\right]$
since these give shortest paths to and from $k$.
\end{enumerate}

\paragraph{Dynamic Programming}

Both algorithms are examples of dynamic programming:
\begin{enumerate}
\item Break down the main problem into sub-problems.
\item Sub-problems are ordered and culminate in the main problem.
\end{enumerate}

\subsection{The Travelling Salesman Problem}


\paragraph{Problem}

Given a \emph{complete} weighted graph, find a minimum weight tour
of the graph visiting each node exactly once.


\subsubsection{Bellman-Held-Karp Algorithm}

\begin{lstlisting}[basicstyle={\footnotesize\ttfamily},showstringspaces=false]
#### Bellman-Held-Karp Algorithm ####

start = ...                           # Choose some start node
for x in Nodes \ {start}:             # Set costs on empty set
  C[{},x] = W[start,x]                # as direct weight from start
for S in subsets(Nodes \ {start}):    # For sets S of increasing size
  for x in Nodes \ (S union {start}): # for each node x out of set
    c[S,x] = INFINITY
    for y in S:
      C[S,x]                          # update the cost to minimum
        = min(C[S \ {y},y] + W[y,x],  # that visits all nodes in S
          C[S,x])                     # and then x
opt = INFINITY
for x in Nodes \ {start}:             # Then choose the min value
  opt = min(C[Nodes \ {start,x},x]    # for an S without x plus
        + W[x,start],opt)             # weight of x from start
return opt
\end{lstlisting}


Dynamic programming approach that requires $O\left(n^{2}2^{n}\right)$
steps.


\subsubsection{Nearest Neighbour Algorithm}

Always choose the the shortest available arc. Not optimal but has
$O\left(n^{2}\right)$ complexity.


\section{Algorithm Analysis}


\subsection{Searching Algorithms}


\subsubsection{Searching an Unordered List $\boldsymbol{L}$}

For an element $x$:
\begin{enumerate}
\item If $x$ in $L$ return $k$ s.t. $L\left[k\right]=x$.
\item Otherwise return ``not found''.
\end{enumerate}

\paragraph{Optimal Algorithm: Linear Search}

Inspect elements in turn. With input size $n$ we have time complexity
of $n$ in worst case.

Linear search is optimal. Consider any algorithm $A$ which solves
the search problem:
\begin{enumerate}
\item \emph{Claim:} if $A$ returns ``not found'' then it must have inspected
every entry of $L$.
\item \emph{Proof:} suppose for contradiction that $A$ did not inspect
some $L\left[k\right]$. On an input $L'$ where $L'\left[k\right]=x$,
$A$ will return ``not found'', which is wrong. So in worst case
$n$ comparisons are needed.
\end{enumerate}

\subsubsection{Searching an Ordered List $\boldsymbol{L}$}


\paragraph{Optimal Algorithm: Binary Search}

Where $W\left(n\right)$ is the number of inspections required for
a list of length $n$, $W\left(1\right)=1$ and $W\left(n\right)=1+W\left(\left\lfloor n/2\right\rfloor \right)$.
Gives $W(n)=1+\left\lfloor \log_{2}n\right\rfloor $.
\begin{enumerate}
\item \emph{Proposition}: if a binary tree has depth $d$, then it has $\leq2^{d+1}-1$
nodes.
\item \emph{Base Case}: depth 0 has less than or equal to $2-1=1$ node.
True.
\item \emph{Inductive Step}: assume true for $d$. Suppose tree has depth
$d+1$. Children have depth $\leq d$ and so $\leq2^{d+1}-1$ nodes.
Total number of nodes $\leq1+2\times\left(2^{d+1}-1\right)=2^{(d+1)+1}-1$.
\item \emph{Claim}: any algorithm $A$ must do as many comparisons as binary
search.
\item \emph{Proof}: the tree for algorithm $A$ has $n$ nodes. If depth
is $d$ then $n\leq2^{d+1}-1$. Hence $d+1\geq\left\lceil \log\left(n+1\right)\right\rceil $.
For binary search, $W(n)=1+\left\lfloor \log n\right\rfloor $ which
is equivalent to $\left\lceil \log\left(n+1\right)\right\rceil $.
\end{enumerate}

\subsection{Orders}
\begin{enumerate}
\item $f$ is $O\left(g\right)$ if $\exists m\in\mathbb{N}\exists c\in\mathbb{R}^{+}.\left[\forall n\geq m.\left(f\left(n\right)\leq c\times g\left(n\right)\right)\right]$.
\item $f$ is $\Theta\left(g\right)$ iff $f$ is $O\left(g\right)$ and
$g$ is $O\left(f\right)$.
\end{enumerate}

\subsection{Divide and Conquer}
\begin{enumerate}
\item Divide problem into $b$ subproblems of size $n/c$.
\item Solve each subproblem recursively.
\item Combine to get the result.
\end{enumerate}

\paragraph{E.g. Strassen's Algorithm for Matrix Multiplication}
\begin{enumerate}
\item Add extra row and or column to keep dimension even.
\item Divide up matrices into four quadrants, each $n/2\times n/2$.
\item Compute each quadrant result with just 7 multiplications (instead
of 8).
\end{enumerate}

\subsection{Sorting Algorithms}


\subsubsection{Insertion Sort}
\begin{enumerate}
\item Insert $L\left[i\right]$ into $L\left[0..i-1\right]$ in the correct
position. Then $L\left[0..i\right]$ is sorted.
\item This takes between 1 and $i$ comparisons.
\item \emph{Worst case}: $W\left(n\right)=\sum_{i=1}^{n-1}i=\frac{n(n-1)}{2}$
(occurs when in reverse order). Order $\Theta\left(n^{2}\right)$.
\end{enumerate}

\subsubsection{Merge Sort}
\begin{enumerate}
\item Divide rougly in two.
\item Sort each half seperately (by recursion).
\item Merge two halves.
\item \emph{Worst case}: $W\left(n\right)=n-1+W\left(\left\lceil \frac{n}{2}\right\rceil \right)+W\left(\left\lfloor \frac{n}{2}\right\rfloor \right)=n\log\left(n\right)-n+1$
assuming $n$ can be written as $2^{k}$.
\end{enumerate}

\subsubsection{Quick Sort}
\begin{enumerate}
\item Split around the first element.
\item Sort the two sides recursively.
\item \emph{Worst case}: $W\left(n\right)=n-1+W\left(n-1\right)=\frac{n\left(n-1\right)}{2}$
(occurs when already sorted).
\item \emph{Average case}: $A\left(n\right)=n-1+\frac{1}{n}\sum_{s=1}^{n}\left(A\left(s-1\right)+A\left(n-s\right)\right)=n-1+\frac{2}{n}\sum_{i=2}^{n-1}A\left(i\right)$,
asuming each position $s$ is equally likely. $A\left(n\right)$ is
order $\Theta\left(n\log n\right)$.
\end{enumerate}

\subsubsection{Heap Sort}
\begin{enumerate}
\item Add elements to a priority queue, implemented with binary heap. Read
off the queue.
\item Is $\Theta(n\log n)$ - also can be performed in place.

\begin{enumerate}
\item $\Theta\left(n\right)$ inserts each taking $\Theta\left(\log n\right)$.
\item $\Theta\left(n\right)$ gets each taking $\Theta\left(1\right)$.
\item $\Theta\left(n\right)$ deletions each taking $\Theta\left(\log n\right)$.
\end{enumerate}
\end{enumerate}

\subsubsection{Parallel Sorting}
\begin{enumerate}
\item Merge sort can be parallelised by executing recursive calls in parallel.
\item Work still same but time taken reduced.
\item \emph{Worst case (time taken)}: $W'\left(n\right)=n-1+W'\left(\frac{n}{2}\right)=2n-2-\log n$.
\end{enumerate}

\subsubsection{Odd / Even Merge}
\begin{enumerate}
\item To merge $L_{1}$ and $L_{2}$:

\begin{enumerate}
\item Take odd positions and merge to get $L_{3}$ ($O\left(\log n\right)$).
\item Take even positions and merge to get $L_{4}$ ($O\left(\log n\right)$).
\item Do an interleaving merge on $L_{3}$ and $L_{4}$ ($O\left(1\right)$).
\end{enumerate}
\item Merge is $O\left(\log n\right)$ instead of $O\left(n\right)$ (sequential).
\item Exploits odd/even networks (made up of comparators).
\item Time taken is now $\left(\log n\right)\left(1+\log n\right)/2$ which
is $\Theta\left(\log^{2}n\right)$.
\end{enumerate}

\subsubsection{Lower Bounds}

We express the sorting algorithm as a \emph{decision tree}. Internal
nodes are \emph{comparisons}. Leaves are \emph{results}. The worst
case number of comparisons is given by depth.


\paragraph{Minimising Depth (Worst Case)}
\begin{enumerate}
\item There are $n!$ permutations so we require $n!$ leaves.
\item \emph{Proposition}: if a binary tree had depth $d$ then it has $\leq2^{d}$
leaves.
\item \emph{Proof}: simple proof by induction.
\item \emph{Lower bound (worst case)}: $2^{d}\geq n!$ so $d\geq\left\lceil \log\left(n!\right)\right\rceil $.
\item $\log\left(n!\right)=\sum_{1}^{n}\log\left(k\right)\approx\int_{1}^{n}\log x\mbox{ d}x=n\log\left(n\right)-n+1$.
\end{enumerate}

\paragraph{Minimising Total Path Length (Average Case)}
\begin{enumerate}
\item \emph{Balanced tree}: tree at which every leaf is at depth $d$ or
$d-1$.
\item \emph{Proposition}: if a tree is unbalanced then we can find a balanced
tree with the same number of leaves without increasing total path
length.
\item \emph{Lower bound (average case)}: must peform at least $\left\lfloor \log\left(n!\right)\right\rfloor $
calculations (since total path length is minimum for balanced trees).
\end{enumerate}

\subsection{Master Theorem}

For $T\left(n\right)=aT\left(\frac{n}{b}\right)+f\left(n\right)$,
with the critical exponent $E=\frac{\log a}{\log b}$:
\begin{enumerate}
\item If $n^{E+\epsilon}=O\left(f\left(n\right)\right)$ for some $\epsilon>0$
then $T\left(n\right)=\Theta\left(f\left(n\right)\right)$.
\item If $f\left(n\right)=\Theta\left(n^{E}\right)$ then $T\left(n\right)=\Theta\left(f\left(n\right)\log n\right)$.
\item If $f\left(n\right)=O\left(n^{E-\epsilon}\right)$ for some $\epsilon>0$
then $T\left(n\right)=\Theta\left(n^{E}\right)$.
\end{enumerate}

\section{Complexity}


\subsection{Tractable Problems and P}

\textbf{P}: Class of decision problems that can be easily \emph{solved}.
\begin{enumerate}
\item \emph{Tractable problem}: efficiently computable, focusing on worst
case.
\item \emph{Decision problem}: Has a yes or no answer.
\item \emph{Cool-Karp Thesis}: Problem is tractable if it can be computed
within polynomially many steps in the worst case.
\item \emph{Polynomial Invariance Thesis}: If a problem can be solved in
p-time in some reasonable model of computation, then it can be solved
in p-time in any other reasonable model.
\item P: A decision problem $D\left(x\right)$ is in P if it can be decided
within time $p\left(n\right)$ in some reasonable model of computation,
where $n$ is the size of the input, $\left|x\right|$.
\end{enumerate}

\paragraph{Unreasonable Models}
\begin{enumerate}
\item \emph{Superpolynomial Parallelism}: I.e. take out more than polynomially
many operations in parallel in a single step.
\item \emph{Unary numbers}: Gives input size exponentially larger than binary.
\end{enumerate}

\paragraph{Polynomial Time Functions}
\begin{enumerate}
\item \emph{Arithmetical operations} are p-time.
\item If $f$ is a p-time function, then its \emph{ouput size} is polynomially
bounded in the \emph{input size}. I.e. $\left|f\left(x\right)\right|\leq p\left(\left|x\right|\right)$
- because we only have p-time to build the output.
\item \emph{Composition}: If functions $f$ and $g$ are p-time, then $g\circ f$
is computable. Since $f$ takes $p\left(n\right)$ and $g$ then takes
$q\left(p'\left(n\right)\right)$ where $p'\left(n\right)$ is the
output size of $f$.
\end{enumerate}

\subsection{NP}

\textbf{NP}: Class of decision problems that can be easily \emph{verified}.


\subsubsection{HamPath Problem}

Given a graph $G$ and a list $p$, is $p$ a Ham path of $G$?
\begin{enumerate}
\item Verification of HamPath is in P. $\mbox{HamPath}\left(G\right)$ iff
$\exists p\mbox{Ver-HamPath}\left(G,p\right)$.
\item $D\left(x\right)$ is in NP if there is a problem $E\left(x,y\right)$
in P and a polynomial $p\left(n\right)$ such that:

\begin{enumerate}
\item $D\left(x\right)$ iff $\exists y.E\left(x,y\right)$.
\item If $E\left(x,y\right)$ then $\left|y\right|\leq p\left(\left|x\right|\right)$
($E$ is poly bounded).
\end{enumerate}
\end{enumerate}

\subsubsection{Satisfiability Problem}

Given a propositional formula $\phi$ in CNF, is it satisfiable?
\begin{enumerate}
\item SAT is not decidable in p-time: we have to try all possible truth
assignments - for $m$ variables this is $2^{m}$ assignments.
\item SAT is in NP: Guess an assignment $v$ and verify in p-time that $v$
satisfies $\phi$.
\end{enumerate}

\subsubsection{P$\boldsymbol{\subseteq}$NP}
\begin{enumerate}
\item Suppose that $D$ is in P.
\item To verify $D\left(x\right)$ holds, we don't need to guess a certificate
$y$ - we just decide $D\left(x\right)$ directly.
\item Formally: Define $E\left(x,y\right)$ iff $D(x)$ and $y$ is the
empty string. Then clearly $D\left(x\right)$ iff $\exists y.E\left(x,y\right)$
and $\left|y\right|\leq p\left(\left|x\right|\right)$.
\end{enumerate}

\subsubsection{P$\boldsymbol{=}$NP}

Unknown.


\subsection{Problem Reduction}
\begin{enumerate}
\item \emph{Many-one reduction}: Consider two decision problems $D$ and
$D'$. $D$ many-one reduces to $D'$ ($D\leq D'$) if there is a
p-time function $f$ s.t. $D\left(x\right)$ iff $D'\left(f\left(x\right)\right)$.
\item Reduction is reflexive and transitive.
\item If both $D\leq D'$ and $D'\leq D$, then $D\sim D'$.
\end{enumerate}

\subsubsection{P and Reduction}
\begin{enumerate}
\item Suppose algorithm $A'$ decides $D'$ in time $p'\left(n\right)$.
We define algorithm $A$ to solve $D$ by first computing $f\left(x\right)$
and then running $A'\left(f\left(x\right)\right)$.

\begin{enumerate}
\item Step 1 takes $p\left(n\right)$ steps.
\item Note that $\left|f\left(x\right)\right|\leq q\left(n\right)$ for
some poly $q$. Step 2 takes $p'\left(q\left(n\right)\right)$.
\end{enumerate}
\item Hence: If $D\leq D'$ and $D'\in\mbox{P}$, then $D\in\mbox{P}$.
\end{enumerate}

\subsubsection{NP and Reduction}


\paragraph{Assumption}
\begin{enumerate}
\item Assume $D\leq D'$ and $D'\in\mbox{NP}$.
\item Then $D\left(x\right)$ iff $D'\left(f\left(x\right)\right)$.
\item Also there is $E'\left(x,y\right)\in\mbox{P}$ s.t. $D'\left(x\right)$
iff $\exists y.E'\left(x,y\right)$.
\item Also if $E'\left(x,y\right)$ then $\left|y\right|\leq p'\left(\left|x\right|\right)$.
\end{enumerate}

\paragraph{Proof Part A}
\begin{enumerate}
\item $D\left(x\right)$ iff $\exists y.E'\left(f\left(x\right),y\right)$.
\item Define then $E\left(x,y\right)$ iff $E'\left(f\left(x\right),y\right)$.
We can now prove part (a).
\end{enumerate}

\paragraph{Proof Part B}
\begin{enumerate}
\item Suppose $E\left(x,y\right)$. Then $E'\left(f\left(x\right),y\right)$,
hence $\left|y\right|\leq p'\left(\left|x\right|\right)$.
\item $f\left|x\right|<q\left(n\right)$ for some poly $q$. So $\left|y\right|\leq p'\left(q\left(\left|x\right|\right)\right)$,
proving part (b).
\end{enumerate}

\subsection{NP-Completeness}
\begin{enumerate}
\item $D$ is NP-\emph{hard} if for all problems $D'\in\mbox{NP}$, $D'\leq D$.
\item $D$ is NP-\emph{complete} if for all problems $D\in\mbox{NP}$ and
$D$ is NP-hard.
\end{enumerate}

\paragraph{Cook-Levin Theorem}

SAT is NP-complete.


\paragraph{Intractability}
\begin{enumerate}
\item Suppose $\mbox{P}\neq\mbox{NP}$ and $D$ is NP-hard.
\item Suppose for a contradiction that $D\in\mbox{P}$.
\item Consider any other $D'\in\mbox{NP}$. $D'\leq D$ since $D$ is NP-hard.
Hence $D\in\mbox{P}$.
\item Hence $\mbox{NP}\subseteq\mbox{P}$. But $\mbox{P}\subseteq\mbox{NP}$.
Hence $\mbox{P}=\mbox{NP}$, which is a contradiction.
\end{enumerate}

\paragraph{Proving NP-Completeness}

E.g. Prove TSP is NP-complete.
\begin{enumerate}
\item Prove TSP is NP (easy).
\item Reduce HamPath (a known NP-hard problem) to TSP.\end{enumerate}

\end{document}
