\documentclass[10pt,twoside,twocolumn]{article}
\usepackage[latin9]{inputenc}
\usepackage[landscape]{geometry}
\geometry{verbose,tmargin=0.5in,bmargin=0.75in,lmargin=0.5in,rmargin=0.5in}
\setlength{\parskip}{\smallskipamount}
\setlength{\parindent}{0pt}
\usepackage{amsbsy}

\makeatletter

\providecommand{\tabularnewline}{\\}

\usepackage{amsmath,amssymb,amsthm,mathrsfs,amsfonts,dsfont} 

\setlength{\columnsep}{0.25in}

\renewcommand*{\thefootnote}{\fnsymbol{footnote}}

\newcommand{\R}[0]{\mathds{R}} % real numbers
\newcommand{\Z}[0]{\mathds{Z}} % integers
\newcommand{\N}[0]{\mathds{N}} % natural numbers
\newcommand{\nat}[0]{\mathds{N}} % natural numbers
\newcommand{\Q}[0]{\mathds{Q}} % rational numbers

\makeatother

\begin{document}

\title{Reference Sheet for CO140 Logic}


\date{Autumn 2016}

\maketitle

\section{Definitions}


\subsection{Propositional Logic}


\paragraph{Binding Conventions}

(Strongest) $\lnot,\land,\lor,\rightarrow,\leftrightarrow$ (Weakest).


\paragraph{Propositional Formula}
\begin{enumerate}
\item A propositional atom is a formula.
\item $\top$ and $\bot$ are formulas.
\item If $A$ is a formula then so is $(\lnot A)$.
\item If \textbf{$A,B$} are formulas then so are $(A\land B),(A\lor B),(A\rightarrow B),(A\leftrightarrow B)$.
\end{enumerate}

\paragraph{Principle Connective}

Connective at the root (top) of a formation tree. A formula with principle
connective $\leftrightarrow$ is said to have the \textbf{logical
form $A\leftrightarrow B$}.


\paragraph{Subformulas}

Correspond to the subtrees of a formation tree.


\paragraph{Atomic}

Formula of the form \textbf{$\top,\bot,p$ }for an atom $p$.


\paragraph{Negated Formula, Conjunction, Disjunction, Implication}

Formula whose logical form is $\lnot A,A\land B,A\lor B,A\rightarrow B$
respectively.


\paragraph{Literal}

Formula that is atomic or negated-atomic.


\paragraph{Clause}

Disjunction of one or more literals.


\paragraph{Situation}

Determines whether each propositional atom is true or false.


\paragraph{Evaluation}
\begin{enumerate}
\item $\top$ is true, $\bot$ is false.
\item $\lnot A$ has the opposite truth value to $A$.
\item $A\land B$ is true if $A$ and $B$ are both true.
\item $A\lor B$ is true if one or both of $A$ and $B$ are true.
\item $A\rightarrow B$ is true if $A$ is false or $B$ is true (or both).
\item $A\leftrightarrow B$ is true if $A$ and \textbf{$B$} have the same
truth value.
\end{enumerate}

\paragraph{Valid Argument}

Given formulas $A_{1},A_{2},\dots,A_{n},B$ an argument $A_{1},A_{2},\dots,A_{n}\vDash B$
is valid if $B$ is true in any situation in which $A_{1},A_{2},\dots,A_{n}$
are all true. Here $\vDash$ denotes logical entailment.


\paragraph{Valid Formula}

A formula $A$ is valid if it is true in every situation, i.e. $\vDash A$.
A \textbf{tautology} is a valid propositional formula.


\paragraph{Satisfiable Formula}

True in at least one situation.


\paragraph{Equivalent Formulas}

True in exactly the same situations, i.e. $A\equiv B$.


\paragraph{Disjunctive Normal Form}

Formula as a disjunction of conjunctions of literals, not further
simplifiable.


\paragraph{Conjunctive Normal Form}

Formula as a conjunction of disjunction of literals, not further simplifiable.


\paragraph{Normal Form}
\begin{enumerate}
\item Get rid of $\rightarrow,\leftrightarrow$ using equivalences.
\item Use De Morgan laws to push negations down to atoms. Delete any double
negations.
\item Rearrange using distributivity into the required normal form.
\item Use equivalences to simplify as far as possible (e.g. using absorption,
idempotence, equivalences involving $\top$ and $\bot$).
\end{enumerate}

\paragraph{Theorem}

Formula that can be established by a given proof system, i.e. any
$A$ such that $\vdash A$. (Note that $\vdash$ is syntactic whilst
$\vDash$ is semantic - $A_{1},A_{2},\dots,A_{n}\vDash B$ means there
is a proof of $B$ starting with $A_{1},A_{2},\dots,A_{n}$ as givens).


\paragraph{Soundness}

Any provable formula is valid, i.e. if $A_{1},A_{2},\dots,A_{n}\vdash B$
then $A_{1},A_{2},\dots,A_{n}\vDash B$.


\paragraph{Completeness}

Any valid formula can be proved, i.e. if $A_{1},A_{2},\dots,A_{n}\vDash B$
then $A_{1},A_{2},\dots,A_{n}\vdash B$.


\paragraph{Consistency}

A formula is consistent if $\not\vdash\lnot A$. So a formula is consistent
if and only if it is satisfiable.


\subsection{Predicate Logic}


\paragraph{Binding Conventions}

(Strongest) $(\lnot,\forall x,\exists x),\land,\lor,\rightarrow,\leftrightarrow$ (Weakest).


\paragraph{Signature}

Collection of constants and relation symbols and function symbols
with specified arities.

\paragraph{Term}

For a signature $L$:
\begin{enumerate}
\item Any constant in $L$ is an $L$-term.
\item Any variable is an $L$-term.
\item For an $n$-ary function symbol $f$ in $L$ and $L$-terms $t_{1},t_{2},\dots,t_{n}$,
$f\left(t_{1},t_{2},\dots,t_{n}\right)$ is an $L$-term.
\end{enumerate}

\paragraph{Closed / Ground Term}

Does not involve a variable.


\paragraph{Bound Variable}

For a formula $A$ and variable $x$, $x$ is bound if it lies under
a quantifier $\forall x$ or $\exists x$ in the formation tree of
$A$.


\paragraph{Free Variable}

Variable which is not bound (this includes variables which do not
appear in $A$!).


\paragraph{Sentence}

Formula with no free variables. (Does not require an assignment for
evaluation).


\paragraph{Structure}

For a signature $L$, and $L$-structure $M$:
\begin{enumerate}
\item Identifies a non-empty collection of objects that $M$ `knows about',
i.e. the \textbf{domain} of $M$, $\mbox{dom}\left(M\right)$.
\item Specifies what the symbols of $L$ mean in terms of these objects
(constants specify objects in $\mbox{dom}\left(M\right)$ and relations
specify relations between objects in $\mbox{dom}\left(M\right)$).
\item For an $n$-ary function symbol $f$ in $L$, specifies which object
$f$ associates with each sequence of objects $\left(a_{1},a_{2},\dots,a_{n}\right)$
in $\mbox{dom}\left(M\right)$.
\end{enumerate}
For a constant $c$, $c^{M}$ denotes the object $\mbox{dom}\left(M\right)$
that $c$ names in $M$.

For a function $f$, $f^{M}\left(a_{1},a_{2},\dots,a_{n}\right)$
denotes the object $\mbox{dom}\left(M\right)$ that $f\left(a_{1},a_{2},\dots,a_{n}\right)$
names in $M$.

If a formula $A$ is true in $M$, we say $M\vDash A$.


\paragraph{Assignment}

For a structure $M$, allocates an object in $\mbox{dom}\left(M\right)$
to each variable.

If a formula $A$ is true in $M$ under $h$, we say $M,h\vDash A$.


\paragraph{Value of Term}

For a signature $L$, an $L$-structure $M$ and an assignment $h$,
for any $L$-term $t$, the value of $t$ in $M$ under $h$ is the
object in $\mbox{dom}\left(M\right)$ allocated to $t$ by:
\begin{enumerate}
\item $M$ if $t$ is a constant, i.e. the object $t^{M}$.
\item $h$ if $t$ is a variable, i.e. the object $h\left(t\right)$.
\item $f^{M}$ if $t$ is a function on terms, i.e. $f\left(t_{1},t_{2},\dots,t_{n}\right)$
is the object $f^{M}\left(a_{1},a_{2},\dots,a_{n}\right)$ where $a_{i}$
is the value of $t_{i}$ in $M$ under $h$.
\end{enumerate}

\paragraph{Predicate Formula}

For an $L$-structure $M$ and an assignment $h$:
\begin{enumerate}
\item For an $n$-ary relation symbol in $L$, and $L$-terms $t_{1},t_{2},\dots,t_{n}$,
$R\left(t_{1},t_{2},\dots,t_{n}\right)$ is an atomic $L$-formula.\\
$M,h\vDash R\left(t_{1},t_{2},\dots,t_{n}\right)$ if $M$ says the
sequence $\left(a_{1},a_{2},\dots,a_{n}\right)$ is in the relation
$R$, where $a_{i}$ is the value of $t_{i}$ in $M$ under $h$.
\item For $L$-terms $t_{1},t_{2}$, $t_{1}=t_{2}$ is an atomic $L$-formula.\\
$M,h\vDash t_{1}=t_{2}$ if $t_{1}$ and $t_{2}$ have the same value
in $M$ under $h$.
\item $\top,\bot$ are atomic $L$-formulas.\\
$M,h\vDash\top$ and $M,h\not\vDash\bot$.
\item For $L$-formulas $A,B$, $\left(\lnot A\right),\left(A\land B\right),\left(A\lor B\right),\left(A\rightarrow B\right),\left(A\leftrightarrow B\right)$
are $L$-formulas.\\
$M,h\vDash A\land B$ if $M,h\vDash A$ and $M,h\vDash B$, etc.
\item For an $L$-formula $A$ and a variable $x$, $\left(\forall xA\right)$
and $\left(\exists xA\right)$ are $L$-formulas.\\
$M,h\vDash\forall xA$ if $M,g\vDash A$ for every assignment $g$
into $M$ with $g=_{x}h$ and $M,h\vDash\exists xA$ if $M,g\vDash A$
for some assignment $g$ into $M$ with $g=_{x}h$. (The notation
$g=_{x}h$ here means $g$ agrees with $h$ except perhaps on $x$).
\end{enumerate}

\paragraph{Some Translation Advice}

Note that:
\begin{enumerate}
\item $\forall x\left(\mbox{lecturer}\left(x\right)\land\mbox{human}\left(x\right)\right)$
says everything is a lecturer and a human.
\item $\forall x\left(\mbox{lecturer}\left(x\right)\rightarrow\mbox{human}\left(x\right)\right)$
says every lecturer is a human.
\item $\exists x\left(\mbox{lecturer}\left(x\right)\land\mbox{human}\left(x\right)\right)$
says there is a lecturer that is also a human.
\item $\exists x\left(\mbox{lecturer}\left(x\right)\rightarrow\mbox{human}\left(x\right)\right)$
says there is a non-lecturer, or there is a lecturer that is also
a human.
\end{enumerate}
Counting:
\begin{enumerate}
\item No lecturers: $\lnot\exists x\left(\mbox{lecturer}\left(x\right)\right)$.
\item At least one lecturer: $\exists x\left(\mbox{lecturer}\left(x\right)\right)$.
\item At least two lecturers: $\exists x\exists y\left(\mbox{lecturer}\left(x\right)\land\mbox{lecturer}\left(y\right)\land x\neq y\right)$
or $\forall x\exists y\left(\mbox{lecturer}\left(y\right)\land y\neq x\right)$.
\item At least three lecturers: similar to above, e.g. $\forall x\forall y\exists z\left(\mbox{lecturer}\left(z\right)\land z\neq x\land z\neq y\right)$.
\item At most one lecturer: $\lnot\exists x\exists y\left(\mbox{lecturer}\left(x\right)\land\mbox{lecturer}\left(y\right)\land z\neq y\right)$
or $\forall x\forall y\left(\mbox{lecturer}\left(x\right)\land\mbox{lecturer}\left(y\right)\rightarrow x=y\right)$
or $\exists x\forall y\left(\mbox{lecturer}\left(y\right)\rightarrow y=x\right)$.
\item Exactly one lecturer: at least one lecturer$\land$at most one lecturer
or $\exists x\forall y\left(\mbox{lecturer}\left(y\right)\leftrightarrow y=x\right)$.
\end{enumerate}
Remember:
\begin{enumerate}
\item Always consider the vacuous case when using $\forall$.
\item The order of quantifiers is very important!
\end{enumerate}


\paragraph{Valid Argument}

For a signature $L$, and $L$-formulas $A_{1},A_{2},\dots,A_{n},B$,
the argument $A_{1},A_{2},\dots,A_{n}\vDash B$ is valid if for any
$L$-structure $M$ and assignment $h$ into $M$, if $M,h\vDash B$
given $M,h\vDash A_{1},M,h\vDash A_{2},\dots,M,h\vDash A_{n}$.


\paragraph{Valid Formula}

The $L$-formula $A$ is valid if for for all $L$-structures $M$
and assignment $h$ into $M$, $M,h\vDash A$. We say $\vDash A$.


\paragraph{Satisfiable Formula}

The $L$-formula $A$ is satisfiable if for for some $L$-structure
$M$ and assignment $h$ into $M$, $M,h\vDash A$.


\paragraph{Equivalent Formulas}

The $L$-formulas $A$ and $B$ are equivalent if for for every $L$-structure
$M$ and assignment $h$ into $M$, $M,h\vDash A$ if and only if
$M,h\vDash B$.


\subsection{Many-Sorted Predicate Logic}


\paragraph{Term}

Redefined such that:
\begin{enumerate}
\item Each variable and constant comes with a sort $\mbox{s}$. We indicate
this as $x:\mbox{s}$ and $c:\mbox{s}$.
\item Each $n$-ary function symbol $f$ comes with a template $f:\left(\mbox{s}_{1},\mbox{s}_{2},\dots,\mbox{s}_{n}\right)\rightarrow\mbox{s}$.
\end{enumerate}

\paragraph{Formula}

Redefined such that:
\begin{enumerate}
\item Each $n$-ary relation symbol $R$ comes with a template $R\left(\mbox{s}_{1},\mbox{s}_{2},\dots,\mbox{s}_{n}\right)$.
\item $t_{1}=t_{2}$ is a formula if $t_{1},t_{2}$ have the same sort.
\end{enumerate}
It is polite to indicate the sort of a variable in $\forall,\exists$,
e.g. $\forall x:\mbox{lecturer}\exists y:\mbox{Sun}\left(\mbox{bough\ensuremath{t_{lecturer,Sun}}}\left(x,y\right)\right)$.


\subsection{Formal Specification of Programs}


\paragraph{Pre-condition}

Formula $A\left(x_{1},x_{2},\dots,x_{n}\right)$ such that any arguments
$\left(a_{1},a_{2},\dots,a_{n}\right)$ satisfy the pre-condition
iff $A\left(a_{1},a_{2},\dots,a_{n}\right)$ is true. If there is no
restrictions on arguments beyond type information, we write 'none'
or $\top$.


\paragraph{Post-condition}

Formula expressing intended value of a funtion in terms of arguments.


\paragraph{Lists}

We can define a signature suitable for lists of type {[}Nat{]}:

Constants:
\begin{itemize}
\item $\underbar{0},\underbar{1},\dots:\mbox{Nat}$
\end{itemize}
Relations:
\begin{itemize}
\item $<,\leq,>,\geq:\mbox{(Nat,Nat)}$
\end{itemize}
Functions:
\begin{itemize}
\item $+,-,\times:\mbox{(Nat,Nat)\ensuremath{\rightarrow}Nat}$
\item $[]:\mbox{[Nat]}$
\item $\mbox{cons}(:):\mbox{(Nat,[Nat])\ensuremath{\rightarrow}[Nat]}$
\item $++:\mbox{([Nat],[Nat])\ensuremath{\rightarrow}[Nat]}$
\item $\mbox{head}:\mbox{[Nat]\ensuremath{\rightarrow}Nat}$
\item $\mbox{tail}:\mbox{[Nat]\ensuremath{\rightarrow}[Nat]}$
\item $\#:\mbox{[Nat]\ensuremath{\rightarrow}Nat}$
\item $!!:\mbox{([Nat],Nat)\ensuremath{\rightarrow}Nat}$
\end{itemize}

Pre-conditions for functions on lists are usually $\top$ or sometimes involve checking the list is non-empty ($\#xs > 0$).

Post-conditions for functions on lists often involve:
\begin{enumerate}
\item Checking a property of a list and its length compared to the given list (e.g. $\#xs = \#ys \land \forall i:\mbox{Nat} \left(i < \#xs \rightarrow P\left(ys!!i\right)\right)$ where $ys = f(x)$).
\item Checking a property of an item and its presence in the given list (e.g. $\exists i:\mbox{Nat} \left(i < \#xs \land xs!!i = y\right) \land P\left(y\right)$ where $y = f(x)$).
\end{enumerate}

\section{Checking Validity}

We can use:
\begin{enumerate}
\item Truth tables - but not for predicate logic
\item Direct argument
\item Equivalences
\item Proof systems - e.g. natural deduction
\end{enumerate}

\subsection{Direct Argument}


\paragraph{Propositional Logic}
\begin{enumerate}
\item Take an arbitrary situation.
\item Prove that the formula is true in this situation. (Often this will
require the law of excluded middle - argument by cases).
\end{enumerate}

\paragraph{Predicate Logic}

To show the argument $A_{1},A_{2},\dots,A_{n}\vDash B$ is valid:
\begin{enumerate}
\item Consider any $M$ such that $M\vDash A_{1},M\vDash A_{2},\dots,M\vDash A_{n}$.
\item Show $M\vDash B$, e.g.:

\begin{enumerate}
\item $M\vDash\boldsymbol{\forall}x\left(B\left(x\right)\right)$: Consider
an arbitrary object $a$ in $\mbox{dom}\left(M\right)$. Show $M\vDash B\left(a\right)$.
\item $M\vDash\boldsymbol{\exists}x\left(B\left(x\right)\right)$: Consider
any object $b$ in $\mbox{dom}\left(M\right)$. Show $M\vDash B\left(b\right)$.
\end{enumerate}
\end{enumerate}

\subsection{Equivalences}

When using equivalences, you \emph{must} justify every step by stating
the equivalence you used. Remember you can work from either direction. Note that distributivity is often very helpful when applied backwards.

Be especially careful with $\land$ and $\lor$!


\paragraph{$\boldsymbol{\lnot}$}
\begin{enumerate}
\item $\lnot\top\equiv\bot$
\item $\lnot\bot\equiv\top$
\item $\lnot\lnot A\equiv A$
\item $\lnot\left(A\land B\right)\equiv\lnot A\lor\lnot B$ (De Morgan)
\item $\lnot\left(A\lor B\right)\equiv\lnot A\land\lnot B$ (De Morgan)
\end{enumerate}

\paragraph{$\boldsymbol{\land}$}
\begin{enumerate}
\item $A\land B\equiv B\land A$ (Commutativity)
\item $A\land A\equiv A$ (Idempotence)
\item $A\land\top\equiv A$
\item $\bot\land A\equiv\lnot A\land A\equiv\bot$
\item $\left(A\land B\right)\land C\equiv A\land\left(B\land C\right)$
(Associativity)
\item $A\land\left(B\lor C\right)\equiv\left(A\land B\right)\lor\left(A\land C\right)$
(Distributivity)
\item $A\land\left(A\lor B\right)\equiv A$ (Absorption)
\end{enumerate}

\paragraph{$\boldsymbol{\lor}$}
\begin{enumerate}
\item $A\lor B\equiv B\lor A$ (Commutativity)
\item $A\lor A\equiv A$ (Idempotence)
\item $\top\lor A\equiv\lnot A\lor A\equiv\top$
\item $A\lor\bot\equiv A$
\item $\left(A\lor B\right)\lor C\equiv A\lor\left(B\lor C\right)$ (Associativity)
\item $A\lor\left(B\land C\right)\equiv\left(A\lor B\right)\land\left(A\lor C\right)$
(Distributivity)
\item $A\lor\left(A\land B\right)\equiv A$ (Absorption)
\end{enumerate}

\paragraph{$\boldsymbol{\rightarrow}$}
\begin{enumerate}
\item $A\rightarrow A\equiv\top$
\item $\top\rightarrow A\equiv A$
\item $A\rightarrow\top\equiv\top$
\item $\bot\rightarrow A\equiv\top$
\item $A\rightarrow\bot\equiv\lnot A$
\item $A\rightarrow B\equiv\lnot A\lor B\equiv\lnot\left(A\land\lnot B\right)$
\item $\lnot$$\left(A\rightarrow B\right)\equiv A\land\lnot B$
\end{enumerate}

\paragraph{$\boldsymbol{\leftrightarrow}$}
\begin{enumerate}
\item $A\leftrightarrow B\equiv\left(A\rightarrow B\right)\land\left(B\rightarrow A\right)\equiv\left(A\land B\right)\lor\left(\lnot A\land\lnot B\right)\equiv\lnot A\leftrightarrow\lnot B$
\item $\lnot\left(A\leftrightarrow B\right)\equiv A\leftrightarrow\lnot B\equiv\lnot A\leftrightarrow B\equiv\left(A\land\lnot B\right)\lor\left(\lnot A\land B\right)$
\end{enumerate}

\paragraph{$\boldsymbol{\forall,\exists}$}
\begin{enumerate}
\item $\forall x\forall yA\equiv\forall y\forall xA$
\item $\exists x\exists yA\equiv\exists y\exists xA$
\item $\lnot\forall xA\equiv\exists x\lnot A$
\item $\lnot\exists xA\equiv\forall x\lnot A$
\item $\forall x\left(A\land B\right)\equiv\forall xA\land\forall xB$
\item $\exists x\left(A\lor B\right)\equiv\exists xA\lor\exists xB$
\end{enumerate}
For $\boldsymbol{A}$ in which $\boldsymbol{x}$ does not Occur Free:
\begin{enumerate}
\item $A\equiv\forall xA\equiv\exists xA$
\item $\exists x\left(A\land B\right)\equiv A\land\exists xB$
\item $\forall x\left(A\lor B\right)\equiv A\lor\forall xB$
\item $\exists x\left(A\rightarrow B\right)\equiv A\rightarrow\exists xB$
\item $\forall x\left(A\rightarrow B\right)\equiv A\rightarrow\forall xB$
\item $\exists x\left(B\rightarrow A\right)\equiv\forall xB\rightarrow A${*}
\item $\forall x\left(B\rightarrow A\right)\equiv\exists xB\rightarrow A${*}
\end{enumerate}
{*} Watch out for these two cases!


\paragraph{$\boldsymbol{=}$}
\begin{enumerate}
\item $t=t\equiv\top$
\item $t=u\equiv u=t$
\item \textbf{Leibniz Principle} If $A$ is a formula in which $x$ occurs
free and $y$ does not occur and $B$ is the formula obtained from
$A$ by replacing one or more free occurences of $x$ by $y$, then
$x=y\rightarrow\left(A\leftrightarrow B\right)\equiv\top$.
\end{enumerate}

\paragraph{Renaming Bound Variables}

The formula in which all bound occurrences of a variable and the respective
quantifiers are changed to a new variable is equivalent to the original
formula.


\subsection{Natural Deduction}

When using natural deduction, remember line numbers and reasoning
are required for \emph{every} step. You should take time to check
your answers.


\paragraph{$\boldsymbol{\land}$-Introduction}

\begin{tabular}{llr}
1 & $A$ & \tabularnewline
2 & $B$ & \tabularnewline
3 & $A\land B$ & $\land I\left(1,2\right)$\tabularnewline
\end{tabular}


\paragraph{$\boldsymbol{\land}$-Elimination}

\begin{tabular}{llr}
1 & $A\land B$ & \tabularnewline
2 & $A$ & $\land E\left(1\right)$\tabularnewline
3 & $B$ & $\land E\left(1\right)$\tabularnewline
\end{tabular}


\paragraph{$\boldsymbol{\lor}$-Introduction}

\begin{tabular}{llr}
1 & $A$ & \tabularnewline
2 & $A\lor B$ & $\land I\left(1\right)$\tabularnewline
3 & $B\lor A$ & $\land I\left(1\right)$\tabularnewline
\end{tabular}


\paragraph{$\boldsymbol{\lor}$-Elimination}

\begin{tabular}{|llr|llr|}
\multicolumn{1}{l}{1} & \multicolumn{5}{l}{$A\lor B$}\tabularnewline
\hline 
2 & $A$ & ass & 4 & $B$ & ass\tabularnewline
3 & $C$ &  & 5 & $C$ & \tabularnewline
\hline 
\multicolumn{1}{l}{6} & $C$ & \multicolumn{4}{r}{$\lor E\left(1,2,3,4,5\right)$}\tabularnewline
\end{tabular}


\paragraph{$\boldsymbol{\rightarrow}$-Introduction}

\begin{tabular}{|llr|}
\hline 
1 & $A$ & ass\tabularnewline
2 & $B$ & \tabularnewline
\hline 
\multicolumn{1}{l}{3} & $A\rightarrow B$ & \multicolumn{1}{r}{$\rightarrow I\left(1,2\right)$}\tabularnewline
\end{tabular}


\paragraph{$\boldsymbol{\rightarrow}$-Elimination}

\begin{tabular}{llr}
1 & $A\rightarrow B$ & \tabularnewline
2 & $A$ & \tabularnewline
3 & $B$ & $\rightarrow E\left(1,2\right)$\tabularnewline
\end{tabular}


\paragraph{$\boldsymbol{\leftrightarrow}$-Introduction}

\begin{tabular}{llr}
1 & $A\rightarrow B$ & \tabularnewline
2 & $B\rightarrow A$ & \tabularnewline
3 & $A\leftrightarrow B$ & $\leftrightarrow I\left(1,2\right)$\tabularnewline
\end{tabular}


\paragraph{$\boldsymbol{\leftrightarrow}$-Elimination}

\begin{tabular}{llr}
1 & $A\leftrightarrow B$ & \tabularnewline
2 & $A$ & \tabularnewline
3 & $B$ & $\leftrightarrow E\left(1,2\right)$\tabularnewline
\end{tabular} or %
\begin{tabular}{llr}
1 & $A\leftrightarrow B$ & \tabularnewline
2 & $B$ & \tabularnewline
3 & $A$ & $\leftrightarrow E\left(1,2\right)$\tabularnewline
\end{tabular}


\paragraph{$\boldsymbol{\lnot}$-Introduction}

\begin{tabular}{|llr|}
\hline 
1 & $A$ & ass\tabularnewline
2 & $\bot$ & \tabularnewline
\hline 
\multicolumn{1}{l}{3} & $\lnot A$ & \multicolumn{1}{r}{$\lnot I\left(1,2\right)$}\tabularnewline
\end{tabular}


\paragraph{$\boldsymbol{\lnot}$-Elimination /$\boldsymbol{\bot}$-Introduction}

\begin{tabular}{llr}
1 & $A$ & \tabularnewline
2 & $\lnot A$ & \tabularnewline
3 & $\bot$ & $\lnot E\left(1,2\right)$ or $\bot I\left(1,2\right)$\tabularnewline
\end{tabular}


\paragraph{$\boldsymbol{\bot}$-Elimination}

\begin{tabular}{llr}
1 & $\bot$ & \tabularnewline
2 & $A$ & $\bot E\left(1\right)$\tabularnewline
\end{tabular}


\paragraph{$\boldsymbol{\lnot\lnot}$-Elimination}

\begin{tabular}{llr}
1 & $\lnot\lnot A$ & \tabularnewline
2 & $A$ & $\lnot\lnot E\left(1\right)$\tabularnewline
\end{tabular}


\paragraph{Excluded Middle}

\begin{tabular}{llr}
1 & $A\lor\lnot A$ & lemma\tabularnewline
\end{tabular}


\paragraph{Proof by Contradiction}

\begin{tabular}{|llr|}
\hline 
1 & $\lnot A$ & ass\tabularnewline
2 & $\bot$ & \tabularnewline
\hline 
\multicolumn{1}{l}{3} & $A$ & \multicolumn{1}{r}{$PC\left(1,2\right)$}\tabularnewline
\end{tabular}


\paragraph{$\boldsymbol{\exists}$-Introduction}

\begin{tabular}{llr}
1 & $A\left(t/x\right)$ & \tabularnewline
2 & $\exists xA$ & $\exists I\left(1\right)$\tabularnewline
\end{tabular}


\paragraph{$\boldsymbol{\exists}$-Elimination}

\begin{tabular}{|llr|}
\multicolumn{1}{l}{1} & $\exists xA$ & \multicolumn{1}{r}{}\tabularnewline
\hline 
2 & $A\left(c/x\right)$ & ass\tabularnewline
3 & $B$ & \tabularnewline
\hline 
\multicolumn{1}{l}{4} & $B$ & \multicolumn{1}{r}{$\exists E\left(1,2,3\right)$}\tabularnewline
\end{tabular}


\paragraph{$\boldsymbol{\forall}$-Introduction}

\begin{tabular}{|llr|}
\hline 
2 & $c$ & $\forall I$ const\tabularnewline
3 & $A\left(c/x\right)$ & \tabularnewline
\hline 
\multicolumn{1}{l}{4} & $\forall xA$ & \multicolumn{1}{r}{$\forall I\left(1,2\right)$}\tabularnewline
\end{tabular}


\paragraph{$\boldsymbol{\forall}$-Elimination}

\begin{tabular}{llr}
1 & $\forall xA$ & \tabularnewline
2 & $A\left(t/x\right)$ & $\forall E\left(1\right)$\tabularnewline
\end{tabular}


\paragraph{$\boldsymbol{\forall}\rightarrow$-Elimination}

\begin{tabular}{llr}
1 & $\forall x\left(A\left(x\right)\rightarrow B\left(x\right)\right)$ & \tabularnewline
2 & $A\left(t/x\right)$ & \tabularnewline
3 & $B\left(t/x\right)$ & $\forall\rightarrow E\left(1,2\right)$\tabularnewline
\end{tabular}


\paragraph{Reflexivity}

\begin{tabular}{llr}
1 & $t=t$ & refl\tabularnewline
\end{tabular}


\paragraph{Substitution}

\begin{tabular}{llr}
1 & $A\left(t/x\right)$ & \tabularnewline
2 & $t=u$ & \tabularnewline
3 & $A\left(u/x\right)$ & sub$\left(1,2\right)$\tabularnewline
\end{tabular}


\paragraph{Symmetry}

\begin{tabular}{llr}
1 & $c=d$ & \tabularnewline
2 & $d=c$ & sym$\left(1\right)$\tabularnewline
\end{tabular}
\end{document}
